\section{プログラム評価}
\subsection{評価手法}
\label{sec:eval}
CFDでは計算点の数が膨大であり,計算の正誤や誤差を厳密に評価するのは困難である.
したがってプログラム,計算結果の正しさについて以下の点で検証する.
\begin{itemize}
  \item 粒子の大きさを変化させたとき発生する運動量$p$が一定である.
  \item 時間間隔を変化させたとき発生する運動量$p$が一定である.
\end{itemize}
\subsection{評価}
表\ref{tab:eval_case}にシミュレーション条件を示す.
また表\ref{tab:eval_res}に粒子の大きさ,時間間隔(計算条件)を変化させたときの運動量を示す.
表\ref{tab:eval_res}から時間間隔を変化したときの運動量$p$の標準偏差は0.09,
粒子直径を変化させたときの運動量$p$の標準偏差は0.06となり,計算条件によらず概ね一定であると言える.
以上からプログラムは正常に動作してると考えられる.
\begin{table}[h]
\caption{シミュレーション条件}
\label{tab:eval_case}
\centering
\begin{tabular}{rc}
\hline
シミュレート時間&0.12 \si{\second}\\
流体密度&1000 \si{\kilo\gram.\second^{-1}}\\
動粘性係数&$1.0\times10^{-6}$ \si{\metre^2\second}\\
圧力$P$&4000 \si{\pascal}\\
衝突係数&0.2\\
\hline
\end{tabular}
\end{table}
\begin{table}[h]
\caption{計算条件と運動量}
\label{tab:eval_res}
\centering
\begin{tabular}{ccc}
\hline
時間間隔$dt$ / \si{\milli\second}&粒子直径 / \si{\metre}&運動量$p$ / \si{\newton.\second}\\
\hline \hline
0.20&0.0050&1.66\\
0.30&0.0050&1.83\\
0.40&0.0050&1.69\\
0.20&0.010&1.54\\
0.20&0.0025&1.63\\
\hline
\end{tabular}
\end{table}