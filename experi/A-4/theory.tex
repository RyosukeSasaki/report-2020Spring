\section{理論}
\subsection{円管内を流れる流体\cite{rikougaku}}
円管内を流れる流体の流れは大きく層流と乱流に分類できる.
層流は流量が少ないとき発生し,流体は管壁に対して平行かつ層状に流れる.
一方で乱流は流量が多いとき発生し,流体は管壁に対して無秩序な軌道で流れる.
層流から乱流への遷移はレイノルズ数$Re$という無次元数に依存し,以下の式で与えられる.
\begin{align}
  \label{equ:レイノルズ}
  Re=\cfrac{U_bD}{\nu}
\end{align}
ここで$U_b$は断面平均流速, $\nu$は動粘性系数, $D$は管内径である.
レイノルズ数$Re$は無次元数であるため,様々な流れにおいて同様の数値を適用できる.
特に$Re\simeq2000~2300$を臨界レイノルズ数$Re_c$と呼び,層流と乱流の遷移が起きる.

また,層流における流速の勾配はポアズイユによって与えられ,中心から$r$の位置での
流速$u$は,管の内径$R$と長さ$L$,両端の圧力$P_1$, $P_2$,粘性係数$\mu$を用いて以下の式で与えられる.
\begin{align}
  \label{equ:ポアズイユ}
  u=\cfrac{P_1-P_2}{4\mu L}(R^2-r^2)
\end{align}
(\ref{equ:ポアズイユ})式から層流の流速は壁面上で0,管の中央で最大値をとる2次曲線に従い分布する.
\subsection{染料注入法について\cite{rikougaku}}
染料注入法とは流れの可視化の手法の一つである.
流れの中にわずかに染料を流すことで,流れの軌道を可視化することができる.
\subsection{LDVについて\cite{LDVLaser}}
LDVとはレーザー光の干渉を用いて流れの流速を測定する装置である.
2つのコヒーレントな光束を一点で交差させると,その交点では干渉縞が発生する.
その点に流体中のホコリなどの大きな粒子が通過すると,干渉縞が散乱する.
LDVではその散乱光を光電素子などで測定する.
干渉縞中を粒子が高速で通過すると,散乱光の時間幅は短くなり,遅く通過すると時間幅が長くなる.
干渉縞の間隔$\delta$は,レーザーの交差角を$2\theta$,波長を$\lambda$とすると
\begin{align}
  \delta=\cfrac{\lambda}{2\sin\theta}
\end{align}
なので,速度$U_c$で粒子が交点を通過した時に観測されるバースト信号の周波数$f$は
\begin{align}
  f=\cfrac{2U_c\sin\theta}{\lambda}
\end{align}
となる.
したがって検出された縞の周波数$f$と干渉縞の間隔$\delta$を用いて,流れの流速$U_c$は以下のように与えられる.
\begin{align}
  U=f\sigma
\end{align}