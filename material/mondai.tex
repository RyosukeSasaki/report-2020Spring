\section{問題点}
\subsection{オゾン層破壊}
塩素を含む特定フロン(クロロフルオロカーボン:CFC,ハイドロクロロフルオロカーボン:HCFC)は
大気中で安定であるため,分解されることなく対流によって成層圏まで運ばれる.
ここでこれらのフロンガスは紫外線を吸収して塩素ラジカルを発生し,これがオゾンと反応することによってオゾン層を破壊する.\cite{気象庁フロンによ62:online}
表\ref{tab:ozone}にいくつかのフロンガスのオゾン破壊係数を示す.オゾン破壊係数とはCFC-11単位質量あたりのオゾン層破壊効果を1として
各物質のオゾン層破壊効果を表した数値である.
1987年に採択されたモントリオール議定書により1996年時点でオゾン層破壊効果の大きいCFCは全廃され,比較的オゾン層破壊効果の小さいHCFCも段階的に生産量を削減している.
図\ref{fig:gl_o3_2017.png}にオゾン全量の経年変化を示す.80年代以降オゾン全量は減少傾向だったが,
CFCの全廃によりその傾向は見られなくなった.一方でオゾン全量は未だ1970年代の水準までは回復していない.
\begin{table}[h]
   \caption{代表的なフロンガスのオゾン破壊係数\cite{平成17年度オゾ26:online}}
   \label{tab:ozone}
   \centering
   \begin{tabular}{c|cc}
     \hline
     グループ&名称&オゾン破壊係数\\
     \hline \hline
     \multirow{3}{*}{CFC}&CFC-11&1.0\\
     &CFC-12&1.0\\
     &CFC-113&0.8\\
     \hline
     \multirow{3}{*}{HCFC}&HCFC-22&0.055\\
     &HCFC-123&0.02\\
     &HCFC-141b&0.11\\
     \hline
   \end{tabular}
\end{table}
\mfig[width=12cm]{gl_o3_2017.png}{世界のオゾン全量の経年変化\cite{気象庁フロンによ62:online}}
\subsection{温室効果}
オゾン層破壊効果のある特定フロンを代替するものとして,代替フロン(ハイドロフルオロカーボン:HFC)が冷媒などとして用いられるようになった.
しかし,代替フロンは非常に大きな温室効果があり,地球温暖化の促進が懸念される.表\ref{tab:gwp}にいくつかのフロンガスの地球温暖化係数を示す.
地球温暖化係数とはガス単位質量あたりの温室効果を二酸化炭素の温室効果を1として換算した量である.
平成30年度において,日本では$2.856\times10^6$ \si{\tonne}の$\mathrm{CO_2}$に相当するフロンガスが流出しており\cite{環境省漏えい量の66:online},
オゾン層破壊に並んで重大な懸念である.
\begin{table}[h]
   \caption{地球温暖化係数\cite{GWPpdf45:online}}
   \label{tab:gwp}
   \centering
   \begin{tabular}{c|cc}
    \hline
    グループ&名称&地球温暖化係数\\
    \hline \hline
    \multirow{3}{*}{CFC}&CFC-11&4750\\
    &CFC-12&10900\\
    &CFC-113&6130\\
    \hline
    \multirow{2}{*}{HCFC}&HCFC-22&1810\\
    &HCFC-123&77\\
    \hline
    \multirow{3}{*}{HFC}&HFC-23&14800\\
    &HFC-32&675\\
    &HFC-143a&1430\\
    \hline
   \end{tabular}
\end{table}