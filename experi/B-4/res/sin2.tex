\subsection{$f(x)=\sin^2x$の場合}
\subsubsection{結果}
表\ref{tab:sin2_res}に$A_2$で正規化したフーリエ係数,位相を示す.
またデジタイザからの入力波形と1次の波形を図\ref{fig:asset/sin2/wave1.png},入力波形と2次の波形を図\ref{fig:asset/sin2/wave2.png},
各次数の振幅を図\ref{fig:asset/sin2/Amp.png},各次数の位相を図\ref{fig:asset/sin2/phase.png}に示す.
\begin{table}[h]
   \caption{$\sin^2x$のフーリエ係数,位相}
   \label{tab:sin2_res}
   \centering
   \begin{tabular}{ccccc}
     \hline
     次数$n$&$a_n$&$b_n$&$A_n$&$\phi_n$\\
     \hline \hline
     0 & 2.0531 & 0.0000 & 1.0265 & 0.0000 \\
1 & -0.0085 & -0.0087 & 0.0121 & -2.3472 \\
2 & -1.0000 & 0.0010 & 1.0000 & 3.1406 \\
3 & 0.0203 & -0.0115 & 0.0233 & -0.5137 \\
4 & 0.0082 & -0.0201 & 0.0217 & -1.1818 \\
5 & -0.0054 & -0.0101 & 0.0115 & -2.0634 \\
6 & -0.0115 & -0.0017 & 0.0116 & -2.9970 \\
7 & -0.0068 & 0.0032 & 0.0075 & 2.7050 \\
8 & -0.0006 & 0.0070 & 0.0070 & 1.6498 \\
9 & 0.0006 & 0.0034 & 0.0034 & 1.4017 \\
10 & -0.0004 & 0.0018 & 0.0018 & 1.7971 \\
     \hline
   \end{tabular}
\end{table}
\begin{figure}[htbp]
  \begin{minipage}{0.5\hsize}
    \mfig[width=7cm]{asset/sin2/wave1.png}{$\sin^2x$の入力波形と1次の波形}
  \end{minipage}
  \begin{minipage}{0.5\hsize}
    \mfig[width=7cm]{asset/sin2/wave2.png}{$\sin^2x$の入力波形と2次の波形}
  \end{minipage} 
\end{figure}
\begin{figure}[htbp]
  \begin{minipage}{0.5\hsize}
    \mfig[width=7cm]{asset/sin2/Amp.png}{$\sin^2x$の各次数での振幅}
  \end{minipage}
  \begin{minipage}{0.5\hsize}
    \mfig[width=7cm]{asset/sin2/phase.png}{$\sin^2x$の各次数での位相}
  \end{minipage} 
\end{figure}
\newpage
\subsubsection{考察・課題}
$A\sin^2 x=A\frac{1+\cos (2x+\pi)}{2}$より,理論的にこの曲線は,定常成分$A_0=\frac{A}{2}$, 2次成分$A_2=\frac{A}{2}$, $\phi_2=\pi$のみの成分を持つことがわかる.
図\ref{fig:asset/sin2/Amp.png}から,実際に定常成分, 2次成分がほぼ等しい値を持ち,それ以外の値が非常に小さいことがわかる.
また$\phi_2$も$\pi$に近い値を取っている.