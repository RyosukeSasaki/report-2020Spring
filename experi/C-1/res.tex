\section{実験結果}
\subsection{溶液の調製}
表\ref{tab:res_siyaku}に各溶液の色を示す.
\begin{table}[h]
   \caption{各溶液の濃度と色の変化}
   \label{tab:res_siyaku}
   \centering
   \begin{tabular}{ccccccc}
     \hline
     \multirow{2}{*}{番号}&\multicolumn{3}{c}{体積 / \si{\milli L}}&\multirow{2}{*}{濃度 / \si{\gram.L^{-1}}}&\multirow{2}{*}{溶液の色}&\multirow{2}{*}{発色後の色}\\
     &初めの読み&終わりの読み&標準溶液の体積&&&\\
     \hline \hline
     ①&0.10&0.63&0.53&0.106&薄青色&青色\\
     ②&0.63&1.75&1.12&0.224&薄青色&深青色\\
     ③&1.75&3.20&1.45&0.290&薄青色&深青色\\
     ④&0.29&2.34&2.05&0.410&薄青色&深青色\\
     ⑤&2.34&4.75&2.51&0.502&薄青色&深青色\\
     対照液&-&-&-&-&無色&無色\\
     \hline
   \end{tabular}
\end{table}
\subsection{最大吸収波長の決定}
最大吸収波長は625 \si{\nano\meter}だった.
\subsection{検量線の作成}
表\ref{tab:kyuko}に各溶液の吸光度$\rm A$を示す.
\begin{table}[h]
   \caption{吸光度A}
   \label{tab:kyuko}
   \centering
   \begin{tabular}{cc}
     \hline
     溶液番号&吸光度A\\
     \hline \hline
     ①&0.099\\
     ②&0.209\\
     ③&0.285\\
     ④&0.390\\
     ⑤&0.476\\
     \hline
   \end{tabular}
\end{table}
%\appendix
\renewcommand{\thefigure}{\Alph{figure}}
\mfig[width=10cm]{asset/graph.png}{検量線(手書き)}
\renewcommand{\thefigure}{\arabic{figure}}
\setcounter{figure}{0}
\mfig[width=10cm]{asset/graph2.png}{検量線(Excel)}
図\ref{fig:asset/graph2.png}の最尤曲線は最小二乗法によるもので,直線は以下の式で与えられる.
\begin{align}
  \label{equ:saiyu}
  y=0.9526x
\end{align}
\subsection{黄銅中の銅の定量}
試料の質量は$25.6001-25.4673=1.328\times10^{-1}$ \si{\gram}であり,吸光度$A$は0.332であった.
図\ref{fig:asset/graph.png}をから読み取った試料の銅イオン濃度は0.344 \si{\gram.L^{-1}}であった.
一方で(\ref{equ:saiyu})式より,図\ref{fig:asset/graph2.png}での試料の銅イオン濃度は0.349 \si{\gram.L^{-1}}であった.
銅イオン濃度を用いて,黄銅釘中の銅の含有量 [\si{\milli\gram}],含有率 [重量\%]は以下の式で与えられる.
\begin{align}
  \label{equ:ryou}
  含有量\ [\si{\milli\gram}]&=銅イオン濃度\ [\si{\gram.L^{-1}}]\times25\ \si{\milli L}\times\cfrac{100\ \si{\milli L}}{10\ \si{\milli L}}\\
  \label{equ:ritu}
  含有率\ [重量\%]&=\cfrac{含有量\ [\si{\milli\gram}]}{釘の質量\ [\si{\milli\gram}]}
\end{align}
よって各グラフから求めた含有量,含有率は以下の通りである.
\begin{table}[h]
   \caption{銅の含有量と含有率}
   \label{tab:kekka}
   \centering
   \begin{tabular}{lcc}
     \hline
     グラフ&含有量 / \si{\milli\gram}&含有率 / 質量\%\\
     \hline \hline
     手書き&86.0&64.8\\
     Excel&87.1&65.6\\
     \hline
   \end{tabular}
\end{table}