\subsection{矩形波と三角波}
\subsubsection{矩形波の結果}
表\ref{tab:square_res}に正規化したフーリエ係数,位相を示す.
またデジタイザからの入力波形と1次から17次の波形を図\ref{fig:asset/square/wave1.png}〜図\ref{fig:asset/square/wave13.png},
各次数の振幅を図\ref{fig:asset/square/Amp.png},各次数の位相を図\ref{fig:asset/square/phase.png}に示す.
\begin{table}[htbp]
   \caption{矩形波のフーリエ係数,位相}
   \label{tab:square_res}
   \centering
   \begin{tabular}{ccccc}
     \hline
     次数$n$&$a_n$&$b_n$&$A_n$&$\phi_n$\\
     \hline \hline
      0 & 1.6271 & - & 0.8135 & - \\
      1 & -0.9999 & 0.0114 & 1.0000 & 3.1302 \\
      2 & -0.0080 & -0.0014 & 0.0081 & -2.9718 \\
      3 & 0.3328 & -0.0180 & 0.3333 & -0.0541 \\
      4 & 0.0080 & -0.0017 & 0.0082 & -0.2135 \\
      5 & -0.1991 & 0.0153 & 0.1997 & 3.0647 \\
      6 & -0.0081 & -0.0005 & 0.0081 & -3.0840 \\
      7 & 0.1415 & -0.0171 & 0.1425 & -0.1199 \\
      8 & 0.0081 & -0.0007 & 0.0081 & -0.0842 \\
      9 & -0.1095 & 0.0161 & 0.1107 & 2.9957 \\
      10 & -0.0085 & -0.0006 & 0.0086 & -3.0709 \\
     \hline
   \end{tabular}
\end{table}

\begin{figure}[htbp]
  \begin{minipage}{0.5\hsize}
    \mfig[width=7cm]{asset/square/wave1.png}{矩形波の入力波形と1次の波形}
  \end{minipage}
  \begin{minipage}{0.5\hsize}
    \mfig[width=7cm]{asset/square/wave5.png}{矩形波の入力波形と5次の波形}
  \end{minipage} 
\end{figure}

\begin{figure}[htbp]
  \begin{minipage}{0.5\hsize}
    \mfig[width=7cm]{asset/square/wave9.png}{矩形波の入力波形と9次の波形}
  \end{minipage}
  \begin{minipage}{0.5\hsize}
    \mfig[width=7cm]{asset/square/wave13.png}{矩形波の入力波形と13次の波形}
  \end{minipage} 
\end{figure}

\begin{figure}[htbp]
  \begin{minipage}{0.5\hsize}
    \mfig[width=7cm]{asset/square/Amp.png}{矩形波の各次数での振幅}
  \end{minipage}
  \begin{minipage}{0.5\hsize}
    \mfig[width=7cm]{asset/square/phase.png}{矩形波の各次数での位相}
  \end{minipage} 
\end{figure}
