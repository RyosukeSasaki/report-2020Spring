\section{結果・考察}
動粘性系数$\nu$は20 \si{\degreeCelsius}で
$\nu=1.004\times10^6$ \si{\metre^2.\second^{-1}}, 25 \si{\degreeCelsius}で
$\nu=0.893\times10^6$ \si{\metre^2.\second^{-1}}であることから,線形補間を用いて,
21 \si{\degreeCelsius}では$\nu=0.982\times10^6$ \si{\metre^2.\second^{-1}}となる.
\subsection{染料注入法による流れの可視化}
\subsubsection{結果}
図\ref{fig:sketch.jpg}に管内の流れの様子のスケッチを示す.また表に各レイノルズ数での平均流速$U_b$を示す.
ただし管路Aの内径は0.035 mである.
\mfig[width=8cm]{sketch.jpg}{各レイノルズ数での流れのスケッチ}
\begin{table}[htbp]
   \caption{各レイノルズ数での平均流速}
   \label{tab:染料注入法}
   \centering
   \begin{tabular}{cc}
     \hline
     レイノルズ数$R_e$&断面平均流速$U_b$ / \si{\metre.\second^{-1}}\\
     \hline \hline
     1400&$3.9\times10^{-2}$\\
     2300&$6.5\times10^{-2}$\\
     6500&$1.8\times10^5{-1}$\\
     \hline
   \end{tabular}
\end{table}
\subsubsection{考察}
レイノルズ数$Re$が臨界レイノルズ数$Re_c$より低いとき,流れは管壁に平行な直線であった.
$Re$が$Re_c$に近づくにつれ,流れは振動し,徐々に振幅が大きく,周期が速くなっている.
そして$Re$が$Re_c$を完全に超えたところでは染料の流れは途切れ途切れになり,また線上の軌跡は見られず染料は拡散した.
以上から,確かに臨界レイノルズ数$Re_c$の前後で層流と乱流が遷移していることがわかった.
\subsection{LDVによる流速測定}
表に各レイノルズ数での断面平均流速$U_b$,平均流速$U_c$, $u_{rms}$,乱流強度$u_{rms}/U_c$, $U_c/U_b$を示す.
また,図に各レイノルズ数での速度のヒストグラムを示す.
\begin{table}[htbp]
   \caption{LVDによる測定結果}
   \label{tab:LVD}
   \centering
   \begin{tabular}{cccccc}
     \hline
     $R_e$&断面平均流速$U_b$ / \si{\metre.\second^{-1}}&平均流速$U_c$ / \si{\metre.\second^{-1}}&$u_{rms}$ / \si{\metre.\second^{-1}}&$\cfrac{U_c}{U_b}$&$\cfrac{u_{rms}}{U_c}$\\
     \hline \hline
     1400 & 0.0859  & 0.146 & 0.002 & 1.70 & 0.0119 \\
2300 & 0.141  & 0.205 & 0.005 & 1.45 & 0.0251 \\
6500 & 0.399  & 0.544 & 0.010 & 1.36 & 0.0179 \\
     \hline
   \end{tabular}
\end{table}
\mfig[width=14cm]{1400.jpg}{速度の度数分布($Re=1400$)}
\mfig[width=14cm]{2300.jpg}{速度の度数分布($Re=2300$)}
\mfig[width=14cm]{6500.jpg}{速度の度数分布($Re=6500$)}
\subsubsection{考察}
\begin{description}
  \item[ヒストグラムについて]\mbox{}\\
  ヒストグラムとは度数分布を示すグラフで,縦軸にはある量が一定の範囲内に出現する回数(度数), 横軸にはその量を取っている. 
  ここでの量は流速であるので,ヒストグラムは複数回の流速測定の結果の分布を示している.
  したがって各レイノルズ数での測定値の最大値,最小値は表\ref{tab:maxmin}のようになる.
  \begin{table}[h]
     \caption{各レイノルズ数での測定値の最大値と最小値}
     \label{tab:maxmin}
     \centering
     \begin{tabular}{ccc}
       \hline
       $R_e$&最大値 / \si{\metre.\second^{-1}}&最小値 / \si{\metre.\second^{-1}}\\
       \hline \hline
       1400&0.150&0.140\\
       2300&0.191&0.218\\
       6500&0.574&0.515\\
       \hline
     \end{tabular}
  \end{table}
  \item[レイノルズ数と$U_c/U_b$, $u_{rms}/U_c$の関係]\mbox{}\\
  図\ref{fig:graph1.eps}にレイノルズ数$R_e$と$U_c/U_b$,図\ref{fig:graph2.eps}にレイノルズ数$R_e$と乱流強度$u_{rms}/U_c$の関係を示す.
  レイノルズ数が小さいとき(\ref{equ:ポアズイユ})式のポアズイユの法則から,管の中心と管壁周辺では大きな流速差があり,その平均$U_b$は中心での速度を$u^r$として$U_b\simeq u^r/2$となる.
  一方で,レイノルズ数が大きくなると乱流により管内部での速度勾配が消え,平均流速$U_b\simeq u^r$となる.
  LDVは管の中心での流速を測定していることから$U_c\simeq u^r$なので,
  $U_c/U_b$はレイノルズ数が$Re_c$より小さい時$U_c/U_b\simeq 2$,大きいとき$U_c/U_b\simeq 1$となると考えられる.
  図\ref{fig:graph1.eps}から実際に$Re < Re_c$で$U_c/U_b\simeq 2$であり,また$Re\rightarrow\infty$で$U_c/U_b\rightarrow 1$へ漸近していることがわかる.

  また,図\ref{fig:graph2.eps}から,乱流強度$u_{rms}/U_c$は臨界レイノルズ数$Re_c$付近でピークを持っていることがわかる.
  この理由として, $Re<Re_c$では層流が支配的であり,管中心での流量が安定しているため$u_{rms}$が低く,また$Re>Re_c$では平均流量が大きくなり,乱流よりも平均の流れが支配的になるからだと考えられる.
  \mfig[width=10cm]{graph1.eps}{レイノルズ数と$U_c/U_b$の関係}
  \mfig[width=10cm]{graph2.eps}{レイノルズ数と乱流強度$u_{rms}/U_c$の関係}
  \item[乱流強度を用いる理由]\mbox{}\\
  標準偏差を平均流速で除算することにより,乱流強度は無次元数になっている.
  値を無次元化したことにより,数値の実験系に対する依存を少なくし,様々な系での指標として用いられると考えられる. 
  \item[層流・乱流の利用]
  航空機や自動車,大気,海洋など自然界の多くの流れ場は乱流が支配的である.
  したがってそれらの流れ場を解析する上で,乱流を発生させる乱流風洞が必要である\cite{2002409}.
  乱流風洞では格子などを用いてカルマン渦を発生させ,意図的に乱流を作ることができる.
  一方で低速風洞などでは乱流強度が低く,層流が多い低乱風洞が必要になる\cite{1986KJ00001465850}.
  低乱風洞では金網などを用いて整流を行い,乱流強度を低減する.
  また,乱流は媒体の振動を伴うため大きな騒音を発生する,ためジェットエンジンなどではその抑制が研究されている\cite{1995354}.
  一方で,航空機などの飛翔体においては翼端で乱流を発生させると,流れの剥離を抑制し,失速性能を高められることがわかってる.
  よって今日の航空機ではボルテックスジェネレータなどを用いて意図的に乱流を発生させ,失速性能を高めている\cite{199322}.
  同様のことはゴルフボールのディンプルなどで経験的に知られていた.
\end{description}