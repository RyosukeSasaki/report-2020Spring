\section{初期状態・推力計算}
\label{sec:init}
3次元では計算量が非常に膨大になるため,ここでは図\ref{fig:init.png}のような平面の筒で水ロケットを近似する.
\begin{itemize}
    \item 壁面はダミー粒子3層と壁粒子3層から成る.
    \item 筒の直径は0.1 \si{\metre}とする.
    \item 水を加圧する圧力は1000 \si{\pascal}から4000 \si{\pascal}で500 \si{\pascal}刻みとする.
    \item 水は$0\leq{z}\leq0.1$の厚さで存在するものとする.
    \item 粒子の重さは$\cfrac{水の密度\rho\times 初期状態の水の量V}{水粒子数}$で求める.
\end{itemize}
また${y}\leq-0.01$となった粒子の運動量を各タイムステップにおける推力とし,その総和を水ロケットが生み出す運動量とする.
\section{境界条件}
境界条件として、Dirichlet境界条件とNeumann境界条件を課す.
Dirichlet境界条件とは基準となる液面での圧力を設定するものである.
具体的には,粒子数密度が一定以下の粒子が液面にあると判定し,この粒子に対する圧力を与える.
Neumann境界条件とは壁面付近での圧力勾配を0とするもので,粒子が壁面を透過しないことを課す.
具体的には,壁粒子の更に外側にダミー粒子を配置し,この粒子と近傍の粒子の圧力値を等しくすることで近似的にNeumann境界条件を課す.