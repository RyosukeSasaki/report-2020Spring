\section{考察}
\subsection{実験1}
\subsubsection{$\mathrm{He}$-$\mathrm{Ne}$レーザーの波長}
$\mathrm{He}$-$\mathrm{Ne}$レーザーの波長は$632.81646~\si{\nano\metre}$である\cite{kiso}.
図\ref{fig:graph3.eps}のように実験値と文献値には差があり,またその誤差率は$-3.0~\%$である.
レーザーは非常に鋭いスペクトルを持っているため,透過によりレーザーの波長が大きく変化したとは考えにくい.
また,明暗の変化の数え間違いは無いものとする.
よって,誤差は鏡$\mathrm{M_1}$の移動量に由来すると考えられる. $\mathrm{M_1}$を動かす機構は何らかのバックラッシがあることがわかっている.
したがって,ステージコントローラの表示よりも実際の移動量が少ないことが考えられる.仮にステージコントローラの表示より$0.0005~\si{\milli\metre}$実際の移動量が少なかったとすると,
実験値は$632.8~\si{\nano\metre}$になり,文献値との誤差率は$1.6\times10^{-3}\%$と非常に近い.またステージコントローラの最小表示桁数は$0.0001~\si{\milli\metre}$であり, $0.0005~\si{\milli\metre}$の不確かさは十分に考えうる.
このことから,マイケルソン干渉計による波長の測定においては鏡の移動量の僅かな差が結果を大きく変化させるため,鏡の移動量をより正確に測定することが重要であると考えられる.
\mfig[width=13cm]{graph3.eps}{実験値と文献値の比較}
\subsubsection{$\mathrm{M_1}$の移動方向と干渉縞の湧き出し}
\label{sec:con_wakidasi}
実験1では$L_1$を小さくしたときに干渉縞が消滅し,大きくしたときに干渉縞が湧き出した.
図\ref{fig:fig3.png}のように光路を直線上に展開して考える.ただし$L_1>L_2$と仮定した.図の位置に明縞があったとすると以下が成り立つ.
\begin{align*}
  \left|\sqrt{d^2+L_1^2}-\sqrt{d^2+L_2^2}\right|=m\lambda
\end{align*}
図\ref{fig:graph4.eps}にいくつかの$L_1$対してグラフの概形を示した.図からわかるように,$L_1$が小さくなると$d$の絶対値も小さくなっており,この仮定では$L_1$を小さくすることで干渉縞は消滅していることがわかる.
これは実験結果と一致しており,今回の実験では$L_1>L_2$だったと考えられる.
\mfig[width=10cm]{fig3.png}{光路を直線上にしたイメージ}
\mfig[width=12cm]{graph4.eps}{$L_1$と$d$の関係($L_2=1$)}
\subsubsection{マイケルソン干渉計の応用例}
マイケルソン・モーリーの実験において,この干渉計はエーテルの風の検出のために考案された.
当時,光はエーテルという媒質を媒介して伝達すると考えられていた.
宇宙空間に対して固定されていたエーテルの中を地球が運動することで,相対的にエーテルが動くため,自転のradial方向とaxial方向では光速が異なるはずだと考えられていた.
しかし実際には光速は全ての方向について一定であり,エーテル説は否定されることになった.

また,マイケルソン干渉計は重力波観測にも用いられている.ブラックホールなど,大質量天体の連星が運動する際に生じる重力波によって光路長が変化するため,これを測定することで重力波を検出する.
\subsubsection{波長の測定方法}
波長の測定方法は以下のようなものがある.
\begin{itemize}
  \item プリズムに目的の光を入射し,屈折角を測定することで波長を算出できる.
  \item 回折格子やニュートンリングなど,マイケルソン干渉計以外の装置で干渉を起こす.
  \item 波長が十分に短くエネルギーが大きい場合,光電子のエネルギーを測定することで波長を算出できる.
\end{itemize}
\subsection{実験2}
\subsubsection{屈折率$n$}
\ref{sec:res_2}の結果から,
\begin{align*}
  n-1=\alpha P=0.000248\pm0.000015
\end{align*}
である.一方, Edlenの実験式から
\begin{align*}
  n-1=0.0002711
\end{align*}
となり,誤差率は$8.5~\%$である.
光学セルの厚さは静的でありノギスを用いた測定が容易なため,誤差は発生しにくいと考えられる.
一方で,圧力は常に減少し続けるため,実際の値よりも圧力差が大きくなることが考えられる.
動画から,調節弁を開けたときには$50~\si{\mmHg.\second^{-1}}$程度の速度で圧力は減少していた.
仮に誤差が全て圧力の測定に起因するものであるとすると,圧力差の真値は$89.1~\si{\mmHg}$程度である.
実験値の平均は$97.3~\si{\mmHg}$であり,真値との差は$8.2~\si{\mmHg}$である.
圧力が減少する速度が一定であるならばこの差は$0.16~\si{\second}$に相当し,操作の遅延によりこの程度の誤差が生じうると考えられる.

また,空気の屈折率の文献値は波長$\lambda=0.65~\si{\micro\metre}$にて$1.0002767$程度であり\cite{rika},妥当である.
\subsubsection{圧力の減少と干渉縞の湧き出し}
\ref{sec:con_wakidasi}項から,$L_1$の光路長が短くなると干渉縞が消滅する.
実際,圧力が減少すると屈折率が低下し,光路長が短くなることから,圧力の低下に伴い干渉縞が消滅するのは妥当である.
\subsubsection{空気の屈折率の測定方法}
Lorentz-Lorenzの法則を用いずに空気の屈折率の測定方法は以下のようなものが考えられる.
\begin{itemize}
  \item 屈折率が既知の物質,あるいは真空から空気に光を入射し,屈折角を精密に測定する.
  \item 空気の組成とその分極率から計算する.
\end{itemize}
