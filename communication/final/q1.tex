\section{課題1}
インターネットの階層化はインターネットアーキテクチャと呼ばれる.
階層は5層からなり,上位レイヤーほど抽象度が高い.
各レイヤーは上位からアプリケーション層,トランスポート層,ネットワーク層,リンク層,物理層と呼ばれる.
\mfig[width=4cm]{layer.png}{インターネットアーキテクチャの階層}
\subsection{物理層・リンク層}
物理層ではケーブルやコネクタの形状,符号化方式などネットワークの電気的,物理的要素を定義する.
リンク層では媒体アクセス制御や論理リンク制御を行う.
\subsubsection{IEEE802.3 CSMA/CD}
IEEE 802.3はLANなどで用いられるプロトコルでEthernetとして知られる.
CSMA/CDはEthernetで用いられる媒体アクセス制御の方式であり,
CSMA with CollisionDetectの意である.搬送波検知に加えて衝突検知を行い,高い効率を達成する.
初期の10Base5などではリピータとケーブルタップを用いたバス型ネットワークが用いられていたが,
100Base-Tx以降はスイッチングハブを用いたスター型ネットワークになった.
\subsubsection{IEEE802.11 Wireless LAN}
IEEE802.11はLANで用いられるプロトコルで無線LAN, Wi-Fiとして知られる.
無線LANでは2.4GHz, 5GHzの周波数帯を用いている.
無線LANの媒体アクセス制御はCSMA/CA方式である.
CSMA/CAではCSMA/CD同様に搬送波検知を行うが,衝突検知は不可能である.
したがって衝突回避(CollisionAvoidance)を行う.
衝突回避では,他ノードの通信終了後,一定時間+ランダムなバックオフ時間を待ってから通信を開始する.
ここで一定の待ち時間をDIFSと呼び,組み込み機器などの性能が低い機器が通信できるよう設定されている\cite{sunahara}.
\subsection{ネットワーク層}
ネットワーク層では論理アドレスの定義や経路制御を行い,パケットの伝送を行う.
\subsubsection{IP}
現在のインターネットにおける事実上の標準.
IP層において,各ホスト(ルータ)は論理的なIPアドレスを付与され,これに基づき通信を行う.
各ホストは経路制御表を持ち,宛先とそこ到達するための次のルータが示されている.
ルータはパケットを受け取ると,自身の経路制御表を元に次のルータへパケットを送信する.
\subsection{トランスポート層\cite{TCPとUDPの58:online}}
トランスポート層ではアプリケーション間での接続の確立を行う.データをパケットに分割しネットワーク層に渡す.
\subsection{TCP}
TCPではコネクション型の通信を提供する.
誤り回復や輻輳制御などを行うため信頼性が高いが,ヘッダサイズが大きくスループットは下がる.
HTTPやFTPなどのアプリケーションで用いられる.
\subsection{UDP}
UDPではコネクションレス型の通信を提供する.
誤り回復などは行わないため信頼性は低いが,スループットが高い.
NTPやSNMPなどのアプリケーションで用いられる.
\subsection{アプリケーション層}
アプリケーション内での通信の手順を定義する.
\subsection{FTP}
ファイル転送プロトコル(File Transfer Protocol)はファイル転送に用いられる.
FTPではデータ転送用と制御用の2本のコネクションを確立する.
制御用コネクションではディレクトリの変更や送受信の開始などの制御を行い,
データ転送用コネクションではデータの送受信を行う\cite{FTP(ファイル83:online}.
\subsection{NTP}
NTP(Network Time Protocol)はネットワーク上のコンピュータの時刻同期に用いられる.
NTPではネットワークの遅延や考慮して正確な時刻同期を行う.
NTPではサーバを階層構造で分類し,各層をstratumと呼ぶ.
原子時計やGPSに接続されたサーバをstratum0とし,
stratum0を参照するサーバをstratum1,
stratum1を参照するサーバをstratum2と時刻源から離れるにつれて階層が下がっていく\cite{インターネット用25:online}.
