\section{実験方法}
\subsection{使用試薬}
実験には以下の試薬を用いた.
\begin{table}[h]
   \caption{使用試薬}
   \label{tab:siyaku}
   \centering
   \begin{tabular}{ll}
    \hline
    番号&内容\\
    \hline \hline
     ①&$0.2M\ CuSO_4$,~~~~~~$0.5M\ Na_2SO_4$\\
     ②&$0.2M\ ZnSO_4$,~~~~~~$0.5M\ Na_2SO_4$\\
     ③&$0.15M\ CuSO_4$,~~~~~$0.5M\ Na_2SO_4$\\
     ④&$0.5M\ Na_2SO_4$\\
     ⑤&$0.05M\ H_2SO_4$\\
     \hline
   \end{tabular}
\end{table}
\subsection{実験操作}
\subsubsection{Daniell電池の起電力測定}
ビーカー,電池容器などの器具を適切に共洗いした.その後電池容器の各セルに①,②を入れた.
Cu電極, Zn電極をそれぞれ濡らしたエメリー紙で金属光沢があらわれるまで研磨した.
各電極を洗浄し,その後⑤に30秒間浸け不純物を溶解し,その後に再び洗浄した.
次に各電極を電池容器にセットし, 1分ごとに起電力を測定した.測定は5分間行った.
測定前後で電極表面の様子を観察した.
\subsubsection{溶液の調製}
0.025M, 0.05M, 0.1M $CuSO_4$水溶液を調整した.
\begin{itemize}
  \item ホールピペットで①を50 \si{\milli L}はかり取り, 100 \si{\milli L}メスフラスコに入れ,標線まで④を加え, 0.1M $CuSO_4$溶液(⑥)を調整した.
  \item ホールピペットで①を25 \si{\milli L}はかり取り, 100 \si{\milli L}メスフラスコに入れ,標線まで④を加え, 0.05M $CuSO_4$溶液(⑦)を調整した.
  \item 調製した0.1M 溶液をホールピペットで25 \si{\milli L}はかり取り, 100 \si{\milli L}メスフラスコで標線まで④を加え, 0.025M $CuSO_4$溶液(⑧)を調整した.
\end{itemize}
\subsection{濃淡電池の起電力測定}
電池容器の一方に⑥を入れ, もう一方に⑧,⑦,③,①を入れてそれぞれについて起電力を測定した.起電力は1分ごとに測定し,最低で5分間,連続する2回の測定結果が0.1 \si{\milli\volt}以内になるまで測定した.
測定毎に両方の電極を3.2.1と同様の手順で研磨した.また,測定毎に電極表面の様子を観察した.