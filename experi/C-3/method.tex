\section{実験方法}
\subsection{使用試薬}
実験には以下の試薬を用いた.
\begin{table}[h]
   \caption{使用試薬}
   \label{tab:siyaku}
   \centering
   \begin{tabular}{l|c}
     \hline
     名称&濃度\\
     \hline \hline
     特級NaOH&粒状\\
     酢酸溶液&1M\\
     HCl標準溶液(f=1.003)&0.1M\\
     標準pH液(pH7)&-\\
     標準pH液(pH4)&-\\
     フェノールフタレイン溶液(pp溶液)&-\\
     メチルオレンジ溶液(MO溶液)&-\\
     \hline
   \end{tabular}
\end{table}
\subsection{実験操作}
\subsubsection{溶液の調製}
\renewcommand{\theenumi}{\ajLabel\ajMaru{enumi}}
\begin{enumerate}
  \item 特級NaOH 2.04\si{\gram}を手早く計量し,メスシリンダーではかり取った水500 \si{\milli L}と混ぜ,\\
  0.1 M 水酸化ナトリウム水溶液を調整した.これはポリ試薬瓶に保存した.
  \item 1M 酢酸溶液をホールピペットで10 \si{\milli L}はかり取り,200 \si{\milli L}メスフラスコに移し,標線まで水を加え0.05M 酢酸水溶液を調整した.
  これはポリ試薬瓶に保存した.
\end{enumerate}
\subsubsection{pHメーターの校正}
pH4,\ pH7標準溶液をそれぞれ標準溶液用ビーカーに適量注いだ.その後pHメーターの電極を洗浄し,CALキーを押して校正モードにした.
電極をpH7標準溶液に浸けてから, STARTキーを押し校正した.電極を洗浄した後同様の手順でpH4標準溶液を用いて校正した.
\subsubsection{水酸化ナトリウムのファクター決定}
3.2.1で調製した0.1M 水酸化ナトリウム水溶液でビュレットを共洗いし,漏斗を用いて水酸化ナトリウム水溶液を注いだ.
0.1M HCl標準溶液をホールピペットで10 \si{\milli L}はかり取り,300 \si{\milli L}ビーカーに入れ,水を約90\si{\milli L}加え全体を約100 \si{\milli L}とした.また,塩酸にpp溶液を2,3滴加えておく.

上記の300 \si{\milli L}ビーカーをスターラーに設置し,pHメーターの電極を入れ滴定を開始した.
最初の9 \si{\milli L}は1 \si{\milli L}ずつ,その後0.1 \si{\milli L}ずつ滴下し,滴下量12.09 \si{\milli L}以降から再び1 \si{\milli L}ずつ滴下した.
水酸化ナトリウムを滴下するごとにpHと滴下量を記録した.
\subsubsection{酢酸の滴定}
3.2.1で調製した0.05M 酢酸水溶液をホールピペットで20 \si{\milli L}はかり取り,300 \si{\milli L}ビーカーに入れ,水を約80 \si{\milli L}加え全体を約100 \si{\milli L}とした.また,MO溶液、pp溶液をそれぞれ2,3滴加えておく.

上記の酢酸水溶液に対して3.2.3と同様に滴定を行った.最初の9 \si{\milli L}は1 \si{\milli L},その後10.92 \si{\milli L}まで0.1 \si{\milli L},その後再び1 \si{\milli L}ずつ滴下した.
3.2.3と同様にpHと滴下量を記録した.