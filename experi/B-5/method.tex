\section{実験方法}
\subsection{コンデンサーの作製}
2枚の金属板の間に紙を挟むことでコンデンサーとした.紙の厚みをマイクロメーターで,金属板の一辺の長さを定規で測定した.
また,極板間距離$d$が小さくなるように煉瓦で圧縮した.作成したコンデンサーのキャパシタンスをDMTを用いて測定した.
\subsection{RLC回路の作製}
図\ref{fig:fig2.jpg}のようなRLC回路を組んだ.周波数発生器の電源電圧はインピーダンスの変化とともに全く異なった値を示すので,
並列に接続した電圧計の値を元に電圧を調整した.
\mfig[width=7cm]{fig2.jpg}{実験の回路\cite{rikougaku}}
\subsection{共振周波数の探索}
計算機を用いて$f_0=\frac{1}{2\pi\sqrt{LC}}$をあらかじめ計算し,大まかに周波数発生器の値を合わせておく.
その後$V_L=V_C$となるように周波数を変化させ, $f_0$を探索した.
また,周波数発生器の出力電圧が一定のもとでは,共振周波数において実際に測定される電圧が最低になるので,これを用いて探索することもできる.
\subsection{測定}
電源電圧,抵抗,インダクタンス,キャパシタンスの条件を固定し,周波数を変化させながら電流を測定することで共振曲線を作製した.