\subsection{減衰振動}
表\ref{tab:decay_res}に正規化したフーリエ係数,位相を示す.
またデジタイザからの入力波形と6次の波形を図\ref{fig:asset/decay/wave6.png},入力波形と15次の波形を図\ref{fig:asset/decay/wave15.png},入力波形と30次の波形を図\ref{fig:asset/decay/wave30.png},各次数の振幅を図\ref{fig:asset/decay/Amp.png},各次数の位相を図\ref{fig:asset/decay/phase.png}に示す.
\begin{table}[h]
   \caption{減衰振動のフーリエ係数,位相}
   \label{tab:decay_res}
   \centering
   \begin{tabular}{ccccc}
     \hline
     次数$n$&$a_n$&$b_n$&$A_n$&$\phi_n$\\
     \hline \hline
     0 & -0.1360 & - & 0.0680 & - \\
1 & -0.2485 & -0.0462 & 0.2528 & -2.9579 \\
2 & -0.2743 & -0.0415 & 0.2775 & -2.9915 \\
3 & -0.3126 & -0.0742 & 0.3213 & -2.9085 \\
4 & -0.4030 & -0.1319 & 0.4240 & -2.8252 \\
5 & -0.5422 & -0.3763 & 0.6600 & -2.5349 \\
6 & -0.0065 & -1.0000 & 1.0000 & -1.5773 \\
7 & 0.4272 & -0.3288 & 0.5390 & -0.6559 \\
8 & 0.2683 & -0.1100 & 0.2899 & -0.3892 \\
9 & 0.1948 & -0.0622 & 0.2045 & -0.3091 \\
10 & 0.1368 & -0.0262 & 0.1393 & -0.1889 \\
     \hline
   \end{tabular}
\end{table}

\begin{figure}[htbp]
  \begin{minipage}{0.5\hsize}
    \mfig[width=7cm]{asset/decay/wave6.png}{減衰振動の入力波形と6次の波形}
  \end{minipage}
  \begin{minipage}{0.5\hsize}
    \mfig[width=7cm]{asset/decay/wave15.png}{減衰振動の入力波形と15次の波形}
  \end{minipage} 
\end{figure}
\mfig[width=7cm]{asset/decay/wave30.png}{減衰振動の入力波形と30次の波形}

\begin{figure}[htbp]
  \begin{minipage}{0.5\hsize}
    \mfig[width=7cm]{asset/decay/Amp.png}{減衰振動の各次数での振幅}
  \end{minipage}
  \begin{minipage}{0.5\hsize}
    \mfig[width=7cm]{asset/decay/phase.png}{減衰振動の各次数での位相}
  \end{minipage} 
\end{figure}
\subsubsection{考察・課題}
図\ref{fig:asset/decay/wave6.png},図\ref{fig:asset/decay/wave15.png}のように,曲線の始端と終端付近で入力データと合成波の
ズレが大きくなっている.これは本来$[0,\infty)$周期である減衰振動を$[0,2\pi]$周期として計算したために生じている.
確かに,図\ref{fig:asset/decay/wave6.png}では始端と終端がつながっている様子がわかる.