\section{実験原理}
ガスクロマトグラフ法の実験装置を図\ref{fig:GC.png}に示す.
\mfig[width=7cm]{GC.png}{ガスクロマトグラフ法の装置\cite{BN05476353}}
分離カラムは直径3〜6mm程度の金属管で,試料を溶融する不揮発性液体Aを塗布した固体粒子(固定相)が詰められている.
分離カラムは恒温槽に導入され試料は注入されると直ちに揮発する.
分離カラムには常に一定の流速でAに溶解しないガス(移動相)が流されており,これをキャリアガスと呼ぶ.
ここで試料注入口から少量の試料Bを注入すると,Bの蒸気は直ちにAに溶解する.
しかそ直後にBを含まないキャリアガスに接するためBは直ちに揮発し移動するが,わずかに進んだところで再びAに溶解する.
Bはこれを繰り返しながら分離カラムから脱出する.ここで試料に溶解度の異なる物質$B_1,B_2,B_3...$が含まれていたとすると,
これらは異なる速度で溶解,揮発を繰り返すため次第に分離カラム内で分離していく.結果$B_1,B_2,B_3...$は異なる時間(保持時間)に分離カラムから脱出するため,検出器を用いてこれを検出できる.
保持時間は各物質固有であり,事前に保持時間を調べておくことで未知物質の定性分析をできる.
また,固定相Aとして蒸気に対して溶解度の異なる物質ではなくて,吸着度の異なる物質を用いることもできる.

検出器としては熱伝導検出器(TCD)が一般に用いられる.TCDはホイートストンブリッジ回路の2つの抵抗をキャリアガス,キャリアガス+試料蒸気で
冷却することにより構成され,キャリアガスと熱伝導率の異なる試料蒸気が流れると抵抗率が変化し,電位差が生じる.
これを測定することで,試料の存在を検出できる.故にガスクロマトグラフ装置の出力は時間-電圧グラフになる.
出力グラフの面積は,物質ごとに見れば物質量に比例するため,絶対検量線などを用いてグラフから物質量を測定できる.