\section{実験原理}
\subsection{ガスクロマトグラフ法の原理}
ガスクロマトグラフ法の実験装置を図\ref{fig:GC.png}に示す.
分離カラムは金属管に試料を溶融する不揮発性液体Aを塗布した固体粒子(固定相)が詰められた管である.
分離カラムは恒温槽に導入されており試料は注入されると直ちに揮発する.
分離カラムには常に一定の流速でAに溶解しないガス(移動相)が流されており,これをキャリアガスと呼ぶ.
ここで試料注入口から少量の試料Bを注入すると,Bの蒸気は直ちにAに溶解する.
しかし直後にBを含まないキャリアガスに接するためBは直ちに揮発し移動するが,わずかに進んだところで再びAに溶解する.
Bはこれを繰り返しながら分離カラムから脱出し,検出器によって検出される.
\subsection{定性分析}
試料に溶解度の異なる物質$B_1,B_2,B_3...$が含まれていたとすると,
これらは異なる速度で溶解,揮発を繰り返すため次第に分離カラム内で分離していく.結果$B_1,B_2,B_3...$は異なる時間(保持時間)に分離カラムから脱出するため,検出器を用いて異なるパルスを検出できる.
保持時間は各物質固有であり,事前に保持時間を調べておくことで未知物質の定性分析をできる.
\subsection{定量分析}
ピークの面積は成分の物質量とその熱伝導率に依存する.このことから絶対検量線法,補正面積百分率法などの手法で定量分析を行える.
絶対検量線法とは,物質量,ピーク面積をプロットすることで検量線を作成し,これを用いて物質量を直接算出する方法である.

補正面積百分率法とは,組成が既知な試料を用いて,各成分のピーク面積について標準物質のピーク面積との補正係数$F_i$を(\ref{equ:Fi})式で計算し\cite{rikougaku},
(\ref{equ:molper})式を用いて組成モル比を計算する手法である.
\begin{equation}
  \label{equ:Fi}
  F_i=\frac{標準物質のピーク面積}{成分iのピーク面積}\times\frac{成分iの物質量}{標準物質の物質量}
\end{equation}
\begin{equation}
  \label{equ:molper}
  成分iのモル比(\%)=\frac{A_i F_i}{\sum_i A_i F_i}\times100
\end{equation}
\subsection{検出器}
検出器としては熱伝導検出器(TCD)が一般に用いられる.TCDはホイートストンブリッジ回路の2つの抵抗をそれぞれキャリアガス,キャリアガス+試料蒸気で
冷却することで構成される.キャリアガスと熱伝導率の異なる蒸気が流れると温度が変化し抵抗値が変化することで電位差が生じる.
これを測定することで,試料の存在を検出できる.故にガスクロマトグラフ装置の出力は時間-電圧グラフになる.
\mfig[width=7cm]{GC.png}{ガスクロマトグラフ法の装置\cite{BN05476353}}