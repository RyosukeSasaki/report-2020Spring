\section{実験方法}
\subsection{実験1}
偏光板と検光板のみを光学台に設置し,検光板の角度$-100\si{\degree}\leq\theta\leq100\si{\degree}$に対する透過光の強度$I_n$を調べた.
ただし,偏光板と検光板の透過容易軸が平行になる角度を$\theta=0\si{\degree}$とする.
\subsection{実験2}
$\theta=0\si{\degree},\ 90\si{\degree}$について$\frac{\lambda}{4}$板の角度$-10\si{\degree}\leq\phi\leq100\si{\degree}$に対する透過光の強度$I_n$を調べた.
ただし,偏光板の透過容易軸と$\frac{\lambda}{4}$板の光軸が平行となる角度を$\phi=0\si{\degree}$とする.
\subsection{実験3}
$\phi=0\si{\degree},\ 15\si{\degree},\ 30\si{\degree},\ 45\si{\degree}$について, $0\si{\degree}\leq\theta\leq360\si{\degree}$
に対する透過光の強度$I_n$を調べた.