\subsubsection{パルス(狭)}
デジタイザからの入力波形と4次の波形を図\ref{fig:asset/pulse_narrow/wave4.png},入力波形と30次の波形を図\ref{fig:asset/pulse_narrow/wave30.png},各次数の振幅を図\ref{fig:asset/pulse_narrow/Amp.png},各次数の位相を図\ref{fig:asset/pulse_narrow/phase.png}に示す.
\begin{figure}[htbp]
  \begin{minipage}{0.5\hsize}
    \mfig[width=7cm]{asset/pulse_narrow/wave4.png}{パルス(狭)の入力波形と4次の波形}
  \end{minipage}
  \begin{minipage}{0.5\hsize}
    \mfig[width=7cm]{asset/pulse_narrow/wave30.png}{パルス(狭)の入力波形と30次の波形}
  \end{minipage} 
\end{figure}
\begin{figure}[htbp]
  \begin{minipage}{0.5\hsize}
    \mfig[width=7cm]{asset/pulse_narrow/Amp.png}{パルス(狭)の各次数での振幅}
  \end{minipage}
  \begin{minipage}{0.5\hsize}
    \mfig[width=7cm]{asset/pulse_narrow/phase.png}{パルス(狭)の各次数での位相}
  \end{minipage} 
\end{figure}