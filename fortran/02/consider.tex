\section{考察}
\subsection{シミュレーション1}
圧力$P$と運動量$p$の相関係数は0.971であり,強い正の相関が見られる.
このことから圧力が高ければ高いほどより多くの運動量を獲得できるとわかる.
一方で2500 \si{\pascal}と3000 \si{\pascal}の間で運動量の減少が見られ,完全な比例関係でないことも予測される.

図\ref{fig:res/pre_raw.eps}から条件1.1,条件1.3,条件1.7を取り出し,移動平均を掛けたものを図\ref{fig:consider/pre.eps}に示す.
図\ref{fig:consider/pre.eps}から圧力が高いほどピークの推力が高く,ピーク幅が狭いことがわかる.
直感的には圧力が高いほど高速で水が噴射され,すぐに水を消費しきるので,直感に合致する.
\mfig[width=10cm]{consider/pre.eps}{移動平均を掛けた推力}
\subsection{シミュレーション2}
水の量$V$と運動量$p$の相関係数は0.995であり,強い正の相関が見られる.
このことから水の量が多ければ多いほどより多くの運動量を獲得できるとわかる.
また水の量$V$と運動量$p$の関係は最小二乗法を用いて$y=1701x+0.06$の比例関係があると考えられる.

図\ref{fig:res/size_raw.eps}から条件2.1,条件2.6,条件2.11を取り出し,移動平均を掛けたものを図\ref{fig:consider/size.eps}に示す.
図\ref{fig:consider/size.eps}から水の量が多いほどピークが低くなっていることがわかる.
直感的には水の量が多いと同じ圧力でも水の速度が小さくなるので,直感に合致する.
\mfig[width=10cm]{consider/size.eps}{移動平均を掛けた推力}
\mfig[width=10cm]{consider/V_size.eps}{水の量と運動量の比例関係}