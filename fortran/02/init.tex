\section{初期状態・推力計算}
\label{sec:init}
3次元では計算量が非常に膨大になるため,ここでは図\ref{fig:init.eps}のような平面の筒で水ロケットを近似する.
\begin{itemize}
    \item 壁面はダミー粒子3層と壁粒子3層から成る.
    \item 筒の直径は0.1 \si{\metre}とする.
    \item 水を加圧する圧力は1000 \si{\pascal}, 2000 \si{\pascal}, 3000 \si{\pascal}, 4000 \si{\pascal}とする.
    \item 水粒子は筒の中の$0\leq{y}\leq0.1$, $0\leq{y}\leq0.2$, $0\leq{y}\leq0.3$に等間隔で分布する.また,水は$0\leq{z}\leq0.1$の厚さで存在するものとする.
    \item 粒子の重さは$\cfrac{水の密度\times 初期状態の水の体積}{水粒子数}$で求める.
\end{itemize}
また${y}\leq-0.01$となった粒子の運動量を各タイムステップにおける推力とし,その合計を$\Delta{v}$とする.
\section{境界条件}
境界条件として、Dirichlet境界条件とNeumann境界条件を課す.
Dirichlet境界条件とは基準となる液面での圧力を設定するものである.
具体的には,粒子数密度が一定以下の粒子が液面にあると判定し,この粒子に対する圧力を与える.
Neumann境界条件とは壁面付近での圧力勾配を0とするもので,粒子が壁面を透過しないことを課す.
具体的には,壁粒子の更に外側にダミー粒子を配置し,この粒子と近傍の粒子の圧力値を等しくすることで近似的にNeumann境界条件を課す.