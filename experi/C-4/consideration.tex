\section{考察}
\subsection{Daniell電池の起電力}
Cu, Znの標準電極電位の文献値\cite{rikougaku}はそれぞれ
$\phi_{Cu}=0.337 \si{\volt}$, $\phi_{Zn}=-0.763\ \si{\volt}$なので, Daniell電池の起電力$E_D$は
$E_D=\phi_{Cu}-\phi_{Zn}=1.100\ \si{\volt}$である.文献値は25 \si{\degreeCelsius}での値であるが実験値は22 \si{\degreeCelsius}での値であり近似しうると考えられる.
一方,実験値は表\ref{tab:kidenryoku}より$1.10154\ \si{\volt}$であった.
実験値の相対誤差は$0.14\ \%$であり\cite{buturi},非常によく一致している.
%一方測定値の不確かさは$0.00001\ \si{\volt}$であるので相対不確かさは$9.0\times10^{-4}\ \%$である.また
\subsection{Daniell電池の平衡定数及びGibbs energyの変化}
(\ref{equ:K_D})式からDaniell電池の平衡定数を求める.
\begin{align}
  \label{equ:K_Djikken}
  K_D&=\exp(\cfrac{2F}{RT}E_D)\\
  &=\exp(\cfrac{2\times9.64853\times10^4}{8.31446\times295}\times1.10154)=4.34\times10^{37}
\end{align}
%べき乗演算について相対不確かさの伝搬則が成り立ち,各物理定数が十分な桁数を持って計算されていると仮定した場合,$K_D$の相対不確かさは1.0程度である
文献値を用いて$E_D=1.100\ \si{\volt}$としたとき $K_D=1.61\times10^{37}$程度である.ただしこの値は25 \si{\degreeCelsius} (298 \si{\kelvin})での値であることに注意すべきである.
また,べき乗演算によって測定値の相対不確かさが大きくなっていると考えられる.

また標準生成Gibbs energyの変化$\Delta_rG^{\circ}$は(\ref{equ:gibbs})式で与えられる\cite{rikougaku}.
\begin{align}
  \begin{aligned}
  \label{equ:gibbs}
  \Delta_rG^{\circ}&=-RT \ln K_D\\
  &=-RT\times\cfrac{2F}{RT}E_D\\
  &=-2F\times E_D\\
  &=-2.13\times10^5\ \si{\joule.\mole^{-1}}
  \end{aligned}
\end{align}
一方で文献値を用いた場合は$\Delta_rG^{\circ}=-2.12\times10^5\ \si{\joule.\mole^{-1}}$であり,相対誤差は$0.4\ \%$である.

Daniell電池の平衡は以下のようになる.
\begin{equation}
  \label{equ:daniellheiko}
  Cu^{2+}+Zn \rightleftharpoons Cu+Zn^{2+}
\end{equation}
平衡定数が非常に大きいことから平衡は大きく右に偏ることがわかる.またGibbs energyはある系から取り出せる仕事の最大値である\cite{kagakunetu}.
故にDaniell電池は理想的には$1\ \si{\mole}$あたり$2.13\times10^5\ \si{\joule}$の仕事をしうる.
しかし実際には内部抵抗などによる損失があり, $100\ \%$の仕事を取り出すことはできない.

また,平衡定数が文献値と実験値で大きな差が出たのに対し, Gibbs energyはよく一致しており,温度依存が少ない,あるいは無いと考えられる.
\subsection{Nernstの式の確認}
\label{sec:Nernst}
図\ref{fig:graph.png}に濃淡電池のモル濃度比の自然対数と起電力の関係を示す.
\mfig[width=9cm]{graph.png}{濃淡電池のモル濃度比の自然対数と起電力の関係}
最小二乗法より,実験値の近似曲線は$y=12.583x-0.91510$の直線である.一方で理論曲線は(\ref{equ:noutan})式
に縦軸の単位が$\si{\milli\volt}$であることに注意して値を代入し$y=12.711x$とした.
定常的なバイアスを除いた場合,実験結果はNernstの式とよく一致するため, Nernstの式は正しいと考えられる.

傾きは相対誤差$1.0\ \%$程度で近い値を持っているのに対し,切片では定常的な誤差が生じている.ここで$E_C$の表式を改めて示す.
\begin{equation}
  \label{equ:E_C}
  E_C=\phi_2-\phi_1=E^{\circ}_{Cu^{2+}/Cu}-E^{\circ}_{Cu^{2+}/Cu}+\cfrac{RT}{2F}\ln \cfrac{a_2}{a_1}
\end{equation}
(\ref{equ:E_C})式から明らかなように,切片は$E^{\circ}_{Cu^{2+}/Cu}-E^{\circ}_{Cu^{2+}/Cu}=0$であり,本質的に切片は0である.
以下のことからこの誤差の原因は実験装置ではない.
\begin{itemize}
  \item 電極は実験毎に研磨されており,組成が異ならない限り定常的なバイアスが乗る原因にはなりえない.
  \item 電池に用いた溶液は0.2Mと0.15Mの2種類の標準溶液から調製されており,希釈の度合いも異なるため原因にはなりえない.
  \item 電気回路での電圧降下の場合,起電力が異なれば電流も異なるため一定のバイアスが乗ることを説明できない.またDMTを流れる電流は極めて小さいはずである.
  \item DMTは校正されており,更に$0.001\ \si{\milli\volt}$もの正確さを持っていることから$0.9\ \si{\milli\volt}$ものバイアスが乗るとは考えにくい.
\end{itemize}
以上を踏まえてこの定常的な誤差の原因は活量係数にあると考える.詳細な議論は\S\ref{sec:katuryo}にて行う.
\subsection{銅電極表面の変化}
電流が流れていないこと,正極,負極双方で金属光沢が失われていた事から,これが電極反応のみによるものでないのは明らかである.
各セルには$SO_4^{2-}$, $OH^{-}$などの陰イオンが存在する.これらの化合物,水酸化銅(II)や硫酸銅は青色である.
一方で電極は黒ずんでいたことから実際に生成した物は酸化銅(II)であると考えられる.
本来,組成が同じ金属では電池にはなりえないが,表面の不純物などにより局部電池を作る場合がある\cite{1961166}\cite{alma99403791704031}.
ここで以下のような反応が考えられる.
\begin{equation*}
  \label{equ:youkyoku}
  2OH^{-} \rightarrow H_2O + \cfrac{1}{2}O + 2e^{-}
\end{equation*}
\begin{equation*}
  \label{hukyoku}
  Cu^{2+} + 2e^{-} \rightarrow Cu
\end{equation*}
\begin{equation*}
  Cu + O \rightarrow CuO
\end{equation*}
このように局部電池で発生した原子状の酸素が銅と反応したことで酸化銅(II)が生じたことが考えられる.
また,亜鉛電極についても無色の$ZnO$が発生したと考えれば結果と一致する.
\subsection{活量係数について}
\label{sec:katuryo}
活量係数$\gamma_{\pm}$はDebye-H\"{u}ckelの式によって以下のように与えられる\cite{kiso}.
\begin{align}
  \label{debye}
  \gamma_{\pm}=(1+0.001mM_r)\cfrac{d_0}{d}y_{\pm}\\
  y_{\pm}=\exp(\cfrac{-|z_1z_2|A_{\gamma}\sqrt{I_m}}{1+a^{\circ}B_{\gamma}\sqrt{I_m}})\\
  I_m=\cfrac{1}{2}\sum m_iz_i^2
\end{align}
ここで$A_{\gamma}$, $B_{\gamma}$は溶媒に対する定数で,水においてはそれぞれ$0.5101$, $0.3285$である.
$m$, $M_r$はそれぞれ電解質の質量モル濃度,分子量であり, $d_0$, $d$はそれぞれ純溶媒と溶液の密度である.
また,$a^{\circ}$は陽イオンと陰イオンの最近接距離 [cm]であり, $m_i$, $z_i$はそれぞれ溶液中に存在する各イオンの質量モル濃度,電荷である.
ここでは$0.1M\ CuSO_4$, $0.5M\ Na_2SO_4$について計算してみる.ただし仮に$a_{\circ}=10^{-6}$, $\tfrac{d_0}{d}=0.8$と置いている.また,水は溶質を溶かしても体積は変化していないものとする.
\begin{align*}
  I_m&=\cfrac{1}{2}\left(\cfrac{0.1\cdot\left(64+96\right)\cdot0.1\cdot2^2+1.0\cdot23\cdot0.1\cdot1^2+0.5\cdot96\cdot0.1\cdot2^2}{100+0.1\cdot160\cdot0.1+0.5\cdot142\cdot0.1}\right)\\
  &=0.1283
\end{align*}
\begin{align*}
  y_{Cu^{2+}}&=\exp(\cfrac{-|-2\times2|\cdot0.5101\sqrt{0.128334}}{1+10^{-6}\cdot0.3285\sqrt{0.128334}})\\
  &=0.4814
\end{align*}
\begin{align*}
  \gamma_{Cu^{2+}}&=\left(1+0.001\cdot160\cfrac{0.1\cdot160\cdot0.1}{100+0.1\cdot160\cdot0.1+0.5\cdot142\cdot0.1}\right)\cdot0.8\cdot\gamma_{Cu^{2+}}\\
  &=0.386
\end{align*}
このように活量係数を計算できた.図\ref{fig:graph.eps}に各$CuSO_4$濃度における活量係数のグラフを示した.
\mfig[width=15cm]{graph.eps}{$CuSO_4濃度と活量係数の関係$}
濃淡電池の実験では0.025M, 0.05M, 0.15M, 0.2M $CuSO_4$溶液を用いたので図\ref{fig:graph.eps}より,活量係数$\gamma_2$は0.4101から0.3595の間になる.
ここで$E_C$の表式を活量係数を省略せずに表すと(\ref{equ:E_Ckaturyo})式のようになる.
\begin{equation}
  \label{equ:E_Ckaturyo}
  E_C=\cfrac{RT}{2F}\ln \cfrac{c_2}{c_1} + \cfrac{RT}{2F}\ln \cfrac{\gamma_2}{\gamma_1}
\end{equation}
故に$E_C$に対する活量係数の寄与は$\cfrac{RT}{2F}\ln \cfrac{\gamma_2}{\gamma_1}$となる. $\gamma_2$の範囲と$\gamma_1$の値から活量係数の寄与は
\begin{equation}
  \label{equ:hanni}
  7.67\times10^{-4} \leq \cfrac{RT}{2F}\ln \cfrac{\gamma_2}{\gamma_1} \leq 9.04\times10^{-4}
\end{equation}
となる. \S\ref{sec:Nernst}で求めた近似式を用いると,実験値と理想値の誤差は各点において表\ref{tab:gosa}のようになる.
\newpage
\begin{table}[h]
   \caption{各点における理想値と近似値の誤差}
   \label{tab:gosa}
   \centering
   \begin{tabular}{c|c}
     \hline
     $\ln \frac{c_2}{c_1}$&誤差 / \si{\volt}\\
     \hline \hline
     $-1.386$&$7.274\times10^{-4}$\\
     $-0.6931$&$8.537\times10^{-4}$\\
     $0.4055$&$9.127\times10^{-4}$\\
     $0.6931$&$1.041\times10^{-3}$\\
     \hline
   \end{tabular}
\end{table}
この結果は(\ref{equ:hanni})の範囲とよく一致しており,定常的な誤差が活量係数に起因するものであるとわかる.
このことから,実験に用いたDMTでは活量係数の寄与を検出できており,テキスト(18)式において活量をモル濃度で代用したことは実際には不適切であったと考えられる.

一方, Daniell電池の実験において$\gamma_{Cu^{2+}}=0.3595$, $\gamma_{Zn^{2+}}=0.3591$なので,
\begin{equation*}
  \cfrac{RT}{2F} \ln \cfrac{\gamma_Cu^{2+}}{\gamma_Zn^{2+}}=1.415\times10^{-5}
\end{equation*}
である.相対誤差は$1.2\times10^{-3}\ \%$程度であり,十分に無視できると考えられる.
$Cu$と$Zn$は原子量が近く,またイオンの電荷,モル濃度も等しいため活量係数も近い値になる.
またDaniell電池の起電力は濃淡電池に比べて大きく,活量係数の寄与の割合も小さくなる.

\subsection{課題}
\subsubsection{Daniell電池の起電力の温度依存性}
(\ref{equ:gibbs})式から以下を導ける
\begin{equation}
  \label{equ:hyoujunngibss}
  E_D=-\cfrac{\Delta_rG^{\circ}}{2F}
\end{equation}
また(\ref{equ:hyoujunngibss})式のGibbs energy変化はエンタルピー変化$\Delta H$,エントロピー変化$\Delta S$を用いて以下のように表せる\cite{jiten}.
\begin{equation}
  \label{equ:enen}
  E_D=-\cfrac{\Delta H - T\Delta S}{2F}
\end{equation}
このことからDaniell電池の起電力は温度依存性を示し,また温度が高いほうが起電力も高くなると考えられる.
\subsubsection{Fe-Zn電池}
Fe-Zn電池は以下の反応式で表される.
\begin{equation*}
  Fe^{2+}+Zn\rightleftharpoons Fe+Zn^{2+}
\end{equation*}
FeとZnの標準電極電位はそれぞれ-0.763 \si{\volt}, -0.440 \si{\volt}である.
活量が1だから起電力,平衡定数,標準生成Gibbs energyは(\ref{equ:ED})式, (\ref{equ:K_D})式, (\ref{equ:gibbs})式から
\begin{align*}
  E_D&=-0.440-(-0.763)=0.323\ \si{\volt}\\
  K_D&=\exp(\cfrac{2F}{RT}E_D)=8.42\times10^{10}\\
  \Delta_fG^{\circ}&=-2F\cdot E_D=-6.23\times10^4\ \si{\joule.\mole^{-1}}
\end{align*}
である.
\subsubsection{Al-Au電池}
Al-Au電池は以下の反応式で表される.
\begin{equation*}
  Au^{3+}+Al\rightleftharpoons Au+Al^{3+}
\end{equation*}
AlとAuの標準電極電位はそれぞれ-1.7 \si{\volt}, +1.42 \si{\volt}である.
活量が1だから起電力,平衡定数,標準生成Gibbs energyは(\ref{equ:ED})式, (\ref{equ:K_D})式, (\ref{equ:gibbs})式から
\begin{align*}
  E_D&=1.42-(-1.7)=3.12\ \si{\volt}\\
  K_D&=\exp(\cfrac{3F}{RT}E_D)=1.98\times10^{158}\\
  \Delta_fG^{\circ}&=-3F\cdot E_D=-9.03\times10^5\ \si{\joule.\mole^{-1}}
\end{align*}
である.
\subsubsection{Zn-Ag電池}
Zn-Ag電池は以下の反応式で表される.
\begin{equation}
  2Ag^+ + Zn \rightleftharpoons 2Ag + Zn^{2+}
\end{equation}
故にそれぞれの半反応式は
\begin{align}
  Ag^+ + e^- \rightleftharpoons Ag\\
  Zn^{2+} + 2e^{-} \rightleftharpoons Zn
\end{align}
ZnとAgの標準電極電位はそれぞれ-0.763 \si{\volt}, +0.799 \si{\volt}である.
ZnとAgの標準電極電位$\phi_{Zn}$, $\phi_{Ag}$は
\begin{align}
  \phi_{Zn}&=E^{\circ}_{Zn^{2+}/Zn}+\cfrac{RT}{2F}\ln a_{Zn^{2+}}\\
  \phi_{Ag}&=E^{\circ}_{Ag^{+}/Ag}+\cfrac{RT}{F}\ln a_{Ag^{+}}
\end{align}
である.ここで起電力$E_D$は
\begin{align}
  E_D&=\phi_{Ag}-\phi_{Zn}=E^{\circ}_{Ag^{+}/Ag}-E^{\circ}_{Zn^{2+}/Zn}+\cfrac{RT}{2F} \ln \cfrac{(a_{Ag^+})^2}{a_{Zn^{2+}}}
\end{align}
ZnとAgの標準電極電位はそれぞれ-0.763 \si{\volt}, +0.799 \si{\volt}である.
活量が1だから起電力,平衡定数,標準生成Gibbs energyは(\ref{equ:ED})式, (\ref{equ:K_D})式, (\ref{equ:gibbs})式及び平衡の式から
\begin{align*}
  E_D&=0.799-(-0.763)=1.56\ \si{\volt}\\
  K_D&=\exp(\cfrac{2F}{RT}E_D)=6.81\times10^{52}\ \si{L.\mole^{-1}}\\
  \Delta_fG^{\circ}&=-2F\cdot E_D=-3.01\times10^5\ \si{\joule.\mole^{-1}}
\end{align*}


