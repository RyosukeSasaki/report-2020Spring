\section{課題4}
以下の機能を持つ論理関数$F$, $G$の論理式を示せ.
\begin{description}
  \item[論理関数$F$]\mbox{}\\4本の信号線$x_3$, $x_2$, $x_1$, $x_0$を用いて0から15の2進数を表し,
10以上なら$F=1$,それ以外で$F=0$とする.
  \item[論理関数$G$]\mbox{}\\上の2進数が6以下なら$G=1$,それ以外で$G=0$とする.
\end{description}
\subsection{$F$について}
$F$の真理値表を表\ref{tab:q4_f}に示す.したがってカルノー図は図\ref{fig:q4_f.png}のようになる.
ここで赤いループは$x_3x_1$,青いループは$x_3x_2$なので
\begin{align*}
  F(x_3,x_2,x_1,x_0)=x_3x_1+x_3x_2
\end{align*}
となる.
\begin{table}[h]
\caption{$F$の真理値表}
\label{tab:q4_f}
\centering
\begin{tabular}{cccccc}
\hline
$x_3$&$x_2$&$x_1$&$x_0$&表す整数&$F$\\
\hline \hline
0 & 0 & 0 & 0 & 0 & 0 \\
0 & 0 & 0 & 1 & 1 & 0 \\
0 & 0 & 1 & 0 & 2 & 0 \\
0 & 0 & 1 & 1 & 3 & 0 \\
0 & 1 & 0 & 0 & 4 & 0 \\
0 & 1 & 0 & 1 & 5 & 0 \\
0 & 1 & 1 & 0 & 6 & 0 \\
0 & 1 & 1 & 1 & 7 & 0 \\
1 & 0 & 0 & 0 & 8 & 0 \\
1 & 0 & 0 & 1 & 9 & 0 \\
1 & 0 & 1 & 0 & 10 & 1 \\
1 & 0 & 1 & 1 & 11 & 1 \\
1 & 1 & 0 & 0 & 12 & 1 \\
1 & 1 & 0 & 1 & 13 & 1 \\
1 & 1 & 1 & 0 & 14 & 1 \\
1 & 1 & 1 & 1 & 15 & 1 \\
\hline
\end{tabular}
\end{table}
\mfig[width=4cm]{q4_f.png}{$F$のカルノー図}
\subsection{$G$について}
$F$の真理値表を表\ref{tab:q4_g}に示す.したがってカルノー図は図\ref{fig:q4_f.png}のようになる.
ここで赤いループは$\overline{x_3x_1}$,青いループは$\overline{x_3x_2}$,緑のループは$\overline{x_3}x_1\overline{x_0}$なので
\begin{align*}
  G(x_3,x_2,x_1,x_0)=\overline{x_3x_1}+\overline{x_3x_2}+\overline{x_3}x_1\overline{x_0}
\end{align*}
となる.
\begin{table}[h]
\caption{$G$の真理値表}
\label{tab:q4_g}
\centering
\begin{tabular}{cccccc}
\hline
$x_3$&$x_2$&$x_1$&$x_0$&表す整数&$G$\\
\hline \hline
0 & 0 & 0 & 0 & 0 & 1 \\
0 & 0 & 0 & 1 & 1 & 1 \\
0 & 0 & 1 & 0 & 2 & 1 \\
0 & 0 & 1 & 1 & 3 & 1 \\
0 & 1 & 0 & 0 & 4 & 1 \\
0 & 1 & 0 & 1 & 5 & 1 \\
0 & 1 & 1 & 0 & 6 & 1 \\
0 & 1 & 1 & 1 & 7 & 0 \\
1 & 0 & 0 & 0 & 8 & 0 \\
1 & 0 & 0 & 1 & 9 & 0 \\
1 & 0 & 1 & 0 & 10 & 0 \\
1 & 0 & 1 & 1 & 11 & 0 \\
1 & 1 & 0 & 0 & 12 & 0 \\
1 & 1 & 0 & 1 & 13 & 0 \\
1 & 1 & 1 & 0 & 14 & 0 \\
1 & 1 & 1 & 1 & 15 & 0 \\
\hline
\end{tabular}
\end{table}
\mfig[width=4cm]{q4_g.png}{$G$のカルノー図}