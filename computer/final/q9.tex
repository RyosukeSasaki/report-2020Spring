\section{課題9}
以下の機械語をMIPSアセンブリに逆アセンブルする.
no.1の命令はopコードが0x23なのでlw命令である.
no.2の命令はopコードが0x00, functが0x20なのでadd命令である.
no.3の命令はopコードが0x2bなのでsw命令である.
また0x08はt0レジスタ, 0x09はt1レジスタ, 0x12はs2レジスタである.
よってMIPSアセンブリはソースコード\ref{src:q9_mips}のようになる.
\begin{table}[h]
\caption{機械語}
\label{tab:q9_machine}
\centering
\begin{tabular}{c|c|c|c|c|c|c}
\hline
&&&&\multicolumn{3}{c}{address/immidiate}\\
no.&op&rs&rt&rd&shamt&funct\\
\hline \hline
1&0x23&0x09&0x08&\multicolumn{3}{c}{0x190}\\\hline
2&0x00&0x12&0x08&0x08&0x00&0x20\\\hline
3&0x2b&0x09&0x08&\multicolumn{3}{c}{0x190}\\
\hline
\end{tabular}
\end{table}
\begin{lstlisting}[caption=MIPSアセンブリコード, label=src:q9_mips]
lw $t0, 400($t1)
add $t0, $s2, $t0
sw $t0, 400($t1)
\end{lstlisting}