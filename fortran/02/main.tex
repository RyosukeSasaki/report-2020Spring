\documentclass[uplatex,a4j,11pt]{jsarticle}


\renewcommand{\thesection}{\arabic{section}.}
\bibliographystyle{jplain}

\renewcommand{\thesubsection}{\arabic{subsection}}

\usepackage{listings,jlisting}
\usepackage{url}

% 数式
\usepackage{amsmath,amsfonts}
\usepackage{bm}
\usepackage{siunitx}
% 画像
\usepackage[dvipdfmx]{graphicx}
\makeatletter
\def\fgcaption{\def\@captype{figure}\caption}
\makeatother
\newcommand{\mfig}[3][width=15cm]{
\begin{center}
    \includegraphics[#1]{#2}
\fgcaption{#3 \label{fig:#2}}
\end{center}
}


\begin{document}

\title{プログラミング実習 期末レポート}
\author{佐々木良輔}
\date{\today}
\maketitle

\section*{題目:粒子法による箱の中の水のシミュレーション}
\subsection{背景}
室内の空気の対流からプレートテクトニクスまで流体として扱われる現象は非常に多い。
こういった現象を理解し,解析するに当たり数値流体解析(CFD)は非常に有用である.
本研究ではCFDを用いて単純化された箱の中の水の動きをシミュレートする.
\subsection{ソルバーについて}
ソルバーには大きく分けて粒子法と格子法が存在するが,本研究では粒子法を用いる.
粒子法とは流体を多数の粒子の集合であるとし,
各粒子について計算を行うことで流体の動きをシミュレートする手法である.
一方で格子法とは事前に定められた計算領域を格子状の計算点に分割し,固定された計算点において計算を行う手法である.
格子法に比べて粒子法は,事前に解析領域を定める必要がない,飛散などの現象を再現できる,メッシュの切り方に依存しないなどの特徴がある.
一方で粒子法は各粒子の座標を変数として取り扱うため,メモリの使用量が増加する.

今回は非圧縮性流れを扱うためMPS (Moving Particle Semi-implicit) 法を用いる.\cite{ryuusi}\cite{ELEM13}
MPS法では流体の圧力を粒子数密度を用いて表し,これを一定とするように計算することで非圧縮性を課す.
検証に用いるモデルは図\ref{fig:pixt.jpg}のような簡単な箱の中での液体の挙動をシミュレーションする.
\mfig[width=10cm]{init.eps}{検証モデル(青点:流体粒子,灰点:ダミー粒子,黒点:壁粒子)\cite{ELEM13}}
水は非圧縮性流体とできるので, Navier-Stokes方程式は
\begin{align}
    \label{NSequ}
    \frac{\partial\bm{u}}{\partial t}+(\bm{u}\cdot\nabla)=-\cfrac{1}{\rho}\nabla p+\mu\nabla^2\bm{u}+\bm{f}
\end{align}
ここで$\bm{u}$は速度である.連続の式は
\begin{align}
    \label{cont1}
    \cfrac{\partial\rho}{\partial t}+\rho\nabla\cdot\bm{u}=0\\
    \label{cont2}
    \nabla\cdot\bm{u}=0
\end{align}
したがって
\begin{align}
    \label{NSequ2}
    \frac{\partial\bm{u}}{\partial t}=-\cfrac{1}{\rho}\nabla p+\mu\nabla^2\bm{u}+\bm{f}
\end{align}
と表される.

MPS法では粒子間相互作用モデルを用いる.粒子間相互作用モデルでは重み係数$w$を用いて粒子間の相互作用に重みをつける.
重み係数$w$を以下のように定義する.\cite{sibas}\cite{MPSriron}
\begin{align}
    w(r)=\begin{cases}
        \cfrac{r_e}{r}-1&0\leq r\leq r_e\\
        0&r_e<r
    \end{cases}
\end{align}
$r$は粒子間の距離である.これは粒子間距離が$r_e$以下の粒子同士が相互作用することを表している.
これを用いて粒子数密度$n_i$を以下のように定義する.
\begin{align}
    n_i=\sum_{i\neq j}w(|\bm{r}_i-\bm{r}_j|)
\end{align}
$n_i$が初期条件$n^0$と一致するようにすることで非圧縮性条件を課す.
粒子間相互作用モデルでは各演算子を以下のように離散化する.\cite{soba}
\begin{align}
    \label{grad}
    \langle\nabla f\rangle_i&=\cfrac{d}{n^0}\sum_{i\neq j}\left(\cfrac{f(\bm{x}_j)-f(\bm{x}_i)}{|\bm{x}_j-\bm{x}_i|^2}(\bm{x}_j-\bm{x}_i)w(|\bm{x}_j-\bm{x}_i|)\right)\\
    \label{div}
    \langle\nabla\cdot \bm{u}\rangle_i&=\cfrac{2d}{n^0}\sum_{i\neq j}\cfrac{\bm{u}\cdot(\bm{x}_j-\bm{x}_i)}{|\bm{x}_j-\bm{x}_i|^2}w(|\bm{x}_j-\bm{x}_i|)\\
    \label{lap}
    \begin{split}
    \langle\nabla^2 f\rangle_i&=\cfrac{2d}{\lambda n^0}\sum_{i\neq j}\left(f(\bm{x}_j)-f(\bm{x}_i)\right)w(|\bm{x}_j-\bm{x}_i|)\\
    where~\lambda&=\cfrac{\sum_{i\neq j}|\bm{x}_j-\bm{x}_i|^2w(|\bm{x}_j-\bm{x}_i|)}{\sum_{i\neq j}w(|\bm{x}_j-\bm{x}_i|)}
    \end{split}
\end{align}
ここで$d$は空間の次元であり,今回は$d=2$である.

MPS法はSemi-implicitとあるとおり,陰解法と陽解法を組み合わせて計算する.
陽解法とは,現在のタイムステップの値をのみを用いて次のタイムステップの値を決定する手法である.
一方では陰解法ではまず現在のタイムステップから次のベクトル場を仮に決定し,それに基づいて計算する.
その後,条件を用いて補正するのが陰解法である.
MPS法ではNavier-Stokes方程式の第2項(粘性項)と第3項(外力項)を陽的に解き,第1項(移流項)と連続の式を陰的に解く.

(\ref{grad})から(\ref{lap})式を用いて(\ref{cont2}), (\ref{NSequ2})式の離散化をする.
\begin{align}
    \cfrac{\Delta \bm{u}}{\Delta t}=\left[-\cfrac{1}{\rho}\nabla p\right]^{k+1}+\left[\mu\nabla^2\bm{u}\right]^k+\left[\bm{f}\right]^k\\
    \left[\nabla\cdot\bm{u}\right]^{k+1}=0
\end{align}
$[\phi]^k$は$\phi$のタイムステップ$k$時点での値を指す.移流項,連続の式は陰的に解くため$k+1$ステップである.
$\bm{u}$を移流項,粘性項と外力項成分に分け,それぞれ$\bm{u}^*$, $\bm{u}'$とする.
\begin{align}
    \label{itiko}
    \cfrac{\Delta \bm{u'}}{\Delta t}=\cfrac{\bm{u}^{k+1}-\bm{u}^*}{\Delta t}=\left[-\cfrac{1}{\rho}\nabla p\right]^{k+1}\\
    \label{nisannko}
    \cfrac{\Delta \bm{u^*}}{\Delta t}=\cfrac{\bm{u}^*-\bm{u}^{k+1}}{\Delta t}=\left[\mu\nabla^2\bm{u}\right]^k+\left[\bm{f}\right]^k
\end{align}
ここで$\bm{u}^*$は仮速度であり,移流項での補正を行っていない速度ベクトルである.
(\ref{nisannko})式において未知数は$\bm{u}^*$のみなので,計算することができる.
$u^*$から速度を計算した後,接近した粒子間に弾性衝突を仮定し,速度と位置を修正する.
これを行うことで,後の圧力計算において粒子数密度が以上に高くなるのを避け,時間間隔を大きくすることができる.
また(\ref{itiko})式に辺々$\nabla$をかけると
\begin{align}
    \cfrac{\langle\nabla\cdot\bm{u}\rangle^{k+1}-\langle\nabla\cdot\bm{u}\rangle^{*}}{\Delta t}=-\cfrac{1}{\rho}\langle\nabla^2p\rangle^{k+1}
\end{align}
ここで連続の式から$\langle\nabla\cdot\bm{u}\rangle^{k+1}=0$なので
\begin{align}
    \label{nabnabpr}
    \langle\nabla^2p\rangle^{k+1}=\rho\cfrac{\langle\nabla\cdot\bm{u}\rangle^{*}}{\Delta t}
\end{align}
となる.つぎに(\ref{cont1})式を離散化すると
\begin{align}
    \cfrac{\rho^*-\rho}{\Delta t}+\rho\langle\nabla\cdot\bm{u}\rangle^*&=0\\
    \langle\nabla\cdot\bm{u}\rangle^*&=-\cfrac{1}{\Delta t}\cfrac{\rho^*-\rho}{\rho}
\end{align}
ここで密度$\rho$が粒子数密度$n_i$と比例することを用いて
\begin{align}
    \langle\nabla\cdot\bm{u}\rangle^*\simeq-\cfrac{1}{\Delta t}\cfrac{n^*-n^0}{n^0}
\end{align}
これと(\ref{nabnabpr})式から
\begin{align}
    \label{nabnabpn}
    \langle\nabla^2p\rangle^{k+1}=-\cfrac{\rho^0}{\Delta t^2}\cfrac{n^*-n^0}{n^0}
\end{align}
以上と(\ref{grad})から(\ref{lap})式を用いることで,実際に計算機上で連立方程式として圧力を計算することができる.
よって(\ref{itiko})式, (\ref{nisannko})式が解け,タイムステップの計算が終了する.\cite{ryuusinyuumon}
\subsection{初期状態}
\label{sec:init}
初期状態は図\ref{fig:init.eps}のような箱の中の一部に流体が分布したような状態を考える.
具体的には以下のような条件とする.
\begin{itemize}
    \item 壁面はダミー粒子2層と壁粒子2層から成る
    \item 箱内部の大きさは$0\leq x\leq1$, $0\leq y\leq 0.6$
    \item 流体は$0\leq x\leq0.25$, $0\leq y\leq 0.5$に等間隔で存在
\end{itemize}
\subsection{境界条件}
境界条件として、Dirichlet境界条件とNeumann境界条件を課す.
Dirichlet境界条件とは基準となる液面での圧力を設定するもので,ここでは0 Paを基準圧力とする.
具体的には,粒子数密度が一定以下の粒子が液面にあると判定し,この粒子に対する圧力を0として計算する.
Neumann境界条件とは壁面付近での圧力勾配を0とするもので,粒子が壁面を透過しないことを課す.
具体的には,壁粒子の更に外側にダミー粒子を配置し,この粒子と近傍の粒子の圧力値を等しくすることで近似的にNeumann境界条件を課す.
\subsection{評価手法}
\label{sec:eval}
CFDでは計算点の数が膨大であり,誤差を厳密に定義するのは困難である.
したがって,プログラム・計算結果の正しさについて以下の点で検証する.
\begin{enumerate}
    \item 粒子の初期状態を描画し,これが意図したとおりであることを確認する.
    \item 図\ref{fig:init.eps}のような初期条件において,十分な時間が経過すれば流体が図\ref{fig:end.png}のように箱の中で安定な終状態になることが予想できる.数値計算の結果が実際にそうなることを確認する.
    \item 粒子の大きさを変えて計算を行い,同様な状態に収束することを確認する.
\end{enumerate}
\mfig[width=10cm]{end.png}{終状態}
\subsection{結果・考察}
各シミュレーション結果のgif画像を以下のリンクに示す.

\url{https://drive.google.com/drive/folders/1_8svElxSwZ7_AIx99lqpTLAQhdo7Z8aQ?usp=sharing}
\subsubsection{シミュレーション1}
\label{sec:sim1}
表\ref{tab:sim1}にパラメーターを図\ref{fig:sim1/out0000.eps}に初期状態を示す.
また,図\ref{fig:sim1/out0791.eps}から図\ref{fig:sim1/out0800.eps}に終状態の直前100ステップを示す.

図\ref{fig:sim1/out0000.eps}のように,初期状態は\ref{sec:init}節に示した条件を満たしている.
また,図\ref{fig:sim1/out0791.eps}から図\ref{fig:sim1/out0800.eps}より,
流体は図\ref{fig:end.png}のような終状態で安定な状態になっていることがわかる.
\begin{table}[h]
   \caption{パラメーター}
   \label{tab:sim1}
   \centering
   \begin{tabular}{cc}
     \hline
     名称&値\\
     \hline \hline
     時間間隔$h$&0.0005 \si{\second}\\
     タイムステップ数&8000\\
     粒子間隔(粒子直径)&0.025 \si{\metre}\\
     流体密度&1000 \si{\kilo\gram}\\
     弾性衝突係数$e$&0.2\\
     動粘性係数&$1.0\times10^{-6}$ \si{\metre^{2}.\second^{-1}}\\
     圧縮率&$4.5\times10^{-10}$ \si{\pascal^{-1}}\\
     \hline
   \end{tabular}
\end{table}
\mfig[width=10cm]{sim1/out0000.eps}{初期状態}
\begin{figure}[htbp]
    \begin{minipage}{0.5\hsize}
        \mfig[width=7cm]{sim1/out0791.eps}{タイムステップ:7910}
    \end{minipage}
    \begin{minipage}{0.5\hsize}
        \mfig[width=7cm]{sim1/out0792.eps}{タイムステップ:7920}
    \end{minipage} 
\end{figure}
\begin{figure}[htbp]
    \begin{minipage}{0.5\hsize}
        \mfig[width=7cm]{sim1/out0793.eps}{タイムステップ:7930}
    \end{minipage}
    \begin{minipage}{0.5\hsize}
        \mfig[width=7cm]{sim1/out0794.eps}{タイムステップ:7940}
    \end{minipage} 
\end{figure}
\begin{figure}[htbp]
    \begin{minipage}{0.5\hsize}
        \mfig[width=7cm]{sim1/out0795.eps}{タイムステップ:7950}
    \end{minipage}
    \begin{minipage}{0.5\hsize}
        \mfig[width=7cm]{sim1/out0796.eps}{タイムステップ:7960}
    \end{minipage} 
\end{figure}
\begin{figure}[htbp]
    \begin{minipage}{0.5\hsize}
        \mfig[width=7cm]{sim1/out0797.eps}{タイムステップ:7970}
    \end{minipage}
    \begin{minipage}{0.5\hsize}
        \mfig[width=7cm]{sim1/out0798.eps}{タイムステップ:7980}
    \end{minipage} 
\end{figure}
\begin{figure}[htbp]
    \begin{minipage}{0.5\hsize}
        \mfig[width=7cm]{sim1/out0799.eps}{タイムステップ:7990}
    \end{minipage}
    \begin{minipage}{0.5\hsize}
        \mfig[width=7cm]{sim1/out0800.eps}{タイムステップ:8000}
    \end{minipage} 
\end{figure}
\subsubsection{シミュレーション2}
シミュレーション2では粒子の大きさを変更して同様の条件でシミュレーションを行っている.
表\ref{tab:sim2}にパラメーターを図\ref{fig:sim2/out0000.eps}に初期状態を示す.
また,図\ref{fig:sim2/out0791.eps}から図\ref{fig:sim2/out0800.eps}に終状態の直前100ステップを示す.

図\ref{fig:sim2/out0000.eps}のように,初期状態は\ref{sec:init}節に示した条件を満たしている.
また,シミュレーション1と比較すると,粒子のサイズを変更しても同様に\ref{fig:end.png}のような終状態で安定な状態になっているとわかる.
以上から2つの粒子サイズにおける検証で\ref{sec:eval}節に示した条件を満たしており,このプログラムが正常に動作していると言える.
\begin{table}[h]
   \caption{パラメーター}
   \label{tab:sim2}
   \centering
   \begin{tabular}{cc}
     \hline
     名称&値\\
     \hline \hline
     時間間隔$h$&0.0005 \si{\second}\\
     タイムステップ数&8000\\
     粒子間隔(粒子直径)&0.020 \si{\metre}\\
     流体密度&1000 \si{\kilo\gram}\\
     弾性衝突係数$e$&0.2\\
     動粘性係数&$1.0\times10^{-6}$ \si{\metre^{2}.\second^{-1}}\\
     圧縮率&$4.5\times10^{-10}$ \si{\pascal^{-1}}\\
     \hline
   \end{tabular}
\end{table}
\mfig[width=10cm]{sim2/out0000.eps}{初期状態}
\begin{figure}[htbp]
    \begin{minipage}{0.5\hsize}
        \mfig[width=7cm]{sim2/out0791.eps}{タイムステップ:7910}
    \end{minipage}
    \begin{minipage}{0.5\hsize}
        \mfig[width=7cm]{sim2/out0792.eps}{タイムステップ:7920}
    \end{minipage} 
\end{figure}
\begin{figure}[htbp]
    \begin{minipage}{0.5\hsize}
        \mfig[width=7cm]{sim2/out0793.eps}{タイムステップ:7930}
    \end{minipage}
    \begin{minipage}{0.5\hsize}
        \mfig[width=7cm]{sim2/out0794.eps}{タイムステップ:7940}
    \end{minipage} 
\end{figure}
\begin{figure}[htbp]
    \begin{minipage}{0.5\hsize}
        \mfig[width=7cm]{sim2/out0795.eps}{タイムステップ:7950}
    \end{minipage}
    \begin{minipage}{0.5\hsize}
        \mfig[width=7cm]{sim2/out0796.eps}{タイムステップ:7960}
    \end{minipage} 
\end{figure}
\begin{figure}[htbp]
    \begin{minipage}{0.5\hsize}
        \mfig[width=7cm]{sim2/out0797.eps}{タイムステップ:7970}
    \end{minipage}
    \begin{minipage}{0.5\hsize}
        \mfig[width=7cm]{sim2/out0798.eps}{タイムステップ:7980}
    \end{minipage} 
\end{figure}
\begin{figure}[htbp]
    \begin{minipage}{0.5\hsize}
        \mfig[width=7cm]{sim2/out0799.eps}{タイムステップ:7990}
    \end{minipage}
    \begin{minipage}{0.5\hsize}
        \mfig[width=7cm]{sim2/out0800.eps}{タイムステップ:8000}
    \end{minipage} 
\end{figure}
\subsection{結論}
粒子法により箱の中の水の挙動をシミュレートできた.
\newpage
\bibliography{ref.bib}
\newpage
%ここからソースコードの表示に関する設定
\lstset{
  basicstyle={\ttfamily},
  identifierstyle={\small},
  commentstyle={\smallitshape},
  keywordstyle={\small\bfseries},
  ndkeywordstyle={\small},
  stringstyle={\small\ttfamily},
  frame={tb},
  breaklines=true,
  columns=[l]{fullflexible},
  numbers=left,
  xrightmargin=0zw,
  xleftmargin=3zw,
  numberstyle={\scriptsize},
  stepnumber=1,
  numbersep=1zw,
  lineskip=-0.5ex
}
%ここまでソースコードの表示に関する設定
\appendix
\section{ソースコード}
ソースコードは以下のレポジトリで管理されている.\\
\url{https://github.com/RyosukeSasaki/particle-fortran}
\begin{lstlisting}[caption=メインプログラム]
program main
  use consts_variables
  implicit none
  integer, parameter :: interval = 10
  integer :: i, j
  character :: filename*128
  character :: command*128
  call omp_set_num_threads(8)

  command = 'mkdir -p ' // dir // '; rm ' // dir // '/data*'
  call system(command) 

  filename = dir // "/thrust.dat"

  open (19, file=filename, status='replace')
  write (*, *) "write data to ", dir, "/. Pressure is ", WaterPressure

  call InitParticles()
  call calcConsts()
  call writeInit()
  call output(0)
  do i = 0, 240-1
  do j = 1, interval
    call calcGravity()
    call calcViscosity()
    call moveParticleExplicit()
    call collision()
    call calcNumberDensity()
    call setBoundaryCondition()
    call setSourceTerm()
    call setMatrix()
    call GaussEliminateMethod()
    call removeNegativePressure()
    call setMinPressure()
    call calcPressureGradient()
    call moveParticleImplicit()
    call detach(i*interval+j)
    call output(i*interval+j)
  enddo
  write (*, *) "Timestep: ", (i+1)*interval
  enddo

  write (19, *) "#deltaV: ", deltaV
  close (19)

end
\end{lstlisting}
\begin{lstlisting}[caption=ファイル出力用ルーチン]
subroutine output(i)
  use consts_variables
  implicit none
  integer :: i, k
  character :: filename*128

  write (filename, '("data2/data", i4.4, ".dat")') i
  open (11, file=filename, status='replace')
  do k = 1, NumberOfParticle
    write (11, '(f6.3,X)', advance='no') Pos(k, 1)
    write (11, '(f6.3,X)', advance='no') Pos(k, 2)
    write (11, '(I2,X)', advance='no') ParticleType(k)
    write (11, '(f15.5,X)', advance='no') Pressure(k)
    if (CollisionState(k) .eqv. .true.) then
      write (11, '(I2)', advance='no') 1
    else
      write (11, '(I2)', advance='no') 0
    endif
    write (11, *)
  enddo
  close (11)

end
\end{lstlisting}
\begin{lstlisting}[caption=定義値]
module define
  implicit none
  integer, parameter :: PARTICLE_DUMMY = -1, PARTICLE_WALL = 1, PARTICLE_FLUID = 0
  integer, parameter :: BOUNDARY_DUMMY = -1, BOUNDARY_INNER = 0
  integer, parameter :: BOUNDARY_SURFACE_LOW = 1, BOUNDARY_SURFACE_HIGH = 2

end
\end{lstlisting}
\begin{lstlisting}[caption=定数・変数の宣言]
  module consts_variables
  implicit none
  !initial value of particle distance
  real*8, parameter :: PARTICLE_DISTANCE = 0.005
  real*8, parameter :: FLUID_DENSITY = 1000
  real*8, parameter :: KINEMATIC_VISCOSITY = 1.0e-6
  !threshold of whether the particle is surface or inside
  real*8, parameter :: THRESHOLD_RATIO_BETA = 0.97
  !圧力計算の緩和係数
  real*8, parameter :: RELAXATION_COEF_FOR_PRESSURE = 0.2
  !圧縮率(Pa^(-1))
  real*8, parameter :: COMPRESSIBILITY = 0.45e-9
  real*8, parameter :: IMPACT_PARAMETER = 1.0d0
  integer, parameter :: MaxNumberOfParticle = 500000
  integer, parameter :: numDimension = 2

  !筒の大きさ
  real*8, parameter :: sizeX = 0.1d0, sizeY = 0.4d0
  
  real*8 :: Radius_forNumberDensity, Radius_forGradient, Radius_forLaplacian
  real*8 :: N0_forNumberDensity, N0_forGradient, N0_forLaplacian
  real*8 :: Lambda
  real*8 :: collisionDistance = 0.8*PARTICLE_DISTANCE
  real*8 :: MassOfParticle
  real*8 :: deltaV = 0
  
  integer :: NumberOfParticle
  integer :: NumberOfFluid
  
  real*8 :: dt = 0.0002d0
  character :: dir*6 = "data08"
  real*8, parameter :: WaterPressure = 3000
  !水領域の大きさ
  real*8, parameter :: WsizeX = 0.1d0, WsizeY = 0.1d0, WsizeZ = 0.1d0
  
  real*8 :: Pos(MaxNumberOfParticle, numDimension)
  real*8 :: Gravity(numDimension)
  ! 0:Fluid, 1:Wall, 2:Dummy
  integer :: ParticleType(MaxNumberOfParticle)
  data Gravity/0d0, -9.8d0/
  real*8, allocatable :: Vel(:, :)
  real*8, allocatable :: Acc(:, :)
  real*8, allocatable :: Pressure(:)
  real*8, allocatable :: MinPressure(:)
  real*8, allocatable :: NumberDensity(:)
  integer, allocatable :: BoundaryCondition(:)
  real*8, allocatable :: SourceTerm(:)
  real*8, allocatable :: CoefficientMatrix(:, :)
  logical, allocatable :: CollisionState(:)
  logical, allocatable :: detachState(:)

end

\end{lstlisting}
\begin{lstlisting}[caption=重み係数計算関数]
real*8 function calcWeight(distance, radius)
  implicit none
  real*8, intent(in) :: distance, radius
  real*8 :: w

  if (distance >= radius) then
    w = 0
  else
    w = radius/distance - 1d0
  endif
  calcWeight = w
  return

end function
\end{lstlisting}
\begin{lstlisting}[caption=粒子間距離計算関数]
real*8 function calcDistance(i, j)
  use consts_variables
  implicit none
  real*8 :: distance2
  integer :: i, j, k

  distance2 = 0d0
  do k = 1, numDimension
    distance2 = distance2 + (Pos(j, k) - Pos(i, k))**2
  enddo
  calcDistance = sqrt(distance2)
  return

end function
\end{lstlisting}
\section{初期状態・推力計算}
\label{sec:init}
3次元では計算量が非常に膨大になるため,ここでは図\ref{fig:init.png}のような平面の筒で水ロケットを近似する.
\begin{itemize}
    \item 壁面はダミー粒子3層と壁粒子3層から成る.
    \item 筒の直径は0.1 \si{\metre}とする.
    \item 水を加圧する圧力は1000 \si{\pascal}から4000 \si{\pascal}で500 \si{\pascal}刻みとする.
    \item 水は$0\leq{z}\leq0.1$の厚さで存在するものとする.
    \item 粒子の重さは$\cfrac{水の密度\rho\times 初期状態の水の量V}{水粒子数}$で求める.
\end{itemize}
また${y}\leq-0.01$となった粒子の運動量を各タイムステップにおける推力とし,その総和を水ロケットが生み出す運動量とする.
\section{境界条件}
境界条件として、Dirichlet境界条件とNeumann境界条件を課す.
Dirichlet境界条件とは基準となる液面での圧力を設定するものである.
具体的には,粒子数密度が一定以下の粒子が液面にあると判定し,この粒子に対する圧力を与える.
Neumann境界条件とは壁面付近での圧力勾配を0とするもので,粒子が壁面を透過しないことを課す.
具体的には,壁粒子の更に外側にダミー粒子を配置し,この粒子と近傍の粒子の圧力値を等しくすることで近似的にNeumann境界条件を課す.
\begin{lstlisting}[caption=定数計算関連ルーチン]
subroutine calcConsts()
  !定数の計算
  use consts_variables
  implicit none

  Radius_forNumberDensity = 2.1*PARTICLE_DISTANCE
  Radius_forGradient = 2.1*PARTICLE_DISTANCE
  Radius_forLaplacian = 3.1*PARTICLE_DISTANCE
  collisionDistance = 0.5*PARTICLE_DISTANCE

  call calcNZeroLambda()

end

subroutine calcNZeroLambda()
  use consts_variables
  implicit none
  real*8 :: calcWeight
  real*8 :: xj, yj, xi = 0d0, yi = 0d0
  real*8 :: distance2, distance
  integer :: iX, iY

  N0_forNumberDensity = 0d0
  N0_forGradient = 0d0
  N0_forLaplacian = 0d0
  Lambda = 0d0

  do iX = -4, 4
    do iY = -4, 4
      if ((iX == 0) .and. (iY == 0)) cycle
      xj = PARTICLE_DISTANCE*dble(iX)
      yj = PARTICLE_DISTANCE*dble(iY)
      distance2 = (xj - xi)**2 + (yj - yi)**2
      distance = sqrt(distance2)

      N0_forNumberDensity = N0_forNumberDensity + calcWeight(distance, Radius_forNumberDensity)
      N0_forGradient = N0_forGradient + calcWeight(distance, Radius_forGradient)
      N0_forLaplacian = N0_forLaplacian + calcWeight(distance, Radius_forLaplacian)

      Lambda = Lambda + distance2*calcWeight(distance, Radius_forLaplacian)
    enddo
  enddo
  Lambda = Lambda/N0_forLaplacian

end
\end{lstlisting}
\begin{lstlisting}[caption=外力項計算ルーチン]
subroutine calcGravity()
  !外力項(重力)の計算
  use define
  use consts_variables
  implicit none
  integer :: i, j

  Acc = 0d0
  !$omp parallel private(j)
  !$omp do
  do i = 1, NumberOfParticle
  if (ParticleType(i) == PARTICLE_FLUID) then
    do j = 1, numDimension
      Acc(i, j) = Gravity(j)
    enddo
  endif
  enddo
  !$omp end do
  !$omp end parallel

end
\end{lstlisting}
\begin{lstlisting}[caption=粘性項計算ルーチン]
subroutine calcViscosity()
  !粘性項の計算
  use define
  use consts_variables
  implicit none
  real*8 :: ViscosityTerm(numDimension)
  real*8 :: distance, weight
  real*8 :: calcWeight, calcDistance
  real*8 :: m
  integer :: i, j, k

  m = 2d0*numDimension/(N0_forLaplacian*Lambda)*KINEMATIC_VISCOSITY

  !$omp parallel private(ViscosityTerm, distance, weight, j, k)
  !$omp do
  do i = 1, NumberOfParticle
  if (ParticleType(i) == PARTICLE_FLUID) then
    ViscosityTerm = 0d0
    do j = 1, NumberOfParticle
      if (i == j) cycle
      distance = calcDistance(i, j)
      if (distance < Radius_forLaplacian) then
        weight = calcWeight(distance, Radius_forLaplacian)
        do k = 1, numDimension
          ViscosityTerm(k) = ViscosityTerm(k) + (Vel(j, k) - Vel(i, k))*weight
        enddo
      endif
    enddo
    do k = 1, numDimension
      Acc(i, k) = Acc(i, k) + ViscosityTerm(k)*m
    enddo
  endif
  enddo
  !$omp end do
  !$omp end parallel

end
\end{lstlisting}
\begin{lstlisting}[caption=仮速度・仮位置計算ルーチン]
subroutine moveParticleExplicit()
  !陽解法による粒子の移動
  use consts_variables
  implicit none
  integer :: i, j

  !$omp parallel private(j)
  !$omp do
  do i = 1, NumberOfParticle
  do j = 1, numDimension
    Vel(i, j) = Vel(i, j) + Acc(i, j)*dt
    Pos(i, j) = Pos(i, j) + Vel(i, j)*dt
  enddo
  enddo
  !$omp end do
  !$omp end parallel
  Acc = 0d0

end
\end{lstlisting}
\begin{lstlisting}[caption=衝突判定ルーチン]
subroutine collision()
  use define
  use consts_variables
  implicit none
  real*8 :: e = IMPACT_PARAMETER
  real*8 :: distance
  real*8 :: calcDistance
  real*8 :: impulse
  real*8 :: VelocityAfterCollision(NumberOfParticle, numDimension)
  real*8 :: velocity_ix, velocity_iy
  real*8 :: xij, yij
  real*8 :: mi, mj
  integer :: i, j

  CollisionState = .false.
  VelocityAfterCollision = Vel
  do i = 1, NumberOfParticle
    if (ParticleType(i) .ne. PARTICLE_FLUID) cycle
    mi = FLUID_DENSITY
    velocity_ix = Vel(i, 1)
    velocity_iy = Vel(i, 2)
    do j = 1, NumberOfParticle
      if (ParticleType(j) == PARTICLE_DUMMY) cycle
      if (i == j) cycle
      xij = Pos(j, 1) - Pos(i, 1)
      yij = Pos(j, 2) - Pos(i, 2)
      distance = calcDistance(i, j)
      if (distance < collisionDistance) then
        impulse = (velocity_ix - Vel(j, 1))*(xij/distance) + &
                  (velocity_iy - Vel(j, 2))*(yij/distance)
        if (impulse > 0d0) then
          CollisionState(i) = .true.
          CollisionState(j) = .true.
          mj = FLUID_DENSITY
          impulse = impulse*((1d0 + e)*mi*mj)/(mi + mj)
          velocity_ix = velocity_ix - (impulse/mi)*(xij/distance)
          velocity_iy = velocity_iy - (impulse/mi)*(yij/distance)
        endif
      endif
    enddo
    VelocityAfterCollision(i, 1) = velocity_ix
    VelocityAfterCollision(i, 2) = velocity_iy
  enddo

  !$omp parallel
  !$omp do
  do i = 1, NumberOfParticle
    if (ParticleType(i) .ne. PARTICLE_FLUID) cycle
    Pos(i, 1) = Pos(i, 1) + (VelocityAfterCollision(i, 1) - Vel(i, 1))*dt
    Pos(i, 2) = Pos(i, 2) + (VelocityAfterCollision(i, 2) - Vel(i, 2))*dt
    Vel(i, 1) = VelocityAfterCollision(i, 1)
    Vel(i, 2) = VelocityAfterCollision(i, 2)
  enddo
  !$omp end do
  !$omp end parallel

end
\end{lstlisting}
\begin{lstlisting}[caption=粒子数密度計算ルーチン]
subroutine calcNumberDensity()
  !粒子数密度の計算
  use define
  use consts_variables
  implicit none
  real*8 :: distance, weight
  real*8 :: calcDistance, calcWeight
  integer :: i, j

  NumberDensity = 0d0
  do i = 1, NumberOfParticle
    if (ParticleType(i) == PARTICLE_DUMMY) cycle
    do j = 1, NumberOfParticle
      if (ParticleType(j) == PARTICLE_DUMMY) cycle
      if (i == j) cycle
      distance = calcDistance(i, j)
      weight = calcWeight(distance, Radius_forNumberDensity)
      NumberDensity(i) = NumberDensity(i) + weight
    enddo
  enddo

end
\end{lstlisting}
\begin{lstlisting}[caption=ディリクレ境界条件判定ルーチン]
subroutine setBoundaryCondition()
  !境界条件の設定
  use define
  use consts_variables
  implicit none
  integer :: i

  !$omp parallel
  !$omp do
  do i = 1, NumberOfParticle
  if (ParticleType(i) == PARTICLE_DUMMY) then
    BoundaryCondition(i) = BOUNDARY_DUMMY
  elseif (NumberDensity(i) < (THRESHOLD_RATIO_BETA*N0_forNumberDensity)) then
    BoundaryCondition(i) = BOUNDARY_SURFACE
  else
    BoundaryCondition(i) = BOUNDARY_INNER
  endif
  enddo
  !$omp end do
  !$omp end parallel

end
\end{lstlisting}
\begin{lstlisting}[caption=ポアソン方程式右辺設定ルーチン]
subroutine setSourceTerm()
  !ポアソン方程式右辺の設定
  use define
  use consts_variables
  implicit none
  integer :: i

  SourceTerm = 0d0
  !$omp parallel
  !$omp do
  do i = 1, NumberOfParticle
    if (ParticleType(i) == PARTICLE_DUMMY) cycle
    if (BoundaryCondition(i) == BOUNDARY_SURFACE) cycle
    if (BoundaryCondition(i) == BOUNDARY_INNER) then
      SourceTerm(i) = RELAXATION_COEF_FOR_PRESSURE*(1d0/dt**2)* &
                      (NumberDensity(i) - N0_forNumberDensity)/N0_forNumberDensity
    endif
  enddo
  !$omp end do
  !$omp end parallel

end
\end{lstlisting}
\begin{lstlisting}[caption=係数行列の計算ルーチン]
subroutine setMatrix()
  !係数行列の設定
  use define
  use consts_variables
  implicit none
  real*8 :: distance
  real*8 :: calcDistance, calcWeight
  real*8 :: coefIJ, a
  integer :: i, j

  CoefficientMatrix = 0d0
  a = 2d0*numDimension/(N0_forLaplacian*Lambda)
  do i = 1, NumberOfParticle
    if (BoundaryCondition(i) .ne. BOUNDARY_INNER) cycle
    do j = 1, NumberOfParticle
      if (BoundaryCondition(j) == BOUNDARY_DUMMY) cycle
      if (i == j) cycle
      distance = calcDistance(i, j)
      if (distance >= Radius_forLaplacian) cycle
      coefIJ = a*calcWeight(distance, Radius_forLaplacian)/FLUID_DENSITY
      CoefficientMatrix(i, j) = -1d0*coefIJ
      CoefficientMatrix(i, i) = CoefficientMatrix(i, i) + coefIJ
    enddo
    CoefficientMatrix(i, i) = CoefficientMatrix(i, i) + COMPRESSIBILITY/(dt**2)
  enddo

end
\end{lstlisting}
\begin{lstlisting}[caption=ガウスの消去法による圧力計算ルーチン]
subroutine GaussEliminateMethod()
  use define
  use consts_variables
  implicit none
  real*8 :: Terms, c
  integer :: i, j, k

  Pressure = 0d0
  do i = 1, NumberOfParticle - 1
    if (BoundaryCondition(i) .ne. BOUNDARY_INNER) cycle
    do j = i + 1, NumberOfParticle
      if (BoundaryCondition(j) == BOUNDARY_DUMMY) cycle
      c = CoefficientMatrix(j, i)/CoefficientMatrix(i, i)
      !$omp parallel
      !$omp do
      do k = i + 1, NumberOfParticle
        CoefficientMatrix(j, k) = CoefficientMatrix(j, k) - c*CoefficientMatrix(i, k)
      enddo
      !$omp end do
      !$omp end parallel
      SourceTerm(j) = SourceTerm(j) - c*SourceTerm(i)
    enddo
  enddo

  i = NumberOfParticle
  do
    i = i - 1
    if (i == 0) exit
    if (BoundaryCondition(i) .ne. BOUNDARY_INNER) cycle
    Terms = 0d0
    !$omp parallel
    !$omp do reduction(+:Terms)
    do j = i + 1, NumberOfParticle
      Terms = Terms + CoefficientMatrix(i, j)*Pressure(j)
    enddo
    !$omp end do
    !$omp end parallel
    Pressure(i) = (SourceTerm(i) - Terms)/CoefficientMatrix(i, i)
  enddo

end
\end{lstlisting}
\begin{lstlisting}[caption=負圧除去ルーチン]
subroutine removeNegativePressure()
  !負圧の除去
  !粒子枢密を用いて計算しているため,境界付近で負圧が発生する
  use consts_variables
  implicit none
  integer ::i

  !$omp parallel
  !$omp do
  do i = 1, NumberOfParticle
    if (Pressure(i) < 0d0) Pressure(i) = 0d0
    !if (Pressure(i) > 30000d0) Pressure(i) = 30000d0
  enddo
  !$omp end do
  !$omp end parallel

end
\end{lstlisting}
\begin{lstlisting}[caption=近傍粒子での最小圧力設定ルーチン]
subroutine setMinPressure()
  !最小圧力設定ルーチン
  !引力項による計算の発散を抑制する
  use define
  use consts_variables
  implicit none
  real*8 :: distance
  real*8 :: calcDistance
  integer :: i, j

  do i = 1, NumberOfParticle
    if (ParticleType(i) == PARTICLE_DUMMY) cycle
    MinPressure(i) = Pressure(i)
    !$omp parallel
    !$omp do
    do j = 1, NumberOfParticle
      if (ParticleType(j) == PARTICLE_DUMMY) cycle
      if (i == j) cycle
      distance = calcDistance(i, j)
      if (distance >= Radius_forGradient) cycle
      if (MinPressure(i) > Pressure(j)) then
        MinPressure(i) = Pressure(j)
      endif
    enddo
    !$omp end do
    !$omp end parallel
  enddo

end
\end{lstlisting}
\begin{lstlisting}[caption=圧力勾配計算ルーチン]
subroutine calcPressureGradient()
  !圧力勾配の計算
  use define
  use consts_variables
  implicit none
  real*8 :: weight, distance, distance2
  real*8 :: calcWeight
  real*8 :: pIJ
  real*8 :: deltaIJ(numDimension)
  real*8 :: gradient(numDimension)
  integer :: i, j, k

  !$omp parallel private(gradient, distance, distance2, weight, deltaIJ, pIJ, j, k)
  !$omp do
  do i = 1, NumberOfParticle
    if (ParticleType(i) .ne. PARTICLE_FLUID) cycle
    gradient = 0d0
    do j = 1, NumberOfParticle
      if (i == j) cycle
      if (ParticleType(j) == PARTICLE_DUMMY) cycle
      distance2 = 0d0
      do k = 1, numDimension
        deltaIJ(k) = Pos(j, k) - Pos(i, k)
        distance2 = distance2 + deltaIJ(k)**2
      enddo
      distance = sqrt(distance2)
      if (distance < Radius_forGradient) then
        weight = calcWeight(distance, Radius_forGradient)
        pIJ = (Pressure(j) - MinPressure(i))/distance2
        do k = 1, numDimension
          gradient(k) = gradient(k) + deltaIJ(k)*pIJ*weight
        enddo
      endif
    enddo
    do k = 1, numDimension
      gradient(k) = gradient(k)*numDimension/N0_forGradient
      Acc(i, k) = -1d0*gradient(k)/FLUID_DENSITY
    enddo
  enddo
  !$omp end do
  !$omp end parallel

end
\end{lstlisting}
\begin{lstlisting}[caption=速度・位置計算ルーチン]
subroutine moveParticleImplicit()
  !陰解法による粒子の移動
  use consts_variables
  implicit none
  integer :: i, j

  !$omp parallel private(j)
  !$omp do
  do i = 1, NumberOfParticle
  do j = 1, numDimension
    Vel(i, j) = Vel(i, j) + Acc(i, j)*dt
    Pos(i, j) = Pos(i, j) + Acc(i, j)*dt**2
  enddo
  enddo
  !$omp end do
  !$omp end parallel
  Acc = 0d0

end
\end{lstlisting}
\end{document}