\section{実験方法}
図に示すような管路を用いて実験を行った.管路Aは染料注入法,管路BはLDV計測に用いた.
ただし用いる流体は水である.
\mfig[width=10cm]{fac.jpg}{実験装置\cite{rikougaku}}
\subsection{染料注入法による流れの可視化}
\label{sec:染料}
給水コックを開き,装置内に給水した.
水面が溢水管面に達してから吐き出し弁Aを開き,管路Aにわずかに水が流れる状態にした.
染料バルブを少し開き,染料を流した.ただし染料の流量は流れを乱さないように最低限の流量とする.
出口にビーカーを置き,定量が満ちるまでの時間から平均の流量を測定した.
レイノルズ数$Re\simeq1400$, $2300$, $6500$における流れの様子をスケッチした.
\subsection{LDVによる流速測定}
\S\ref{sec:染料}で示したのと同様な手順で管路Bに水を流した.
ビームの交差点が流れのほぼ中央にあることを確認し,測定を行った.
PCのLDV計測ソフトウェアを起動し,パラメータを設定した.
バースト信号を捉えるようにInput RangeとTrigger Levelを調整した.
調整後,数値が落ち着いたところで測定を行った.
同様にレイノルズ数$Re\sim1400$, $2300$, $6500$でも測定を行った.
