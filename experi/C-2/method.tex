\section{実験方法}
\subsection{装置条件}
固定相物質,キャリアガス流量(\si{\milli L.min^{-1}}),気化温度(\si{\degreeCelsius}),カラム温度(\si{\degreeCelsius}),検出器のブリッジ電流(\si{\milli\ampere})を記録した.
\subsection{保持時間の測定}
マイクロシリンジでエタノールを採取したのち,マイクロシリンジの針を試料注入口に差し込みすばやくプランジャーを押し込む.
同時にクロマトグラムの印字を開始することでエタノールの保持時間を記録した.同様に各アルコールについて保持時間を記録した.
また,マイクロシリンジで試料を採取する際にはメタノールで洗浄した後,乾燥,各アルコールで共洗いしてから用いた.
\subsection{絶対検量線・補正係数の測定}
各物質の容量比が異なる混合既知試料について2回ずつクロマトグラムを作成した.各物質の密度,分子量から物質量を計算し,各物質量でのクロマトグラムの平均ピーク面積をプロットすることで絶対検量線を作成した.
また,混合既知試料⑤のエタノールを標準物質として(\ref{equ:Fi})式で各物質の補正係数$F_i$を計算した.
\begin{equation}
  \label{equ:Fi}
  F_i=\frac{標準物質のピーク面積}{成分iのピーク面積}\times\frac{成分iの物質量}{標準物質の物質量}
\end{equation}
各$F_i$と各成分のピーク面積$A_i$から成分$i$のモル比を(\ref{equ:molper})式で計算できる.
\begin{equation}
  \label{equ:molper}
  成分iのモル比(\%)=\frac{A_i F_i}{\sum_i A_i F_i}\times100
\end{equation}
\subsection{未知試料の分析}
未知試料についてクロマトグラムを作成した.実験2で測定した保持時間から各ピークと物質を対応させ,
絶対検量線から物質量を算出し組成モル比を計算した.また,補正係数を用いてピーク面積から直接組成モル比を計算した.


