\section{考察}
\subsubsection{各手法の特徴}
補正面積百分率法と絶対検量線法を比較すると,エタノールで2.5\%,1-プロパノールで0.7\%,1-ブタノールで0.4\%,
1ペンタノールで1.3\%の差が出ている.このことから,2つの手法に大きな差異は見られなかった.
両者の精度を検討すると,補正面積百分率法は1つの補正係数に対して2個のデータが用いられてるのに対して,
絶対検量線法は1つの検量線に対して10個のデータが用いられている.このため,平均の実験標準偏差$\sigma_{\bar{x}}$は(\ref{equ:SD})式\cite{buturi}より偏差を$\delta_i$として,絶対検量線法が5倍程度優れていると考えられる.
\begin{equation}
  \label{equ:SD}
  \sigma_{\bar{x}}=\sqrt{\frac{{\Sigma_i {}\delta_i}^2}{n(n-1)}}
\end{equation}
その他の特徴として以下のようなものが挙げられる.\\
補正面積百分率法\\メリット
\begin{itemize}
  \item 単一の混合既知試料から補正係数を算出可能
\end{itemize}
デメリット
\begin{itemize}
  \item 補正係数を算出するクロマトグラムで大きな誤差が生じた場合,これを修正できない
  \item 補正係数の算出に全成分の物質量が既知な試料が必要
\end{itemize}
絶対検量線法\\メリット
\begin{itemize}
  \item 試料の成分が判明しており,絶対検量線がすでに存在する場合,新たに絶対検量線を作成する必要がない
  \item 複数のサンプルから検量線を作成したため,誤差は小さくなる
\end{itemize}
デメリット
\begin{itemize}
  \item 絶対検量線の作成には複数の物質量が既知の多数の試料が必要
  \item 注入量の誤差が直接定量値に反映される
\end{itemize}
以上を踏まえ,測定誤差を少なくする方法として以下のようなものが考えられる.
\begin{itemize}
  \item 検量線,補正係数の算出に用いるデータ数を増やす
  \item 未知試料に対してより多くのクロマトグラムを作成する
  \item 検出器のブリッジ電流を増やし,ピーク面積の分解能を高める
\end{itemize}
ガスクロマトグラフィーの性能指標として理論段数,分離度などが挙げられる\cite{BN00380381}が,今回の実験ではピーク同士が
完全に分離しており理論段数,分離度は十分であると考えられる.
\subsection{保持時間と炭素数・沸点の関係}
図\ref{fig:num_c.png},図\ref{fig:hutten.png}の相関係数$R$はそれぞれ0.9925,\ 0.9949であり双方に
強い正の相関が見られる.沸点が高い物質とは蒸気圧の低い物質そのものであり,固定相中のモル濃度が高く,移動相中のモル濃度が低くなると考えられる.
すなわち,沸点が高い物質ほど保持能が高くなるため,保持時間と沸点に正の相関があると考えられる.
また,図\ref{fig:num_c_hutten.png}のように炭素数と沸点の間にも線形関係が見られる.炭素数が大きいほど分子量が大きくなり,分子間力が大きくなるためこの関係は自然である.
故に炭素数と保持時間にも正の相関があることがわかる.
\mfig[width=15cm]{num_c_hutten.png}{炭素数-沸点の関係}