\section{方法}
\subsection{試験片}
試験片の形状はJIS規格に準拠する.
試験片の材料は
\begin{itemize}
  \item S20C,焼鈍材
  \item A1070
\end{itemize}
である.試験片には$10~\si{\milli\metre}$ごとに標点が刻まれ,
実験前後の標点での直径をノギスで,標点の座標を読み取り顕微鏡で記録した.
\subsection{ひずみゲージの接着}
試験片の一部を紙やすりで軽くやすり,アルコールで洗浄した後,専用の接着剤で接着した.
\subsection{引張試験}
試験片を引張試験機のチャックに取り付けた.この際,試験片の掴み部がチャックから少し出るようにした.
試験片を取り付けた状態でのクロスヘッドの位置を0とする.
その後,鋼では$2000~\si{\newton}$,アルミニウムでは$200~\si{\newton}$ずつ荷重を増やし,
各荷重でのひずみやストロークを記録した.
また,ある程度の範囲で記録したら,その後は自動モードで荷重とストロークを記録した.