\section{加工手順}
表\ref{tab:tejun}に加工手順を示す.
\begin{table}[h]
  \caption{加工手順}
  \label{tab:tejun}
  \centering
  \begin{tabular}{lllll}
    \hline
    工程&計&算出式&詳細な手順&切り込み量\\
    \hline \hline
    外丸削り(A部)&28.5 mm&$28.5=(70-13)/2$&荒削り(右から40 mm程度)&1.0 mm\\
    &&&仕上げ&0.2 mm\\
    \hline
    端面削り&39 mm&$39=70-31$\\
    &右:37 mm&&右端 荒削り&1.0 mm\\
    &&&右端 仕上げ&0.2 mm\\
    &左:約2 mm&$2=39-37$&右端から31.0 mmけがく\\
    &&&左端 荒削り&1.0 mm\\
    &&&左端 仕上げ&0.2 mm\\
    \hline
    ドリル穴あけ&&&センタドリルで下穴あけ\\
    &10.0 mm&&ドリル穴あけ\\
    \hline
    外丸削り(B部)&30.0 mm&$30.0=(70-10)/2$&右端から21.0 mmけがく\\
    &&&荒削り($21.0-0.5$ mm程度)&1.0 mm\\
    &&&仕上げ&0.2 mm\\
    \hline
    外丸削り(C部)&31.0 mm&$31.0=(70-8)/2$&右端から15.0 mmけがく\\
    &&&荒削り($15.0-0.5$ mm程度)&1.0 mm\\
    &&&仕上げ&0.2 mm\\
    \hline
    面取り&5箇所\\
    \hline
  \end{tabular}
\end{table}