\section{実験結果}
\subsection{コンデンサーの作製}
製作したコンデンサーの条件は表\ref{tab:capacitor}のようになった.
\begin{table}[htbp]
   \caption{コンデンサーの条件}
   \label{tab:capacitor}
   \centering
   \begin{tabular}{lc}
     \hline
     項目&値\\
     \hline \hline
     紙の厚さ&0.065 \si{\milli\metre}\\
     紙の厚さ(10枚重ね)&0.675 \si{\milli\metre}\\
     極板の一辺&398.0 \si{\milli\metre}\\
     極板の一辺&400.4 \si{\milli\metre}\\
     キャパシタンス&33.1 \si{\nano\farad}\\
     \hline
   \end{tabular}
\end{table}\\
従って,紙の比誘電率$\overline{\epsilon}$は以下のようになる.
\begin{align*}
  \overline{\epsilon}=\cfrac{33.1\times10^{-9}\times0.0675\times10^{-3}}{\epsilon_0\times0.3980\times0.4004}\approx1.58
\end{align*}
\subsection{測定}
表\ref{tab:condition}に各実験の条件を示す.
\begin{table}[h]
   \caption{実験条件}
   \label{tab:condition}
   \centering
   \begin{tabular}{c|cccc}
     \hline
     実験&電源電圧 / \si{\volt}&抵抗 / \si{\ohm}&キャパシタンス / \si{\nano\farad}&インダクタンス / \si{\milli\henry}\\
     \hline \hline
     1&1.2&0&33.1&50\\
     2&1.2&150&33.1&50\\
     3&1.2&150&33.1&75\\
     \hline
   \end{tabular}
\end{table}
図\ref{fig:graph1.eps}に各実験での共振曲線,図\ref{fig:graph2.eps}に$f$-$\theta$グラフ,図\ref{fig:graph3.eps}に$f$-$I/I_{max}$グラフを示す.
なお,共振曲線の最尤曲線は最小二乗法により作図しており,そのパラメーターを表\ref{tab:saiyu}に示す.
\mfig[width=9cm]{graph1.eps}{共振曲線}
\mfig[width=9cm]{graph2.eps}{$f$-$\theta$グラフ}
\mfig[width=9cm]{graph3.eps}{$f$-$I/I_{max}$グラフ}
\begin{table}[htbp]
   \caption{共振曲線の最尤曲線}
   \label{tab:saiyu}
   \centering
   \begin{tabular}{c|ccc}
     \hline
     実験&抵抗 / \si{\ohm}&キャパシタンス / \si{\nano\farad}&インダクタンス / \si{\milli\henry}\\
     \hline \hline
     1 & $63.99\pm{0.15}$ & $53.09\pm{0.29}$ & $29.07\pm{0.16}$ \\
     2 & $226.1\pm{0.3}$ & $54.17\pm{0.10}$ & $28.51\pm{0.05}$ \\
     3 & $260.1\pm{0.4}$ & $81.32\pm{0.21}$ & $28.67\pm{0.07}$ \\
     \hline
   \end{tabular}
\end{table}
また,実験1〜3に関して$V_R$, $V_L$, $V_C$のグラフを示す.
\begin{figure}[htbp]
  \begin{minipage}{0.5\hsize}
    \mfig[width=7cm]{graph4_2.eps}{実験1の$V_R$グラフ}
  \end{minipage}
  \begin{minipage}{0.5\hsize}
    \mfig[width=7cm]{graph4_1.eps}{実験1の$V_L$, $V_C$グラフ}
  \end{minipage} 
\end{figure}
\begin{figure}[htbp]
  \begin{minipage}{0.5\hsize}
    \mfig[width=7cm]{graph5_2.eps}{実験2の$V_R$グラフ}
  \end{minipage}
  \begin{minipage}{0.5\hsize}
    \mfig[width=7cm]{graph5_1.eps}{実験2の$V_L$, $V_C$グラフ}
  \end{minipage} 
\end{figure}
\begin{figure}[htbp]
  \begin{minipage}{0.5\hsize}
    \mfig[width=7cm]{graph6_2.eps}{実験3の$V_R$グラフ}
  \end{minipage}
  \begin{minipage}{0.5\hsize}
    \mfig[width=7cm]{graph6_1.eps}{実験3の$V_L$, $V_C$グラフ}
  \end{minipage} 
\end{figure}