\section{考察}
\subsection{鋼材とアルミニウムの共通点・相違点}
鋼材とアルミニウムの共通点として以下のような点が挙げられる.
\begin{itemize}
  \item 比例限度のひずみは鋼材の場合0.066 \%,アルミニウムの場合0.042 \%と小さい.すなわち,フックの法則に従う範囲が小さい.
  \item 引張強さ付近ではひずみを変化させても応力の変化が小さい.
  \item 引張強さから破断強さかけて,急激に応力が小さくなっていく.
  \item 図\ref{fig:gnuplot/graph2.eps},図\ref{fig:gnuplot/graph4.eps}のようにどちらも降伏現象が発生している.
  \item 図\ref{fig:gnuplot/graph1.eps},図\ref{fig:gnuplot/graph3.eps}のようにどちらも伸びが大きい点で直径が細くなっている.このことは体積が保存していることを考えれば自然である.
  \item どちらも破断面を含む区間が大きく伸び,他の区間の伸びは比較的小さい.材料の直径に不均一性があると,直径が小さい部分に大きな応力が掛かるので,その部分が絞られ更に応力が集中する.したがって,試験片のわずかに直径が小さい部分が大きく絞られ,他の部分の絞りは小さくなる.そして最も絞りが大きい部分が破断する.
\end{itemize}
一方で相違点として以下のような点が挙げられる.
\begin{itemize}
  \item アルミニウムの降伏点は鋼材のそれに比べて小さく,降伏後に応力が変化しない区間がほぼ存在しない.
  \item アルミニウムの引張強さは鋼材の17 \%程度で弱い.
  \item アルミニウムの破断強さは実験開始時の応力よりも小さい.このことから不可逆な構造の変化が生じていることが推測できる
  \item 鋼材は音を立てて急激に破断したのに対し,アルミニウムは静かに破断した.
  \item 鋼材の破断面は円形だが,アルミニウムは円形でない.
\end{itemize}
\subsection{S20C,焼鈍材の特徴\cite{referen}\cite{misumi}}
鉄鋼は以下のような分類がある.
\begin{itemize}
  \item 一般構造用圧延鋼材\\JIS規格で強度が規定されている.機械部品などに用いられる. (例:SS400)
  \item 機械構造用炭素鋼鋼材\\JIS規格で強度と組成が規定されている. SXXCという記号で識別され, 0.XX \%の炭素を含有する. (例:S45C)
  \item 炭素工具鋼鋼材\\焼入れと焼きなましを行う鋼材.硬度が高く工具などに用いられる. (例:SK4)
  \item 高炭素クロム軸受鋼鋼材\\ベアリングなどに用いるため,耐摩耗性が高い. (例:SUJ2)
\end{itemize}
S20Cは機械構造用炭素鋼鋼材の一種であり炭素含有率が$0.18\sim0.23~\%$程度である.

焼鈍材とは熱処理として焼きなましを行った材料のことである.
焼きなましを行うことで徐々に結晶が冷やされ,内部のひずみや格子欠落が減少する.
したがって展延性が向上する.\cite{anies}
対して,焼入れを行うと,展延性が損なわれる代わりに硬度や強度が向上する.
\subsection{A1070の特徴\cite{misuim_al}}
アルミニウム合金には以下のような分類がある.
\begin{itemize}
  \item A1000系\\最もアルミニウムの純度が高く,耐食性や電気,熱伝導性に優れる.電気器具や容器などに用いられる. (例:A1070)
  \item A2000系\\Al-Cu合金系,ジュラルミン合金と呼ばれ,強度が高いが耐食性や表面処理性に難がある.航空機や機械部品に用いられる. (例:A2024)
  \item A5000系\\Al-Mg合金系,比較的強度が高く耐食性や表面処理性も高い.建材や構造材に用いられる. (例:A5052)
  \item A7000系\\Al-Zn-Mg合金系,アルミニウム合金の中では最も強度いが,耐食性に難あり.航空機や鉄道車両に用いられる. (例:A7075)
\end{itemize}
A1070はA1000系の合金であり, 99.7 \%以上の純度を持つ.
\subsection{測定値の工学的使用法}
\subsubsection{降伏応力について}
S20Cなどの軟鋼の場合,降伏応力を超える応力を加えると応力は下降伏点まで下がり,急激にひずみが大きくなることがわかる.
すなわち,降伏応力以上の応力を掛けることは,材料の大幅で不可逆な変形を意味する.
したがって,設計の際には材料に掛かる応力が降伏応力以下になるように設計を行う必要があり,また十分な安全率をとるべきである.
\subsubsection{0.2 \%耐力について}
0.2 \%耐力とは0.2 \%の塑性変形が発生する点であるといえる.
なぜならば,塑性変形した金属から荷重を除去すると,塑性変形した状態からヤング率の傾斜に沿ってひずみが減少する.
つまり0.2 \%耐力の応力を受け塑性変形した金属は,荷重を除去してもオフセット線にそってひずみが減少し,結果0.2 \%のひずみが残るのである.
アルミニウムなどの材料では,降伏点が明確に表れないので,その代わりとなる指標を元に設計する必要があり,この指標として0.2 \%耐力が用いられる.
\subsubsection{引張強さについて}
引張強さはその材料が破損しない最大の応力である.
引張強さ以上の応力を材料が受けた場合,その材料は破断するため,設計の際には材料に掛かる応力が引張強さ以下になるように設計を行う必要があり,また十分な安全率をとるべきである.
\subsubsection{伸び,絞りについて\cite{hippari}}
伸び,絞りが高いことは延性,展性が高いことを意味する.
延性,展性が高いと部分的な荷重を分散するため,建築などにおいて重要な指標になる.
また,延性,展性が高い材料はプレスやしぼり加工などに適する.
\subsection{カップアンドコーン破壊\cite{hakai}}
カップアンドコーン破壊の破断面は図\ref{fig:hakai.jpg}のようになっている.
図\ref{fig:cup_corn.png},図\ref{fig:sendan.png}にカップアンドコーン破壊のモデルを示す.
金属材料中に不純物が存在すると,応力によってその境界面から空隙が生じる.
この空隙が成長,周囲の空隙と連結することで断面の凹凸が形成される.
また,空隙がある程度成長すると,一部にせん断応力が集中し
図\ref{fig:sendan.png}のように繋がった部分でせん断破壊が生じる.せん断応力は45度方向に最も大きいので,カップの縁に相当する部分が形成される.

またA1070のような不純物の少ない材料では空隙が成長せずカップアンドコーン破壊は起きない.
\mfig[width=8cm]{hakai.jpg}{カップアンドコーン破壊の破断面\cite{hakai}}
\mfig[width=8cm]{cup_corn.png}{カップアンドコーン破壊のモデル\cite{hakai}}
\mfig[width=8cm]{sendan.png}{カップアンドコーン破壊のモデル\cite{hakai}}
\subsection{降伏点について\cite{kouhuku}}
鋼鉄で結晶が欠陥した部分(転移)では,電子同士の斥力が小さくなっているため,炭素などの不純物が集まり安定な状態になっている.
この状態をコットレル固着と呼ぶ.
ある上降伏点の応力を加えるとコットレル固着にあった転移が動き出す.この状態では力は全て転移をコレットル固着から剥がすために
使われるため,応力が低下する.したがって,下降伏点が生じる.
\subsection{誤差要因}
ストロークの値は,引張試験機自体の剛性や,コレットの滑り,ロードセルの変形などを考慮していない.
よって実際の変位よりも大きなひずみが算出されていると考えられる.