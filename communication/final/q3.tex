\section{課題3}
5Gの要件として以下の3つが挙げられる\cite{5G技術の解説と13:online}.
\begin{description}
  \item[高速大容量通信(eMBB)]\mbox{}\\
  最大通信容量を下り20 Gbps, 上り10 Gbpsとする(4Gでは1 Gbps).この実現のために,
  周波数帯の拡大(4G : 3.5GHz→5G : 3.7GHz, 4.5GHz, 28GHz),
  また回り込みの減少に伴うセルの縮小,
  アレイアンテナでのビームフォーミングによる電波到達距離の延伸などといった技術が用いられる.
  \item[超信頼性・低遅延通信(URLLC)]\mbox{}\\
  遅延を最短で1 ms以下とする.この実現のためモバイルエッジ処理といった技術が用いられる.
  モバイルエッジ処理とは,従来の通信では送信端末から基地局,中央センターを介して受信端末へとデータを伝送していたものを,
  基地局間で直接通信を確立することで遅延を低減する技術である.
  \item[多数同時接続(mMTC)]\mbox{}\\
  従来では1つの基地局に接続するノードは100台程度だったものを1万から2万程度に拡大する.
  この実現のためにGrant Free方式を用いる. 
  Grant Free方式ではデータ送信の際の事前許可を廃することでオーバーヘッドを低減する\cite{Microsof84:online}.
\end{description}
特に5Gによって実現するとされるアプリケーションの中では,自動運転技術が重要であると考える.
その理由として,交通事故の抑制,輸送の効率化, MaaSの実現などが挙げられる.
自動運転車では搭載カメラでの映像を遠隔で監視したり,遠隔で運転できることが求められる\cite{2020年には自36:online}.
複数のカメラからの映像を同時に伝送したり,低遅延での運転に関しては5Gの優位性があらわれると考えられる.
また都市部では多くの車両が同時に通信する必要があり,多数同時接続も必要になる.
またトラック運転手の不足などは近年社会問題になるほど深刻であり,自動運転の実現が急がれる.

自動運転以外で5Gのアプリケーションとして挙げられる例に遠隔医療や4kでのストリーミングがあるが,それらに意義があるようにあまり考えられない.
その理由として,まず遠隔医療について,遠隔での執刀などを想定するのであれば何よりも重要となるのは信頼性・低遅延だと思われるが,
5Gが普及したとしても,信頼性で有線ネットワークを超えることはありえない.
有線であればISPからの専用回線を伸ばすことができるが,
5Gでは同じ基地局を多くの端末で利用することから信頼性の低下や一時的な遅延の増加を引き起こす要素が多い.
また,基地局間が有線で接続されることには変わりない.
したがって,有線を用いずにあえて5Gを用いる利点はあまり見いだせない.

また遠隔での診断について,問診程度なら4Gでも十分だし,更に診断を行う機械を各家庭に配置することは考えにくい.
仮に診断を行う機械を各家庭に配置したとして,そのデータが10Gbpsの回線を必要としたり,リアルタイム性を求めるとも考えにくい.

更に, 4kでのストリーミングについては,
4k解像度のディスプレイを搭載するスマートフォンはSONY Xperia Z5 Premium以降発売されていない.
これは,単純に小さいディスプレイでは4k解像度の恩恵を受けることができないからである.
一方でテレビなどの据え置きする画面の大きいモニターに関しては,そもそも据え置きするのに5Gを用いる意味がない.

以上から遠隔医療や4kでのストリーミングなどは5Gの優位性やそもそもの意義が薄いのではないかと考える.