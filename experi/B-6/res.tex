\section{実験結果}
\subsection{実験1}
\label{sec:res_1}
この実験では鏡$L_1$を小さくすると干渉縞が消滅した.
表\ref{tab:res_1}に実験1の結果を示す.また,図\ref{fig:graph1.eps}にその分布を示す.
\begin{table}[htbp]
   \caption{実験1の結果}
   \label{tab:res_1}
   \centering
   \begin{tabular}{c|cccc}
     \hline
     &開始位置 / \si{\milli\metre}&終了位置 / \si{\milli\metre}&明暗の変化回数&波長 / \si{\nano\metre}\\
     \hline \hline
     1 & 0.0000 & 0.0165 & 50 & 660 \\
2 & 0.0000 & 0.0167 & 50 & 668 \\
3 & 0.0000 & 0.0165 & 50 & 660 \\
4 & 0.0000 & 0.0160 & 50 & 640 \\
5 & 0.0000 & 0.0160 & 50 & 640 \\
6 & 0.0000 & 0.0160 & 50 & 640 \\
7 & 0.0000 & 0.0160 & 50 & 640 \\
8 & 0.0000 & 0.0165 & 50 & 660 \\
9 & 0.0000 & 0.0165 & 50 & 660 \\
10 & 0.0000 & 0.0162 & 50 & 648 \\
     \hline
   \end{tabular}
\end{table}
\mfig[width=12cm]{graph1.eps}{波長$\lambda$の分布}
したがって計測された波長$\lambda$は
\begin{align*}
  \lambda=651.6\pm3.5~\si{\nano\metre}
\end{align*}
である\cite{buturi}.
\subsection{実験2}
\label{sec:res_2}
この実験では圧力の現象に伴い干渉縞が消滅した.
表\ref{tab:res_cell}に光学セルの厚さ$T$の測定結果を示す.また表\ref{tab:res_2}に実験2の結果,図にその分布を示す.ただし,全ての測定で明暗の変化回数は1である.
\begin{table}[htbp]
   \caption{光学セルの厚さ$T$}
   \label{tab:res_cell}
   \centering
   \begin{tabular}{c|c}
     \hline
     &$T$ / \si{\milli\metre}\\
     \hline \hline
     1 & 10.30 \\
2 & 9.55 \\
3 & 9.85 \\
4 & 10.20 \\
5 & 9.80 \\
6 & 9.85 \\
7 & 10.35 \\
8 & 9.75 \\
9 & 9.55 \\
10 & 10.40 \\
     \hline
   \end{tabular}
\end{table}\\
したがって計測された光学セルの厚さ$T$は
\begin{align*}
  T=9.96\pm0.31~\si{\milli\metre}
\end{align*}
である.
\begin{table}[htbp]
   \caption{実験2の結果}
   \label{tab:res_2}
   \centering
   \begin{tabular}{c|ccc}
     \hline
     &開始圧力 / \si{\mmHg}&終了圧力 / \si{\mmHg}&圧力差$\Delta P$ / \si{\mmHg}\\
     \hline \hline
     1 & 142.4 & 54.0 & 88.4 \\
2 & 145.0 & 48.6 & 96.4 \\
3 & 153.6 & 53.8 & 99.8 \\
4 & 200.0 & 107.0 & 93.0 \\
5 & 205.6 & 100.8 & 104.8 \\
6 & 206.0 & 106.0 & 100.0 \\
7 & 215.2 & 112.8 & 102.4 \\
8 & 214.4 & 113.0 & 101.4 \\
9 & 204.8 & 108.0 & 96.8 \\
10 & 202.2 & 112.6 & 89.6 \\
     \hline
   \end{tabular}
\end{table}\\
\mfig[width=12cm]{graph2.eps}{$\alpha$の分布}
したがって計測された比例定数$\alpha$は,誤差の伝搬式を用いて
\begin{align*}
  \alpha=3.27\times10^{-7}\pm0.20\times10^{-7}~\si{\mmHg^{-1}}
\end{align*}
である.