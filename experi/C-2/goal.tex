\section{目的}
クロマトグラフ法とは,一般に物質の吸着力の差を用いて混合物を分離,分析する方法である.代表的なものとして
,ペーパクロマトグラフ法,液体クロマトグラフ法,ガスクロマトグラフ法
などがある.特にガスクロマトグラフ法は沸点300℃程度以下の有機物の混合物を,少量の試料で迅速に分析できる.
また,ガスクロマトグラフィーの分解能は各種クロマトグラフ法の中で最高であり,また高感度である.\cite{BN05476353}
また,固定相の物質を変えることで様々な種類の試料を分析することができる.
