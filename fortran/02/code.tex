%ここからソースコードの表示に関する設定
\lstset{
  basicstyle={\ttfamily},
  identifierstyle={\small},
  commentstyle={\smallitshape},
  keywordstyle={\small\bfseries},
  ndkeywordstyle={\small},
  stringstyle={\small\ttfamily},
  frame={tb},
  breaklines=true,
  columns=[l]{fullflexible},
  numbers=left,
  xrightmargin=0zw,
  xleftmargin=3zw,
  numberstyle={\scriptsize},
  stepnumber=1,
  numbersep=1zw,
  lineskip=-0.5ex
}
%ここまでソースコードの表示に関する設定
\appendix
\section{ソースコード}
ソースコードは以下のレポジトリで管理されている.\\
\url{https://github.com/RyosukeSasaki/particle-fortran}
\begin{lstlisting}[caption=メインプログラム]
program main
  use consts_variables
  implicit none
  integer, parameter :: interval = 10
  integer :: i, j
  character :: filename*128
  character :: command*128
  call omp_set_num_threads(8)

  command = 'mkdir -p ' // dir // '; rm ' // dir // '/data*'
  call system(command) 

  filename = dir // "/thrust.dat"

  open (19, file=filename, status='replace')
  write (*, *) "write data to ", dir, "/. Pressure is ", WaterPressure

  call InitParticles()
  call calcConsts()
  call writeInit()
  call output(0)
  do i = 0, 240-1
  do j = 1, interval
    call calcGravity()
    call calcViscosity()
    call moveParticleExplicit()
    call collision()
    call calcNumberDensity()
    call setBoundaryCondition()
    call setSourceTerm()
    call setMatrix()
    call GaussEliminateMethod()
    call removeNegativePressure()
    call setMinPressure()
    call calcPressureGradient()
    call moveParticleImplicit()
    call detach(i*interval+j)
    call output(i*interval+j)
  enddo
  write (*, *) "Timestep: ", (i+1)*interval
  enddo

  write (19, *) "#deltaV: ", deltaV
  close (19)

end
\end{lstlisting}
\begin{lstlisting}[caption=ファイル出力用ルーチン]
subroutine output(i)
  use consts_variables
  implicit none
  integer :: i, k
  character :: filename*128

  write (filename, '("data2/data", i4.4, ".dat")') i
  open (11, file=filename, status='replace')
  do k = 1, NumberOfParticle
    write (11, '(f6.3,X)', advance='no') Pos(k, 1)
    write (11, '(f6.3,X)', advance='no') Pos(k, 2)
    write (11, '(I2,X)', advance='no') ParticleType(k)
    write (11, '(f15.5,X)', advance='no') Pressure(k)
    if (CollisionState(k) .eqv. .true.) then
      write (11, '(I2)', advance='no') 1
    else
      write (11, '(I2)', advance='no') 0
    endif
    write (11, *)
  enddo
  close (11)

end
\end{lstlisting}
\input{code/define.tex}
\begin{lstlisting}[caption=重み係数計算関数]
real*8 function calcWeight(distance, radius)
  implicit none
  real*8, intent(in) :: distance, radius
  real*8 :: w

  if (distance >= radius) then
    w = 0
  else
    w = radius/distance - 1d0
  endif
  calcWeight = w
  return

end function
\end{lstlisting}
\begin{lstlisting}[caption=粒子間距離計算関数]
real*8 function calcDistance(i, j)
  use consts_variables
  implicit none
  real*8 :: distance2
  integer :: i, j, k

  distance2 = 0d0
  do k = 1, numDimension
    distance2 = distance2 + (Pos(j, k) - Pos(i, k))**2
  enddo
  calcDistance = sqrt(distance2)
  return

end function
\end{lstlisting}
\section{初期状態・推力計算}
\label{sec:init}
3次元では計算量が非常に膨大になるため,ここでは図\ref{fig:init.eps}のような平面の筒で水ロケットを近似する.
\begin{itemize}
    \item 壁面はダミー粒子3層と壁粒子3層から成る.
    \item 筒の直径は0.1 \si{\metre}とする.
    \item 水を加圧する圧力は1000 \si{\pascal}, 2000 \si{\pascal}, 3000 \si{\pascal}, 4000 \si{\pascal}とする.
    \item 水粒子は筒の中の$0\leq{y}\leq0.1$, $0\leq{y}\leq0.2$, $0\leq{y}\leq0.3$に等間隔で分布する.また,水は$0\leq{z}\leq0.1$の厚さで存在するものとする.
    \item 粒子の重さは$\cfrac{水の密度\times 初期状態の水の体積}{水粒子数}$で求める.
\end{itemize}
また${y}\leq-0.01$となった粒子の運動量を各タイムステップにおける推力とし,その合計を$\Delta{v}$とする.
\section{境界条件}
境界条件として、Dirichlet境界条件とNeumann境界条件を課す.
Dirichlet境界条件とは基準となる液面での圧力を設定するものである.
具体的には,粒子数密度が一定以下の粒子が液面にあると判定し,この粒子に対する圧力を与える.
Neumann境界条件とは壁面付近での圧力勾配を0とするもので,粒子が壁面を透過しないことを課す.
具体的には,壁粒子の更に外側にダミー粒子を配置し,この粒子と近傍の粒子の圧力値を等しくすることで近似的にNeumann境界条件を課す.
\input{code/calcconst.tex}
\input{code/calcExplicit.tex}
\input{code/calcImplicit.tex}