\subsection{その他(1)}
カットオフ周波数とは出力波形の振幅が入力波形の$1/\sqrt{2}$倍になる周波数である.
このとき出力の電力は入力の$1/2$倍になる.
また図\ref{fig:que/graph5.eps}のように,漸近線の延長線はカットオフ周波数でx軸と交わる.
すなわち,出力電圧が減衰し始める周波数ということもできる.
カットオフ周波数が大きいとはより広い周波数帯域に対して信号を透過できることを意味する.
\mfig[width=12cm]{que/graph5.eps}{ボード線図(理論値)}
\subsection{その他(2)}
図\ref{fig:que/graph2.eps}に示したように,純粋微分器は高周波成分での利得が大きいため,
リップルノイズなどを過剰に増幅してしまう.
したがって,図\ref{fig:other/c1.png}のようにコンデンサと直列に抵抗を入れ,高周波成分の利得を抑える必要がある\cite{オペアンプの応用42:online}.
\mfig[width=6cm]{other/c1.png}{改良した微分器}
\subsection{その他(3)}
図\ref{fig:other/c2.png}にインダクタを用いた積分器を示す.
ここで回路方程式は
\begin{align}
  \cfrac{y(t)}{R}&=-\cfrac{1}{L}\int_0^tu(\tau)d\tau\\
  y(t)&=-\cfrac{R}{L}\int_0^tu(\tau)d\tau\\
\end{align}
であり,確かに入力の積分が出力されている.

図にインダクタを用いた微分器を示す.
ここで回路方程式は
\begin{align}
  \cfrac{1}{L}\int_0^ty(\tau)d\tau&=-\cfrac{u(t)}{R}\\
\end{align}
両辺を$t$で微分し
\begin{align}
  \cfrac{1}{L}y(t)&=-\cfrac{1}{R}\cfrac{du(t)}{dt}\\
  y(t)&=-\cfrac{L}{R}\cfrac{du(t)}{dt}\\
\end{align}
であり,確かに入力の微分が出力されている.
\mfig[width=6cm]{other/c2.png}{インダクタを用いた積分器}
\mfig[width=6cm]{other/c3.png}{インダクタを用いた微分器}
\subsection{その他(4)}
多点接地とは,回路上の接地点を複数の経路でGNDに接続することである.
多点接地をした場合,経路ごとにインピーダンスが異なると電流により電位差が発生するためノイズ源となる\cite{ノイズ・サージ防20:online}.
今回の回路では入力電圧$u(t)$をファンクションジェネレータから,$V_{cc}$を電源装置から取っており,
それぞれ異なった経路で接地されている.したがって,多点接地によるノイズが発生していると考えられる.
\subsection{その他(5)}
バイパスコンデンサはICなどの電源ラインとGND間をつなぐように入れるコンデンサで,積層セラミックコンデンサなどを用いる.
これを入れることにより以下のような効果が期待できる.
\begin{itemize}
  \item インピーダンスの交流成分を下げ,ICが発するスイッチングノイズをGNDに落とす.
  \item 電源電圧の変動を吸収する.
  \item 一時的な消費電流の増加を補償する.
\end{itemize}