\section{実験結果}
\subsection{各電池の起電力測定}
各電池の起電力は表\ref{tab:kidenryoku}のようになった.
\begin{table}[h]
   \caption{各電池の起電力}
   \label{tab:kidenryoku}
   \centering
   \begin{tabular}{c|ccccccc}
     \hline
     \multirow{2}{*}{名称} & \multicolumn{7}{c}{起電力 / \si{\milli\volt}}\\
     &0分&1分&2分&3分&4分&5分&6分\\
     \hline \hline
     Daniell&1101.97&1101.86&1101.74&1101.63&1101.59&1101.54&-\\
     濃淡(0.1M, 0.025M)&-17.789&-18.535&-18.430&-18.308&-18.328&-18.348&-\\
     濃淡(0.1M, 0.05M)&-10.067&-10.179&-9.896&-9.695&-9.656&-9.664&-\\
     濃淡(0.1M, 0.15M)&1.826&3.103&3.535&3.773&4.075&4.213&4.241\\
     濃淡(0.1M, 0.2M)&6.825&7.599&7.755&7.869&7.811&7.769&-\\
     \hline
   \end{tabular}
\end{table}\\
またすべての実験後において電極は金属光沢を失い,銅電極は黒く,亜鉛電極は白くくすんでいた.
%Daniell電池の起電力は表\ref{tab:daniel}のようになった.
%\begin{table}[h]
%   \caption{}
%   \label{tab:daniel}
%   \centering
%   \begin{tabular}{cc}
%     \hline
%     経過時間 / min&起電力 / \si{\milli\volt}\\
%     \hline \hline
%     0&1101.97\\
%     1&1101.86\\
%     2&1101.74\\
%     3&1101.63\\
%     4&1101.59\\
%     5&1101.54\\
%     \hline
%     定常値&1101.54\\
%     \hline
%   \end{tabular}
%\end{table}
%\subsection{}