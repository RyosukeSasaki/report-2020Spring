\section{考察}
\subsection{吸光係数の算出}
\label{sec:kyuko}
(\ref{equ:kyukodo})式より,吸光係数$a$は吸光度$A$,溶液層の厚さ$b$ [\si{\centi\meter}],銅イオン濃度$c$ [\si{\gram.L^{-1}}]を用いて以下の式で与えられる.
\begin{align}
  \label{equ:a}
  a=\cfrac{A}{bc}
\end{align}
また,モル吸光係数$\epsilon$は(\ref{equ:a})式の銅イオン濃度$c$をモル濃度にすれば良いので,
\begin{align}
  \label{equ:epsilon}
  \epsilon=\cfrac{A\times63.5}{bc}
\end{align}
となる.但し, $b=1.00$ \si{\centi\meter}である.また,最小二乗法を用いた場合, $b=1.00$ \si{\centi\meter}なので, (\ref{equ:saiyu})式の傾きが吸光係数$a$になる.
表\ref{tab:kyukokeisu}に吸光係数を示す.
\begin{table}[h]
   \caption{吸光係数}
   \label{tab:kyukokeisu}
   \centering
   \begin{tabular}{cc|cc}
     \hline
     銅イオン濃度$c$ / \si{\gram.L^{-1}}&吸光度$A$&吸光係数$a$ / \si{L.\gram^{-1}.\centi\meter^{-1}}&モル吸光係数$\epsilon$ / \si{L.\mole^{-1}.\centi\meter^{-1}}\\
     \hline \hline
     0.106 &  0.099  &  0.934  &  59.3\\
     0.224 &  0.209  &  0.933  &  59.2\\
     0.290 &  0.285  &  0.983  &  62.4\\
     0.410 &  0.390  &  0.951  &  60.4\\
     0.502 &  0.476  &  0.948  &  60.2\\
     \hline
     \multicolumn{2}{c|}{最小二乗法}&0.953&60.5\\
     \hline
   \end{tabular}
\end{table}\\
以上から吸光係数$a=0.950\pm0.020$ \si{L.\gram^{-1}.\centi\meter^{-1}}であった.最小二乗法から得た結果との相対誤差は0.3\%程度となり,よく一致している.
またモル吸光係数$\epsilon$は$a$に対してたかだか整数倍しただけなので,同様によく一致している.
\subsection{試料中の銅の含有率}
\label{sec:ganyu}
表\ref{tab:kekka}より,手書きグラフ,及びExcelグラフから銅の含有率は64.8 \%,及び65.6 \%と得られた.
文献値を65 \%とすると,それぞれ相対誤差は0.3 \%, 0.9 \%であり正確に銅を定量できていると考えられる.
ただし,黄銅の銅含有率には幅があり\cite{kiso},黄銅釘中の銅含有量の真値は不明であるため,どちらがより正確かを議論するのに相対誤差だけでは不適切であると考えられる.
\subsection{手書き,及びExcelグラフの比較}
\ref{sec:kyuko}や\ref{sec:ganyu}での考察から,手書きグラフと最小二乗法によるフィットはどちらも近しい結果を与え,
どちらがより優れているかを判断するにはより正確な文献値(例えば黄銅釘の製造者が発行するデータシートなど)が必要である.

最小二乗法とは観測値と近似曲線との残差の二乗が最小となるようにパラメーターを決定する手法である.
そのため,観測値の持つ誤差の期待値が0であるとき,最小二乗法は最尤曲線を与える.
物理量の測定においては誤差が正規分布であることを想定することから,最小二乗法は実験結果のフィットによく用いられる.
本実験において考えられる誤差として
\begin{enumerate}
  \item 溶液調製での目盛りの読み取り誤差
  \item 分光光度計の誤差
  \item 温度などの偶然誤差
\end{enumerate}
などが考えられる. 1, 3は一般に正規分布を仮定しており,また2についてはベースライン調整を行ったことで,定常的なバイアスを除去している.
このことから本実験においては最小二乗法は最尤曲線を与えていると考えられる.

一方で手書きグラフに近似曲線を書き込む際に以下のことに注意して書き込んだ.
\begin{enumerate}
  \item 原点を通る
  \item 曲線の上下に存在する点の数を等しくする
\end{enumerate}
2は観測値の期待値が0であることを仮定して行った操作であり,最小二乗法のパラメーター推定と本質的に同じ手法である.そのため,これらの結果が近しくなるのは当然であると言える.
\subsection{黄銅中の妨害イオン}
黄銅は0.07 \%未満のFeを含んでおり\cite{kiso},これは妨害イオンになりうる.
この実験では以下の反応が起きていると考えられる.
\begin{align*}
  Fe^{3+} + 3NH_3 +3H_2O \rightarrow Fe(OH)_3 + 3NH_4^{+}
\end{align*}
しかし,試料溶液中の水酸化鉄(III)濃度は$6.8\times10^{-4}$ \si{\mole.L^{-1}}程度と非常に微量であり,
またその色は褐色であり,スペクトルのピークも遠いと考えられるので,影響は微小であると考えられる.
