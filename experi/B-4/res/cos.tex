\subsection{$f(x)=\cos x$の場合}
\subsubsection{結果}
表\ref{tab:cos_res}に正規化したフーリエ係数,位相を示す.
またデジタイザからの入力波形と1次の波形を図\ref{fig:asset/cos/wave1.png},各次数の振幅を図\ref{fig:asset/cos/Amp.png},各次数の位相を図\ref{fig:asset/cos/phase.png}に示す.
\begin{table}[h]
   \caption{余弦波のフーリエ係数,位相}
   \label{tab:cos_res}
   \centering
   \begin{tabular}{ccccc}
     \hline
     次数$n$&$a_n$&$b_n$&$A_n$&$\phi_n$\\
     \hline \hline
     0 & 0.0393 & - & 0.0197 & - \\
1 & 0.9999 & 0.0129 & 1.0000 & 0.0129 \\
2 & -0.0030 & -0.0101 & 0.0106 & -1.8554 \\
3 & -0.0092 & -0.0028 & 0.0096 & -2.8423 \\
4 & 0.0031 & 0.0014 & 0.0034 & 0.4293 \\
5 & 0.0023 & -0.0057 & 0.0062 & -1.1836 \\
6 & -0.0028 & -0.0059 & 0.0065 & -2.0184 \\
7 & 0.0017 & 0.0000 & 0.0017 & -0.0086 \\
8 & 0.0000 & -0.0011 & 0.0011 & -1.5334 \\
9 & 0.0028 & -0.0018 & 0.0033 & -0.5723 \\
10 & -0.0009 & 0.0007 & 0.0011 & 2.4475\\
     \hline
   \end{tabular}
\end{table}
\mfig[width=7cm]{asset/cos/wave1.png}{余弦波の入力波形と1次の波形}
\begin{figure}[htbp]
  \begin{minipage}{0.5\hsize}
    \mfig[width=7cm]{asset/cos/Amp.png}{余弦波の各次数での振幅}
  \end{minipage}
  \begin{minipage}{0.5\hsize}
    \mfig[width=7cm]{asset/cos/phase.png}{余弦波の各次数での位相}
  \end{minipage} 
\end{figure}
\newpage
\subsubsection{考察・課題}
図\ref{fig:asset/cos/Amp.png}からわかるように,このフーリエ級数は1次の項のみが大きな値を持ち,他は非常に小さい.
このことから,位相は1次でのみ意味にある値を持ち,他はランダムな値を持つと考えられる. $\phi_1\simeq0$であり,これは理論値と一致する.