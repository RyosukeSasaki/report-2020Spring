\subsection{うなり}
表\ref{tab:groan_res}に正規化していないフーリエ係数,位相を示す.
またデジタイザからの入力波形と4次の波形を図\ref{fig:asset/groan/wave4.png},入力波形と5次の波形を図\ref{fig:asset/groan/wave5.png},入力波形と6次の波形を図\ref{fig:asset/groan/wave6.png},各次数の振幅を図\ref{fig:asset/groan/Amp.png},各次数の位相を図\ref{fig:asset/groan/phase.png}に示す.
\begin{table}[h]
   \caption{うなり波のフーリエ係数,位相(非正規)}
   \label{tab:groan_res}
   \centering
   \begin{tabular}{ccccc}
     \hline
     次数$n$&$a_n$&$b_n$&$A_n$&$\phi_n$\\
     \hline \hline
     0 & 51.2 & - & 25.6 & - \\
1 & -16.9 & -1.5 & 16.9 & -3.1 \\
2 & 4.4 & -12.8 & 13.5 & -1.2 \\
3 & -0.1 & 3.5 & 3.5 & 1.6 \\
4 & -0.2 & 10.8 & 10.8 & 1.6 \\
5 & 323.7 & -16.4 & 324.1 & -0.1 \\
6 & -302.4 & -88.1 & 314.9 & -2.9 \\
7 & -1.1 & -6.6 & 6.7 & -1.7 \\
8 & -1.6 & 0.6 & 1.7 & 2.8 \\
9 & -2.1 & 3.1 & 3.7 & 2.2 \\
10 & -13.1 & -1.0 & 13.1 & -3.1 \\
     \hline
   \end{tabular}
\end{table}

\begin{figure}[htbp]
  \begin{minipage}{0.5\hsize}
    \mfig[width=7cm]{asset/groan/wave4.png}{うなり波の入力波形と4次の波形}
  \end{minipage}
  \begin{minipage}{0.5\hsize}
    \mfig[width=7cm]{asset/groan/wave5.png}{うなり波の入力波形と5次の波形}
  \end{minipage} 
\end{figure}
\mfig[width=7cm]{asset/groan/wave6.png}{うなり波の入力波形と6次の波形}

\begin{figure}[htbp]
  \begin{minipage}{0.5\hsize}
    \mfig[width=7cm]{asset/groan/Amp.png}{うなり波の各次数での振幅}
  \end{minipage}
  \begin{minipage}{0.5\hsize}
    \mfig[width=7cm]{asset/groan/phase.png}{うなり波の各次数での位相}
  \end{minipage} 
\end{figure}
\subsection{考察・課題}
図\ref{fig:asset/groan/Amp.png}から入力データは5次と6次の成分を主に持っていることがわかる.
実際に4次までの合成波では入力データをほとんど再現していないのに対して, 5次, 6次ではズレが大幅に減少しているのがわかる.

2つの余弦波成分を持つ関数は以下のように与えられる.
\begin{align*}
  f(x)=A_n\cos\left(\cfrac{2\pi}{T}nx-\phi_n\right)+A_m\cos\left(\cfrac{2\pi}{T}mx-\phi_m\right)
\end{align*}
表\ref{tab:groan_res}から$A_n=324$, $A_m=315$, $\phi_n=0$, $\phi_m=-\pi$とした場合の曲線を図\ref{fig:asset/groan/theory.png}に示す.
このように,フーリエ級数から推定した$f(x)$と入力データはよく一致している.
\mfig[width=7cm]{asset/groan/theory.png}{入力データと$f(x)$}