\section{考察}
\subsection{誤差要因}
誤差の要因としては以下のようなものが考えられる.
\begin{itemize}
  \item バイトや旋盤など工具の摩耗による誤差
  \item バイト・材料の振動による誤差
  \item 材料の熱膨張による誤差
\end{itemize}
特にバイトや材料の振動はチャックの掴み長さが短い場合や固定が不十分な時に発生すると考えられる.
更に,切り込み量を大きくすると材料や工具の受ける応力が大きくなり,振動が生じると考えられる.
また,材料の熱膨張は材料の硬さなどにもよるが,一般に回転速度が速いほど摩擦が大きくなり発熱も大きくなると考えられる.
また,切削油やクーラントが不足している場合,同様に発熱が大きくなると考えられる.
\subsection{切削面,切りくずと切削条件の関係}
切削速度$v$ \si{\metre.\min^{-1}}は以下のように与えられる.\cite{unti}
\begin{align*}
  v=\cfrac{\pi Dn}{1000}
\end{align*}
ここで$n$は回転速度(rpm), $D$は工作物の直径(mm)である.
バイトを同じ速度で送ったとき,切削速度が大きいほど最大高さ粗さ$Rz$は小さくなる.\cite{rikougaku}
したがって回転数を大きくし,切削速度を上げることで切削面の粗さは良好になる.
しかし,切削速度を上げた場合,摩擦で温度が上昇し,バイトの寿命が減少するので,
通常は各材料やバイトごとにバイトの寿命が60分から100分となるような速度で加工を行う.
これを経済的切削速度と呼ぶ.\cite{unti}

切りくずは主に3種類に分類され,長く連続した流れ型,短いせん断型,被削材の表面をむしりながら発生するむしり型がある.\cite{197674}
特にむしり型は被削材を切り込み量より深く削っているため,切削面の状態が悪化している.
また,流れ型の場合,切削抵抗が常に一定になるため,切削面の状態は良い.
切り込み量を大きくすると切り屑は短くなり,切り込み量を小さくすると細く長い切りくずが発生する.\cite{osikko}
このことから,切り込み量を小さくするとより切削面の状態が良くなるとわかる.
したがって,仕上げ加工で切り込み量を小さくするのは切削面の状態を良くする点で理にかなっている.
\subsection{振動する切削条件}
旋盤加工での振動はビビリという甲高い騒音としてあらわれる.\cite{kuso}
振動が発生する原因としては背分力の増加がある.
背分力とは,切削抵抗のうち回転軸のラジアル方向に掛かる分力で,この影響で刃物または材料が振動する.
背分力が大きくなる条件は以下のようなものが挙げられる.
\begin{itemize}
  \item 切り込み量の増加\cite{whatis}
  \item 横切れ刃角の増加\cite{kuso}
\end{itemize}
切り込み量を増加するとその分切削抵抗が増加するため,必然的に振動もしやすくなる.\cite{whatis}
また,図\ref{fig:sebun.png}のように刃が材料に当たる角度,すなわち横切れ角が大きくなると,切削抵抗における背分力の成分が増加し,振動もしやすくなる.
また,材料が細長く,チャックから長く出ている場合,これも振動しやすい.
\mfig[width=10cm]{sebun.png}{背分力と横切れ刃角}
\subsection{切削条件}
旋盤の切削条件には上記で述べた切削速度$v$やワーク1回転あたりの工具の移動量(\si{\micro\metre})を表す送りがある.
切削速度$v$を大きくすると,送りが同じなら加工時間を短くできるが,その分発熱量も大きくなり,熱変形による誤差の増大や工具寿命が短くなることが考えられる.
送りを小さくすると最大高さ粗さ$Rz$は小さくなるが,加工時間が長くなりトレードオフである.
また,ワークやバイトの材質,切削油の状態も切削条件となる.
\subsection{工作機械の工夫}
工作機械には加工の正確さを保つため以下のような工夫がなされている.
\begin{itemize}
  \item 往復台のように平行に動作する面は非常に高い平面度を持ち,動作させた際のブレが小さい.
  \item 可動部は十分に潤滑油で潤滑しておくことで,長期の使用によって可動部が摩耗し,精度が低下することを防いている.
  \item 工作機械は外部からの振動の影響を小さくするため,強固な床に設置される.
  \item 加工による材料やバイトの加熱は加工誤差を引き起こすので,場合によってはクーラントなどにより冷却を行う物もある.
\end{itemize}
\subsection{加工精度を向上する方法}
加工精度を向上する方法として以下のようなものが考えられる.
\begin{itemize}
  \item 仕上げ加工の際には,定期的に切削部の寸法を測定し,少しずつ加工する.
  \item クーラントを用いる.
  \item 送り速度を小さくする.
  \item 長いワークを加工する際はセンタを用いる.
  \item 嵌る対象がある場合は現物合わせを行う.
\end{itemize}
\subsection{切削油の役割}
切削油には以下のような役割がある.
\begin{itemize}
  \item バイトと材料の間を潤滑し,切削抵抗を減らす.
  \item バイトと材料を冷却し,熱による誤差を抑える.
  \item 切りくずを排出する.
  \item 工具の摩耗を抑制する.
  \item 錆を抑制する.\cite{12}
\end{itemize}