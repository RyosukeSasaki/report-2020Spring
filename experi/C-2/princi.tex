\section{実験原理}
\subsection{ガスクロマトグラフ法の原理}
試料を溶融する不揮発性液体Aを固体粒子に塗布し金属管に詰める.これを分離カラムと呼ぶ.
カラムは恒温槽に導入されており,注入された試料は直ちに揮発する.
カラムには常に一定の流速でAに溶解しないガスC(キャリアガス)が流れている.\ A,\ Cをそれぞれ固定相,移動相と呼ぶ.
ここで試料注入口から少量の試料Bを注入すると,Bの蒸気は直ちにAに溶解する.
しかし直後にBを含まないキャリアガスに接するためBは直ちに揮発し移動するが,わずかに進んだところで再びAに溶解する.
Bはこれを繰り返しながら分離カラムから脱出し,検出器によって検出される.
\subsection{検出器}
検出器としては熱伝導検出器(TCD)が一般に用いられる.TCDはホイートストンブリッジ回路の2つの抵抗をそれぞれキャリアガス,キャリアガス+試料蒸気で
冷却することで構成される.2つの抵抗の間に温度差が生じると電位差が生じ,これを測定することで試料を検出する.
\subsection{定性分析}
試料に溶解度の異なる物質$B_1,B_2,B_3...$が含まれていたとすると,
これらは異なる速度で溶解,揮発を繰り返すため分離カラム内で分離していく.結果$B_1,B_2,B_3...$は異なる時間(保持時間)に分離カラムから脱出する.
保持時間は物質固有であり,保持時間が既知なら未知物質を特定できる.
\subsection{定量分析}
電圧のピーク面積は成分の物質量と物質自体に依存する.このことから絶対検量線法,補正面積百分率法などの手法で定量分析を行える.
絶対検量線法とは,物質量,ピーク面積をプロットすることで検量線を作成し,これを用いて物質量を直接算出する方法である.

補正面積百分率法とは,組成が既知な試料を用いて,各成分のピーク面積について標準物質のピーク面積との補正係数$F_i$を(\ref{equ:Fi})式で計算し\cite{rikougaku},
(\ref{equ:molper})式を用いて組成モル比を計算する手法である.
\begin{equation}
  \label{equ:Fi}
  F_i=\frac{標準物質のピーク面積}{成分iのピーク面積}\times\frac{成分iの物質量}{標準物質の物質量}
\end{equation}
\begin{equation}
  \label{equ:molper}
  成分iのモル比(\%)=\frac{A_i F_i}{\sum_i A_i F_i}\times100
\end{equation}