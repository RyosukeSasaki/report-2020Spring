\section{実験方法}
\subsection{光学系のセットアップ}
光学台に鏡$\mathrm{M_1}$を取り付け,光軸を平行に調整した.
次にビームスプリッタ$\mathrm{G_1}$を取り付け,スクリーン上で光路(i), (ii)の像が一致するように調整した.
また干渉縞を見やすくするため,パラフィン紙でレーザーを拡散させた.
\subsection{実験1 --- 波長の測定}
ステージコントローラを用いて鏡$\mathrm{M_1}$を動かし,干渉縞を$x$回,明$\rightarrow$暗$\rightarrow$明と変化させた.
そのときの$\mathrm{M_1}$の移動量$d$から光源の波長を計算する.
ただし,送りねじのバックラッシ等により最初は$\mathrm{M_1}$が動かないので,
しばらくステージコントローラを動かして,安定して$\mathrm{M_1}$が動くようになってから原点を指定し,測定した.
\subsection{実験2 --- 屈折率の測定}
光学台に光学セルを設置し,マロメーター,ポンプを接続する.まずポンプを用いて光学セル内の圧力を高め,
次にポンプの調節弁を開けて明暗の変化を数えながら圧力を抜いていった.
開始時の圧力と終了時の圧力,明暗が変化した回数から屈折率を計算する.
ただし,光学セルには$200~\si{\mmHg}$以上の圧力を掛けてはいけない.