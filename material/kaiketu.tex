\section{解決方法}
\subsection{オゾン層破壊}
特定フロンはモントリオール議定書により, CFCは全廃, HCFCの段階的な削減が定められている.
これにより,オゾン量の減少は止まったものの回復には数十年を要する.
現在ではスプレーなどの噴射剤として, $\mathrm{CO_2}$やジメチルエーテルなどが用いられている.
また冷媒としては, HFCなどの代替フロンが主に用いられているがこれは前述の通り温室効果があり,
2008年のキガリ改正では代替フロンを含むフロンガスの削減が採択され,日本を含む先進国は2036年までに85\%の削減を行う\cite{ref015pd42:online}.
\mfig[width=8cm]{dme.jpg}{ $\mathrm{CO_2}$,ジメチルエーテルを噴射剤として用いたスプレー}
\subsection{温室効果}
モントリオール議定書のキガリ改正により,温室効果の高い代替フロンの削減が求められている.
代替フロンの代替としては以下のような物質が検討されている\cite{代替フロン(HF79:online}.
\begin{description}
  \item[ハイドロフルオロオレフィン(HFO)]\mbox{}\\
  GWPが0.3程度\cite{AGC}と温室効果が低く,冷媒としての利用が期待される.一方で高コストであり微燃性がある.
  \item[二酸化炭素($\mathrm{CO_2}$)]\mbox{}\\
  冷媒として利用可能だが,高圧が必要である. 
  \item[アンモニア($\mathrm{NH_3}$)]\mbox{}\\
  冷媒として利用可能だが,毒性が高い.   
\end{description}
\section{社会的影響}
前節で述べた代替ガスはいずれも毒性または可燃性であり,また比較的可燃性の低いHFOはコストが高い.
毒性,可燃性のあるガスを冷媒として用いる場合,機構的に複雑になるため,いずれの代替冷媒を用いても最終的な製品価格の上昇が考えられる.

フロンが全廃され,GWPが1程度の冷媒で代替されたとする.このとき,フロンの平均のGWPを1000とするならば, $2.847\times10^6$ \si{\tonne}-$\mathrm{CO_2}$程度の温室効果ガスを削減できることになる.
しかし,日本の$\mathrm{CO_2}$排出量は$1.115\times10^9$ \si{\tonne}程度であり,メタンなど他の温室効果ガスが排出されていることも考えると,代替フロンの全廃だけでは温暖化の防止には不足であると考えられる.

一方で,オゾン量の減少は現時点で止まっており,このまま特定フロンの削減が進めばいずれオゾン層は回復すると考えられる.