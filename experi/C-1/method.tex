\section{実験方法}
実験には以下の試料,試薬を用いた.
\begin{table}[h]
   \caption{使用試料及び試料}
   \label{tab:siryou}
   \centering
   \begin{tabular}{l}
     \hline
     名称\\
     \hline \hline
     $\rm Cu^{2+}$標準溶液($\rm Cu^2+$ 5.00 \si{\gram.L^{-1}})\\
     7.5M $\rm NH_4OH$\\
     6M $\rm HNO_3$\\
     黄銅釘\\
     \hline
   \end{tabular}
\end{table}
\subsection{溶液の調製}
$\rm Cu^{2+}$標準溶液をメスピペットで0.5, 1.0, 1.5, 2.0, 2.5 \si{\milli L}ずつはかり取り,
25 \si{\milli L}メスフラスコに入れた.この際,メスピペットのはじめの読みと終わりの読みを小数点以下第2位
まで記録し,この差ではかり取った量を算出した.同時に各溶液の色を記録した.各溶液を濃度が低い順に①,②,③,④,⑤とした.
次に7.5M $NH_4OH$をホールピペットを用いて10 \si{\milli L}ずつ加えた.この際にも色の変化を記録した.
その後ある程度振り混ぜてから標線まで水を加え,十分に振り混ぜた.また,別のメスフラスコに7.5M $NH_4OH$を10 \si{\milli L}入れ,
標線まで水を入れ,よく振り混ぜた.これを対照液とした.
各溶液の濃度は以下の式で与えられる.
\begin{align*}
  濃度 [\si{\gram.L^{-1}}] = \cfrac{標準溶液の体積 [\si{L}]\times 5.00\ \si{\gram.L^{-1}}}{\cfrac{25}{1000}}
\end{align*}
\subsection{最大吸収波長の決定}
共洗いを行ってから, 1Aセルに対照液, 1Bセルに⑤溶液を入れた.セルに気泡や汚れがないことを確認し,また表面の水滴はキムワイプで拭き取った.
まず1Aセルをすりガラス面を持って分光光度計にセットしベースライン補正を行った.但し,セルは毎回必ず同じ向きにセットする.
次に1Bセルを同様に分光光度計にセットし,スペクトラムモードで測定を行い,ピーク波長を記録した.
\subsection{検量線の作成}
\label{sec:met_kenryo}
①から⑤の溶液に対してフォトメトリックモードで測定を行った.測定毎にセルは洗浄,共洗いし,セルは同じ向きにセットした.
溶液の濃度を横軸,吸光度Aを縦軸としてプロットすることで検量線を作成した.
\subsection{黄銅中の銅の定量}
デシケーター内の乾燥したビーカーを電子天秤で0.1 \si{\milli\gram}単位で精秤した.
これに黄銅釘を入れ,同様に精秤し,釘の重量を算出した.
これに6M $HNO_3$を3 \si{\milli L}入れ,電熱器を用いて加熱溶解させた.
これに約20 \si{\milli L}の水を加え, 100 \si{\milli L}メスフラスコに移し,またビーカーの洗液も合わせた.
更に定期的に振り混ぜながら標線まで水を加え,最後に更によく振り混ぜた.
次にこの溶液を10 \si{\milli L}取り, 25 \si{\milli L}のメスフラスコで 10\si{\milli L}の7.5 M $NH_4OH$と合わせ発色させた.
標線まで水を加え,よく振り混ぜた後に\ref{sec:met_kenryo}と同様に吸光度を測定した.

