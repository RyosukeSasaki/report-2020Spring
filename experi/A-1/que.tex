\subsection{課題1}
積分器の回路方程式において$u(t)=A\sin{\omega t}$, $y(t)=\mathrm{e}^{i(\omega t+\delta)}$とすると
\begin{align}
  \cfrac{dy(t)}{dt}&=-\cfrac{1}{RC}u(t)\\
  i\omega A\mathrm{e}^{i(\omega t+\delta)}&=-\cfrac{E}{RC}\mathrm{e}^{i(\omega t -\pi/2)}\\
  A\mathrm{e}^{i(\omega t+\delta)}&=-\cfrac{E}{RC\omega}\mathrm{e}^{i(\omega t -\pi)}\\
  \therefore~A&=\cfrac{E}{RC\omega}=\cfrac{E}{RC(2\pi f)}
\end{align}
以上から周波数応答は
\begin{align}
  H(f)=20\log_{10}\cfrac{1}{RC(2\pi f)}
\end{align}
となる.図\ref{fig:que/graph1.eps}にボード線図の理論曲線を示す.ただし$R=10~\si{\kilo\ohm}$, $C=0.01~\si{\micro\farad}$とした.図\ref{fig:que/graph1.eps}から,積分器の周波数特性は線形であるとわかる.
\mfig[width=12cm]{que/graph1.eps}{ボード線図}
\subsection{課題2}
図\ref{fig:que/que2.png}に微分器の回路図を示す.
微分器の回路方程式は
\begin{align}
  \cfrac{y(t)}{R}&=-C\cfrac{du(t)}{dt}\\
  y(t)&=-RC\cfrac{du(t)}{dt}
\end{align}
であり,確かに入力の微分が出力されている.
また$u(t)=E\sin\omega t$とすると
\begin{align}
  y(t)&=-RC\cfrac{d}{dt}E\sin\omega t\\
  &=-RCE\omega\sin(\omega t-\cfrac{\pi}{2})\\
  \therefore~A&=RCE\omega=RCE(2\pi f)
\end{align}
したがって,微分器の周波数応答は周波数に比例してゲインが大きくなる.微分機のボード線図は図\ref{fig:que/graph2.eps}のようになる.
微分器のボード線図と積分器のボード線図は$R$と$C$が等しいとき$y=0$に対して線対称である.
\mfig[width=6cm]{que/que2.png}{回路図}
\mfig[width=12cm]{que/graph2.eps}{積分器と微分器のボード線図}
\subsection{課題3}
表\ref{tab:jouken}に理論曲線の条件を示す.
また図\ref{fig:que/graph3.eps}に各条件でのボード線図を示す.
図\ref{fig:que/graph3.eps}から,$R$と$r$が異なるとき,定常ゲインが異なることがわかる.
また$r$が異なっていても$R$が等しければ同じ直線に漸近することがわかる.このことは
\begin{align}
  G(f)&=20\log_{10}\cfrac{r/R}{\sqrt{1+(2\pi frC)^2}}\\
  &=20\log_{10}\cfrac{1}{R\sqrt{\cfrac{1}{r^2}+(2\pi fC)^2}}\\
  &=20\log_{10}\cfrac{1}{Rf\sqrt{\cfrac{1}{r^2f^2}+(2\pi C)^2}}\\
  &\rightarrow20\log_{10}\cfrac{1}{2\pi RfC}
\end{align}
となることからも明らかである.
\begin{table}[h]
  \caption{各場合の条件}
  \label{tab:jouken}
  \centering
  \begin{tabular}{ccc}
    \hline
    場合&$R$ / \si{\kilo\ohm}&$r$ / \si{\kilo\ohm}\\
    \hline \hline
    1&10&20\\
    2&20&10\\
    実験2&10&10\\
    追加実験&20&20\\
    \hline
  \end{tabular}
\end{table}
\mfig[width=12cm]{que/graph3.eps}{各条件でのボード線図(理論値)}