\documentclass[uplatex,11pt,fleqn]{jsarticle}

\usepackage{amsmath}
\usepackage{mathtools}

\begin{document}

\noindent
\mbox{関数の平均より明らかに,}
\[
  a_0=\frac{1}{2}
\]
\mbox{また,}
\[
 b_n=\frac{1}{\pi}\int_{-\pi}^\pi f(x)\cos(nx)dx=0
\]
\begin{align}
 a_n=\frac{1}{\pi}\int_{-\pi}^\pi f(x)\sin(nx)dx
 &=\frac{1}{\pi} \left\{ \int_{-\pi}^0 0dx+\int_0^\pi \sin(nx)dx \right\} \nonumber \\
 &=-\frac{1}{n\pi} \Bigl[cos(n\pi)-1\Bigr] \nonumber \\ 
 &=\begin{dcases}
   \frac{2}{n\pi} & (n:odd) \nonumber \\
   0 & (n:even)
 \end{dcases}
\end{align}
\mbox{よって,自然数} $ k $ \mbox{を用いて}
  \[
f(x)=\frac{1}{2}+\sum_{k=1}^{\infty}\frac{2}{(2k-1)\pi}\sin((2k-1)\pi)
\]
\mbox{となる.}

\end{document}