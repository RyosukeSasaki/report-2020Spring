\section{課題2}
\subsection{符号分割多重\cite{4D69637228:online}}
\mfig[width=12cm]{CDM.png}{符号分割多重}
符号分割多重とは多重化の方式の一つであり,スペクトラム拡散の応用である.
図\ref{fig:CDM.png}に示すようにあるNRZ符号化されたデータに対し,
ある符号を乗算し信号を拡散する.この符号をCDM符号とよび,符号の長さをチップと表す.
複数の通信ごとに直交したCDM符号を割り当て,各通信ごとの拡散された信号を加算して送信する.
CDM符号同士が直交しているため,
受信者側は自分に割り当てられたCDM符号と受信信号の内積を取ることで目的のデータを取り出せる.
スペクトラム拡散と同じように周波数ホッピング(図\ref{fig:hop.png})と直接拡散(図\ref{fig:direct.png})方式があり,
それぞれでCDM符号はホッピングパターンと拡散符号に相当する.
\begin{figure}[htbp]
  \begin{minipage}{0.5\hsize}
      \mfig[width=7cm]{hop.png}{周波数ホッピング方式}
  \end{minipage}
  \begin{minipage}{0.5\hsize}
      \mfig[width=7cm]{direct.png}{直接拡散方式}
  \end{minipage} 
\end{figure}
\newpage