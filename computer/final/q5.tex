\section{課題5}
図\ref{fig:q5.png}のようにNORゲートにA, Bと,その入力にY, Y', X, X'と名前をつける.
YとXはACTIVE HIGHとする.

まずYを1にすることを考える.
Yが1になるとAの出力Qは0になり, X'も0になる.
このときXが0であるのでBの出力$\mathrm{\overline{Q}}$は1になる.
したがってY'は1になり, 出力Qは0のままである.
以上から入力YはRESET信号に相当する.

次にXを1にすることを考える.
Xが1になるとBの出力$\mathrm{\overline{Q}}$は0になり, Y'も0になる.
このときYが0であるのでAの出力Qは1になる.
したがってX'は1になり,出力$\mathrm{\overline{Q}}$は0のままである.
以上から入力XはSET信号に相当する.

またYとXはACTIVE HIGHなので両方がLOWのとき状態は変化しない.
そしてYとXが両方HIGHのとき,出力Qと$\mathrm{\overline{Q}}$は同時に0になるのでこれは禁止状態である.
以上からこの回路の動作は表\ref{tab:q5}のようになる.
また図\ref{fig:q5_wave.png}にX, Yの入力に対する出力Qの波形を示す.
\mfig[width=4cm]{q5.png}{回路図}
\begin{table}[h]
\caption{回路の動作}
\label{tab:q5}
\centering
\begin{tabular}{cccc}
\hline
X&Y&Q&$\mathrm{\overline{Q}}$\\
\hline \hline
L&H&\multicolumn{2}{c}{リセット}\\
H&L&\multicolumn{2}{c}{セット}\\
H&H&\multicolumn{2}{c}{禁止状態}\\
L&L&\multicolumn{2}{c}{前の状態}\\
\hline
\end{tabular}
\end{table}
\mfig[width=12cm]{q5_wave.png}{入力波形と出力波形}