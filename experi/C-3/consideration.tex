\section{考察}
\subsection{課題3}
両方の実験を通して,実験の後半では映像が目盛りを上から覗き込む形になっており,これによって後半の滴下量データが正のバイアスを持っている可能性がある.
また,示差曲線を用いる場合,当量点付近で
\subsubsection{実験3について}
表\ref{tab:hikaku1}に滴定曲線,示差曲線から求めた中和当量,$f_{NaOH}$,pHを示す.
\begin{table}[htbp]
   \caption{実験3における各手法の比較}
   \label{tab:hikaku1}
   \centering
   \begin{tabular}{lccc}
     \hline
     &中和当量/\si{\milli L}&$f_{NaOH}$&pH\\
     \hline \hline
     滴定曲線&11.00&$9.12\times10^{-1}$&6.8\\
     示差曲線&10.03&$1.00$&4.9\\
     \hline
   \end{tabular}
\end{table}\\
手法によって中和当量は10\%, $f_{NaOH}$は8\%程度の相対誤差が出ている.また当量点付近のため中和当量の僅かな差でもpHには大きな差が生じている.塩酸,水酸化ナトリウムは強酸,強塩基であり,pHの変化は非常に急激である.
故に実験3において示差曲線を用いるのは不適切であり,滴定曲線を用いたデータがより真に近いと考えられる.実際に滴定曲線により得た当量点のほうがpH7に近く,
より真の当量点に近いと考えられる.
\subsubsection{実験4について}
表\ref{tab:hikaku2}に滴定曲線,示差曲線から求めた中和当量, $K_A$, pHを示す.
\begin{table}[htbp]
   \caption{実験4における各手法の比較}
   \label{tab:hikaku2}
   \centering
   \begin{tabular}{lccc}
     \hline
     &中和当量/\si{\milli L}&$K_A$&pH\\
     \hline \hline
     滴定曲線&10.39&$2.95\times10^{-5}$&8.10\\
     示差曲線&10.43&$2.88\times10^{-5}$&8.13\\
     \hline
   \end{tabular}
\end{table}\\
手法によって中和当量は0.4\%, 2\%程度の相対誤差が出ている.酢酸,水酸化ナトリウムは弱酸,強塩基であり,実験3に比べてpHの変化は緩やかになる.
したがって実験3に比べて手法間での誤差は小さくなり, pHもほぼ同じ値を示したと考えられる.
一方で図\ref{fig:sakusan_sisa.png}を見れば明らかなように,示差曲線のピーク付近はデータ数が少なく,数式によるフィットやサンプル数を増やすことで更に精度を高められると考えれれる.

しかし,実験で得られた値は文献値\cite{rikougaku}と大きく異なっている.pHメーターの示す温度は実験中21\si{\degreeCelsius}程度だったが,一般に温度が下がれば電離定数は減少するため
結果と矛盾する.この原因は5.3 課題2で言及するようにpHメーターの測定値に原因があると考える.
\subsection{課題1}
図\ref{fig:pp_henka.png}にフェノールフタレイン溶液の構造の変化を示す.
\mfig[width=7cm]{pp_henka.png}{フェノールフタレインのpHによる構造の変化\cite{Shigeki}}
フェノールフタレイン溶液はpHが一定以上になるとヒドロキシ基の水素が離脱し,また塩基により環状エステル(ラクトン環)が加水分解される.これによりフェノールフタレインは$\pi$共役系分子になる. $\pi$共役系分子ではHOMO - LUMO間のエネルギーギャップが小さく,吸収・放射スペクトルが可視光領域に収まるため呈色する.\cite{Muranaka}
一方で図\ref{fig:hidari.png}の左の分子はヒドロキシ基が存在せず,pHの変化により水素が離脱しないため,構造が変化せず,試薬としては使えないと思われる.
また,右の分子はラクトン環が存在せず,図\ref{fig:kakenai.png}のように安定な極限構造式を描けないためフェノールフタレインのような安定した$\pi$共役系分子になれず,発色しないと考えられる.
\mfig[width=10cm]{hidari.png}{課題の分子\cite{rikougaku}}
\mfig[width=10cm]{kakenai.png}{極限構造式は描けない}
\subsection{課題2}
酢酸水溶液が平衡状態にあるとき,水の電離を無視して解離度を$\alpha$,初期濃度を$c$として以下が成り立つ.
\begin{table}[h]
   \centering
   \begin{tabular}{ccccc}
     \hline
     $[CH_3COOH]$&$\rightleftharpoons$&$[H^+]$&+&$[CH_3COO^-]$\\
     \hline
     $c$&&$0$&&$0$\\
     $c(1-\alpha)$&&$c\alpha$&&$c\alpha$\\
     \hline
   \end{tabular}
\end{table}\\
よって以下が成り立つ.
\begin{equation}
  \label{equ:KAB}
  K_A=\cfrac{[H^+][CH_3COO^-]}{[CH_3COOH]}=\cfrac{(c\alpha)^2}{c(1-\alpha)}
\end{equation}
ここで$[H^+]=10^{-pH}$なのでpHと酢酸水溶液の初期濃度から$K_{A_B}$を求められる.
初期濃度は実験4で滴定曲線,示差曲線から得られた物の平均値$\cfrac{0.947+0.951}{2\times20}=4.75\times10^{-2}M$を用いる.最初のpHは3.14なので,
\begin{align*}
  K_{A_B}&=\cfrac{(10^{-3.14})^2}{4.75\times10^{-2}-10^{-3.14}}\\
  &=1.12\times10^{-5}
\end{align*}
この式で用いられている測定値は$C_{酢酸}$, pHのみであり,このどちらか,あるいは両方に誤差があると考えられる.
仮にpHメータの値にバイアスが掛かっていたとして, $C_{酢酸}$は$f_{NaOH}$を含めpHに依存せず当量点での滴下量のみから求められた.
故に$C_{酢酸}$の値が狂っているとは考えにくく,必然的に誤差要因はpHメーターにあると考えられる.