\subsection{実験2}
図\ref{fig:e2/c2.png}に実験2の回路図を示す.
回路方程式は
\begin{align}
  rC\cfrac{dy(t)}{dt}+y(t)=-\cfrac{r}{R}u(t)
\end{align}
となる.ここで$u(t)=E\sin\omega t$とすると
\begin{align}
  rC\cfrac{dy(t)}{dt}+y(t)=-\cfrac{r}{R}E\sin\omega t
\end{align}
ここで$\sin\omega t=\mathrm{Re}(\mathrm{e}^{i(\omega t-\pi/2)})$である.また$y(t)=A\mathrm{e}^{i(\omega t+\delta)}$とすると
\begin{align}
  &rCi\omega A\mathrm{e}^{i(\omega t+\delta)}+A\mathrm{e}^{i(\omega t+\delta)}=-\cfrac{r}{R}E\mathrm{e}^{i(\omega t-\pi/2)}\\
  &A\mathrm{e}^{i(\omega t+\delta)}=-\cfrac{r/R}{1+i\omega rC}E\mathrm{e}^{i(\omega t-\pi/2)}\\
  &y(t)=-\cfrac{r/R}{1+(\omega rC)^2}E\mathrm{e}^{i(\omega t-\pi/2)}(1-i\omega rC)\\
  &y(t)=-\cfrac{r/R}{1+(\omega rC)^2}E\mathrm{e}^{i(\omega t-\pi/2)}\sqrt{1+(\omega rC)^2}\mathrm{e}^{-i\arctan(\omega rC)}\\
  &y(t)=-\cfrac{r/R}{\sqrt{1+(\omega rC)^2}}E\mathrm{e}^{i(\omega t-\pi/2-\arctan(\omega rC))}\\
  &y(t)=-\cfrac{r/R}{\sqrt{1+(\omega rC)^2}}E\sin(\omega t-\arctan(\omega rC))
\end{align}
となり,これが定常解である.
\mfig[width=6cm]{e2/c2.png}{回路図}
\subsubsection{結果}
図\ref{fig:e2/graph1.eps}に実験2のボード線図を示す.
表\ref{tab:e2_outou}に周波数応答を示す.
\mfig[width=12cm]{e2/graph1.eps}{ボード線図}
\begin{table}[h]
  \caption{周波数応答}
  \label{tab:e2_outou}
  \centering
  \begin{tabular}{ccccc}
    \hline
    入力周波数$f$&入力電圧 / \si{\volt}&出力電圧 / \si{\volt}&ゲイン(実験値) / \si{\deci\bel}&ゲイン(理論値) / \si{\deci\bel}\\
    \hline \hline
    $1.00\times10^2$ & 0.682 & 0.668 & $-1.80\times10^{-1}$ & $-1.71\times10^{-2}$ \\
    $2.00\times10^2$ & 0.682 & 0.662 & $-2.59\times10^{-1}$ & $-6.80\times10^{-2}$ \\
    $3.00\times10^2$ & 0.682 & 0.658 & $-3.11\times10^{-1}$ & $-1.52\times10^{-1}$ \\
    $4.00\times10^2$ & 0.682 & 0.650 & $-4.17\times10^{-1}$ & $-2.66\times10^{-1}$ \\
    $5.00\times10^2$ & 0.682 & 0.638 & $-5.79\times10^{-1}$ & $-4.09\times10^{-1}$ \\
    $1.00\times10^3$ & 0.682 & 0.567 & $-1.60\times10^{0}$ & $-1.45\times10^{0}$ \\
    $2.00\times10^3$ & 0.682 & 0.423 & $-4.15\times10^{0}$ & $-4.11\times10^{0}$ \\
    $3.00\times10^3$ & 0.681 & 0.327 & $-6.37\times10^{0}$ & $-6.58\times10^{0}$ \\
    $4.00\times10^3$ & 0.681 & 0.260 & $-8.36\times10^{0}$ & $-8.64\times10^{0}$ \\
    $5.00\times10^3$ & 0.681 & 0.218 & $-9.89\times10^{0}$ & $-1.04\times10^{1}$ \\
    $1.00\times10^4$ & 0.683 & 0.110 & $-1.59\times10^{1}$ & $-1.61\times10^{1}$ \\
    $2.00\times10^4$ & 0.683 & 0.056 & $-2.17\times10^{1}$ & $-2.20\times10^{1}$ \\
    $3.00\times10^4$ & 0.684 & 0.037 & $-2.53\times10^{1}$ & $-2.55\times10^{1}$ \\
    $4.00\times10^4$ & 0.683 & 0.028 & $-2.77\times10^{1}$ & $-2.80\times10^{1}$ \\
    $5.00\times10^4$ & 0.684 & 0.022 & $-2.99\times10^{1}$ & $-2.99\times10^{1}$ \\
    $1.00\times10^5$ & 0.692 & 0.011 & $-3.60\times10^{1}$ & $-3.60\times10^{1}$ \\
    \hline
  \end{tabular}
\end{table}
\subsubsection{考察}
図\ref{fig:e2/graph2.eps}にボード線図の理論値と実験値を示す.
図\ref{fig:e2/graph2.eps}のように,ゲインは理論値と実験値でよく一致している.
図\ref{fig:e2/graph1.eps}より,カットオフ周波数は$1480~\si{\hertz}$であった.
一方,カットオフ周波数$f_B$は以下の式で与えられる.
\begin{align}
  f_B=\frac{1}{2\pi rC}\sqrt{\frac{2r^2}{R^2}-1}
\end{align}
したがって$f_B=1590~\si{\hertz}$となる.相対誤差は$6.9~\%$でありよく一致している.
\begin{table}[h]
  \caption{カットオフ周波数$f_B$の理論値と実験値}
  \label{tab:}
  \centering
  \begin{tabular}{cc}
    \hline
    カットオフ周波数(実験値) / \si{\hertz}&カットオフ周波数(理論値) / \si{\hertz}\\
    \hline \hline
    1480&1590\\
    \hline
  \end{tabular}
\end{table}
\mfig[width=12cm]{e2/graph2.eps}{ボード線図(理論値と実験値)}