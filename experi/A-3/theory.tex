\section{理論}
\subsection{応力}
物体に荷重を加えると,それに釣り合うように物体内に内力が発生する.
この内力の単位面積あたりの量が応力である.
図のように内力の方向と作用面(面積$F_0$)が垂直なとき,これを垂直応力と呼び,その値は
\begin{align}
  \sigma=\cfrac{P}{F_0}
\end{align}
である.
一方,図のように内力の方向と作用面が水平なとき,これをせん断応力と呼び,その値は
\begin{align}
  \tau=\cfrac{P}{F_0}  
\end{align}
である.
\subsection{ひずみ}
荷重によって物体が変形した際,その単位量あたりの量をひずみと呼ぶ.
図のように内力の方向と作用面が垂直なとき,垂直ひずみと呼び,その値は
\begin{align}
  \epsilon=\cfrac{\lambda}{l_0}  
\end{align}
である.
一方,図のように内力の方向と作用面が水平なとき,これをせん断ひずみと呼び,その値は
\begin{align}
  \epsilon=\cfrac{S}{l_0}  
\end{align}
である.
\subsection{応力-ひずみ線図}
図に軟鋼の応力-ひずみ線図を示す.
応力が比例限度$P$より小さい範囲で応力とひずみは比例関係にある.
応力がある値以上になると,材料は塑性変形し,その荷重の限界値を弾性限度$E_{\epsilon}$と呼ぶ.
さらに応力を増やしていくと,応力が一定のままひずみが突然増大する.
この現象を降伏と呼び,降伏が始まる点を降伏点$A$と呼ぶ.
その後,応力は最大値を取り,徐々に減少し破断する.
応力の最大値を引張強さ$M$とよび,破断する点を破断強さ$C$とよぶ.
応力ひずみ線図は材料によって異なった経過を示す.
\subsection{ひずみゲージ}
ひずみゲージとは,材料に貼り付け,材料と共にひずむことで材料のひずみを正確に測定する素子である.
電気抵抗線ひずみゲージでは,針金のひずみに伴う電気抵抗の変化により,ひずみを測定する.
針金の長さを$l$,抵抗を$R$とし,ひずみに伴いこれが$\Delta l$, $\Delta R$変化したとき,ゲージ率$\alpha_0$は
\begin{align}
  \alpha_0=\cfrac{\frac{\Delta R}{R}}{\frac{\Delta l}{l}}
\end{align}
で表され,一般には$\alpha_0\simeq 2.0\sim 2.1$である.